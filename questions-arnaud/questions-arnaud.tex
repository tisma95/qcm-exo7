
\section*{QCM de révisions (Arnaud)}

\textit{Répondre en cochant la ou les cases correspondant à
des assertions vraies (et seulement celles-ci).}

\subsection*{Logique}

\begin{question}
Soit l'équation $E : x^n=27$.
\begin{answers}
    \bad{$E$ a une unique  solution réelle quel que  soit $n \ge 1$.}

    \good{$E$ a au moins une solution réelle quel que  soit $n \ge 1$.}

    \bad{$E$ a $n$ solutions réelles quel que  soit $n \ge 1$.}

    \good{$E$ a au moins $n$  solutions complexes quel que  soit $n \ge 1$.}

    \good{$E$ a exactement $n$ solutions complexes quel que  soit $n \ge 1$.}    
\end{answers}
\end{question}




\begin{question}
Soit $f:\Rr \to \Rr$, $x \mapsto x^2+1$.
\begin{answers}
    \bad{$f$ est injective.}

    \good{$f$ n'est pas injective.}

    \bad{$f$ est surjective.}

    \good{$f$ n'est pas surjective.}

    \good{La restriction de $f$, $f_| : [1,2] \to [2,5]$ est bijective.}
\end{answers}
\end{question}


\begin{question}
Soit $f:\Cc \to \Cc$, $z \mapsto z^2+1$.
\begin{answers}
    \bad{$f$ est injective.}

    \good{$f$ n'est pas injective.}

    \good{$f$ est surjective.}

    \bad{$f$ n'est pas surjective.}

    \good{La restriction de $f$, $f_| : [1,2] \to [2,5]$ est bijective.}
\end{answers}
\end{question}


\begin{question}
Pour $x,y \in \Rr$ et $z=x+iy$, on pose $e^z=e^x \times
e^{iy}=e^{x+iy}$.
\begin{answers}
    \good{$|e^z|=e^x$.}

    \bad{$|e^z|= \sqrt{x^2+y^2}$.}

    \good{Arg $e^z = y$.}

    \bad{Arg $e^z=x+y$.}

    \bad{La fonction $f:\Cc \to \Cc$, $z \mapsto e^z$ est injective.}
\end{answers}
\end{question}


\begin{question}
Par quoi peut on compléter les pointillés pour que les
\textbf{deux} assertions suivantes soient vraies :
$$ z\in \Cc \ \ z=\overline{z} \ldots\ldots z\in \Rr \quad ; \quad z\in \Cc \ \ z^3=-1 \ldots\ldots z=-1$$
\begin{answers}
    \good{$\implies$ et $\Longleftarrow$.}

    \bad{$\iff$ et $\iff$.}

    \bad{$\Longleftarrow$ et $\iff$.}

    \bad{$\implies$ et $\implies$.}

    \good{$\iff$ et $\Longleftarrow$.}
\end{answers}
\end{question}


\begin{question}
Soit la suite $(x_n)_{n\in\Nn^*}$ définie par $x_n =
\frac{(-1)^n}{n}$.
\begin{answers}
    \bad{$\exists N > 0 \quad \forall n \in \Nn^* \qquad (n \ge N \implies x_n \ge 0)$.}

    \bad{$\exists \epsilon >0 \quad \forall n \in \Nn^* \qquad x_n \le \epsilon$.}

    \good{$\forall N \in \Nn^ * \quad \exists n \ge N \qquad  x_n < 0$.}

    \bad{$\exists n \in \Nn^* \qquad x_n =0$.}

    \good{$\forall \epsilon > 0\ \ \exists N \in \Nn^* \ \ \forall n \in \Nn^* \ \ (n\ge N \implies |x_n|\le \epsilon)$.}
\end{answers}
\end{question}


\begin{question}
 Soit $E$ un ensemble, $A,B \subset E$, soit $A\Delta B = (A\cup B)\setminus(A\cap B)$.
Les assertions suivantes sont-elles vraies quels que soient $A$ et
$B$ inclus dans $E$ ?
\begin{answers}
    \good{$A\Delta B = (A\setminus B) \cup (B\setminus A)$.}

    \bad{$A\Delta B = (E \setminus A) \cap (E \setminus B)$.}

    \bad{Si $B\subset A$ alors $ A\Delta B = A$.}

    \good{Si $E$ est un ensemble fini, $\operatorname{Card}(A\Delta B) \le \operatorname{Card} A+\operatorname{Card} B$.}

    \bad{Si $E$ est un ensemble fini, $\operatorname{Card}(A\Delta B) < \operatorname{Card} A+ \operatorname{Card} B$.}
\end{answers}
\end{question}


\begin{question}
Soit la suite $(x_n)_{n\in \Nn}$ définie par $x_0=1$ puis pour
$n \ge 1$ $x_n=\frac {x_{n-1}}{n}$.
\begin{answers}
    \good{$\forall n\in \Nn \qquad x_n > 0$.}

    \good{$\forall n \in \Nn \qquad x_{n+1} \le x_n$.}

    \bad{$\exists N \in \Nn \quad \exists c \in \Rr \quad \forall n \in \Nn \qquad (n \ge N \implies x_n = c)$.}

    \bad{$\forall n \in \Nn \qquad x_n \ge \frac{1}{2}\frac{1}{n!}$.}

    \bad{$\forall n \in \Nn \qquad x_n \le \frac{1}{2}\frac{1}{n!}$.}
\end{answers}
\end{question}


\begin{question}
On lance de façon aléatoire deux dés identiques à $6$
faces (numérotées de $1$ à $6$). On ne tient pas compte de
l'ordre, par exemple le tirage $1$ puis $5$ est le même que $5$
puis $1$, mais les tirages $3$ puis $3$, et $3$ puis $4$ sont
distincts.
\begin{answers}
    \bad{Il y a $36$ tirages distincts possibles.}

    \good{Il y a $30$ tirages distincts possibles.}

    \bad{Il y a $21$ tirages distincts possibles.}

    \good{La somme des deux chiffres a strictement plus de chances d'être $7$ que $2$.}

    \bad{La somme des deux chiffres a strictement plus de chances d'être $\ge 11$ que $\le 3$.}
\end{answers}
\end{question}


\begin{question}
Soit $E$ un ensemble fini de cardinal $n$, soit $A\subset E$ un
ensemble à $p$ éléments, et  $B \subset E$ un ensemble à
$q$ éléments. On note $\mathcal{S} = \{ (a,b) \in A \times B \mid a\neq b \}$ et
 $\mathcal{T} = \{ (I,b) \text{ avec } I\subset A \mid \operatorname{Card} I =r \text{ et } b\in B \}$.
\begin{answers}
    \bad{Si $A \cap B = \emptyset$ alors $\operatorname{Card} \mathcal{S} = p+q$.}

    \good{Si $A \cap B = \emptyset$ alors $\operatorname{Card} \mathcal{S} = pq$.}

    \bad{Si $A \subset B$ alors $\mathcal{S} = \emptyset$.}

    \bad{$\operatorname{Card} \mathcal{T} = C_n^p \times r$.}

    \good{$\operatorname{Card} \mathcal{T} = C_p^r \times q$.}
\end{answers}
\end{question}



\subsection*{Arithmétique}


\begin{question}
Les propositions suivantes sont-elles vraies quels que soient $\ell
\ge 2$ et $p_1,\ldots,p_\ell$  des nombres premiers $> 2$ ?
\begin{answers}
    \bad{$p_1p_2\ldots p_\ell$ est un nombre premier.}

    \bad{Le carré de $p_1$ est un nombre premier.}

    \bad{$p_1p_2\ldots p_\ell + 1$ est un nombre premier.}

    \good{$\prod_{i=1}^\ell p_i$ est un nombre impair.}

    \bad{$\sum_{i=1}^\ell p_i$ est un nombre impair.}
\end{answers}
\end{question}


\begin{question}
\begin{answers}
    \good{Soit $n\in \Nn$ un entier, alors $(n+1)(n+2)(n+3)(n+4)$
est divisible par $24$.}

    \bad{Soit $n \ge 6$ un entier pair alors $\frac{n}{2}$ est impair.}

    \good{La somme et le produit de deux nombres pairs est un nombre pair.}

    \bad{$a|b$ et $a'|b' \quad \implies  \quad aa'|bb'$.}

    \bad{$a|b$ et $a'|b' \quad \implies  \quad a+a'| b+b'$.}

\end{answers}
\end{question}


\begin{question}
\begin{answers}
    \good{Le pgcd de $924$, $441$  et $504$   est $21$.}

    \bad{$627$ et $308$ sont premiers entre eux.}

    \bad{Si $p \ge 3$ est premier, alors $p!$ est premier.}

    \good{Soit $n \ge 2$ alors $n$ et $n+1$ sont premiers entre eux.}

    \good{Soit $n \ge 2$ un entier, le pgcd de $\{ in^i \text{ pour } i=1,\ldots ,100 \}$ est $n$.}
\end{answers}
\end{question}


\begin{question}
Soient $a,b,c \ge 1$ des entiers.
\begin{answers}
    \good{$ab = \operatorname{pgcd}(a,b)\times \operatorname{ppcm}(a,b)$.}

    \bad{$abc = \operatorname{pgcd}(a,b,c)\times \operatorname{ppcm}(a,b,c)$.}

    \good{$\operatorname{ppcm}(a,b,c)$ est divisible par $c$.}

    \bad{$\operatorname{ppcm}(1932,345)= 19320$.}

    \bad{$\operatorname{ppcm}(5,10,15)= 15$.}
\end{answers}
\end{question}


\begin{question}
\begin{answers}
    \bad{Soit $a,b,c \ge 1$ des entiers. Si $a|bc$ et $a$ ne divise pas $b$ alors $a|c$.}

    \good{Sachant que $7$ divise $86419746 \times 111$ alors $7$ divise $86419746$.}

    \good{Si $a = bq + r$ est la division euclidienne de $a$ par $b$ alors $\operatorname{pgcd}(a,b) = \operatorname{pgcd} (b,r)$.}

    \good{Il existe $u,v \in \Zz$ tels que $195u+2380v=5$.}

    \bad{Sachant qu'il existe $u,v$ tels que $2431u+65520v = 39$
alors $\operatorname{pgcd}(2431,65520)=39$.}

\end{answers}
\end{question}


\begin{question}
\begin{answers}
    \good{$\exists P \in \Zz[X] \quad \forall x \in \Rr \qquad P(x) > 0$.}

    \bad{$\forall P \in \Zz[X] \quad \exists x \in \Rr \qquad |P(x)| < 1$.}

    \good{$\forall P \in \Qq[X] \qquad  x\in \Qq \implies P(x) \in \Qq$.}

    \good{$\forall P \in \Cc[X] \text{ de degré}\ge 1 \quad
\exists z\in \Cc\  \qquad  P(z) = 0$.}

    \bad{Tout polynôme de degré $2$ ne s'annulant pas, prend uniquement des valeurs positives.}
\end{answers}
\end{question}


\begin{question}
Soit $P,Q \in\Cc[X]$ des polynômes non nuls  $P = \sum_{i=0}^na_iX^i$, soit $I_P = \{i \in \Nn  \mid a_i \neq 0\}$, soit $\text{val}(P) = \min I_P$.
\begin{answers}
    \good{$\text{val}(-X^7+X^3+7X^2)=2$.}

    \good{$\text{val}(P+Q) \ge \text{val}(P)$.}

    \good{$\text{val}(P\times Q) \ge \text{val}(P)+\text{val}(Q)$.}

    \bad{$\text{val}(k.P) =  k \cdot \text{val}(P)$ où $k \in \Nn^*$.}

    \good{Si $Q|P$ alors $\text{val}(P/Q) = \text{val}(P) - \text{val}(Q)$.}
\end{answers}
\end{question}



\begin{question}
\begin{answers}
    \good{$X^4+X^3-X^2-X$ est divisible par $X(X-1)$.}

    \bad{Le reste la division euclidienne de $X^3+X^2+3$ par $X-1$ est $X+4$.}

    \good{Le quotient de $X^5+2X^3+X^2+2X+1$ par $X^2+1$ est $X^3+X+1$.}

    \good{$X-1$ divise $X^n-1$ pour $n \ge 1$.}

    \bad{$X+1$ divise $X^n+1$ pour $n \ge 1$.}
\end{answers}
\end{question}


\begin{question}
\begin{answers}
    \good{Soit $P\in \Cc[X]$. $X-a$ divise $P$ ssi $P(a)=0$.}

    \good{Soit $P\in \Rr[X]$ de degré impair. Il existe $x \in \Rr$ tel que $P(x) = 0$.}

    \good{Soit $P\in \Rr[X]$, les racines de $P^2$ sont d'ordre au moins $2$.}

    \bad{Soit $P\in \Rr[X]$, $x$ est racine simple ssi $P(x) = 0$.}

    \bad{Un polynôme  $P\in \Cc[X]$ de degré $n$ a $n$ racines réelles.}

\end{answers}
\end{question}


\begin{question}
\begin{answers}
    \bad{$X^4+1$ est irréductible dans $\Rr[X]$.}

    \good{$X^2+7$ est irréductible dans $\Qq[X]$.}

    \bad{$X^2+7$ est irréductible dans $\Cc[X]$.}

    \bad{Dans $\Zz[X]$, $\operatorname{pgcd}(X(X-1)^2(X^2+1) ,X^2(X-1)(X^2-1) ) = X(X-1)$.}

    \good{Dans $\Zz[X]$, $\operatorname{pgcd}(X^4+X^3+X^2+X,X^3-X^2-X+1) = X+1$.}
\end{answers}
\end{question}


\subsection*{Réels}

\begin{question}[Réel et rationnels]
\begin{answers}
    \good{$(x\in \Qq \text{ et }  y\in\Qq) \implies x+y \in \Qq$}
    \bad{$(x\in \Rr \setminus \Qq \text{ et } y\in\Rr \setminus \Qq) \implies x+y \in \Rr \setminus \Qq$}
    \good{$\forall x \in \Rr \setminus \Qq \quad \forall y \in \Rr \setminus \Qq \qquad x < y \implies (\exists z \in \Qq \quad x<z<y)$}
    \good{$(\forall x \in \Rr \setminus \Qq) \quad (\forall y \in \Rr \setminus \Qq) \qquad x < y \implies (\exists z \in \Rr \setminus
\Qq \quad  x<z<y)$}
    \bad{Pour $n\ge 3$, $n \text{ impair } \implies \sqrt{n}\in \Rr \setminus \Qq$}
\end{answers}
\end{question}



\begin{question}
Soient $A,B,C$ des parties de $\Rr$
\begin{answers}
    \bad{Si $\sup A$ existe alors $\max A$ existe.}
    \good{Si $\max A$ existe alors $\sup A$ existe.}
    \good{Pour $A, B$ majorées et $C \subset A\cap B$ alors $\sup C \le \sup A$ et $\sup C \le \sup B$.}
    \bad{Si $A = \left\lbrace \frac{(-1)^n}n + 1 \ | \ n\in \Nn^* \right\rbrace$ alors $\inf A = 0$ et $\sup A =1$.}
    \good{Si $B = \left\lbrace \frac{E(x)}x  \ | x>0 \right\rbrace$ alors $\inf B = 0$ et $\sup B =1$.}
\end{answers}
\end{question}


\begin{question}[Limites de suites]
\begin{answers}
    \good{Si $u_n= n\sin(\frac 1 n)$ alors $(u_n)$ tend vers $1$.}
    \bad{Si $u_n=\ln (\ln(n))$ alors $(u_n)$ a une limite finie.}
    \bad{$u_n = \frac{(\ln n)^2}{\sqrt n}$ alors $(u_n)$ tend vers $+\infty$.}
    \bad{$u_n = 1+\frac12 +\frac14 +\frac18 + \cdots +\frac{1}{2^n}$ alors $(u_n)$ diverge.}
    \good{$u_n = \sin(n)$, il existe une sous-suite de $(u_n)$ convergente.}
\end{answers}
\end{question}


\begin{question}[Suites définies par récurrence]
Soit $f(x) = 2x(1-x)$ et la suite définie par $u_0 \in [0,1]$ et $u_{n+1} = f(u_n)$.
\begin{answers}
    \good{$\forall n\in \Nn \qquad u_n \in [0,1]$.}
    \bad{Quelque soit $u_0$ dans $[0,1]$, $(u_n)$ est monotone.}
    \bad{Si $(u_n)$ converge vers $\ell$ alors $\ell=0$ ou $\ell=1$.}
    \good{Si $(u_n)$ converge vers $\ell$ alors $\ell=0$ ou $\ell=\frac12$.}
    \good{$u_0 \in]0,1[$ alors $(u_n)$ ne converge pas vers $0$.}
\end{answers}
\end{question}


\begin{question}[Fonctions continues]
\begin{answers}
    \bad{La somme, le produit et le quotient de deux fonctions continues est continue.}
    \good{La fonction $\sqrt{\sqrt{x}}\ln x$ est prolongeable par continuité en $0$.}
    \bad{Il existe $a,b \ge 0$ tels que fonction définie par $f(x) = -e^x$ si $x<0$ et $f(x) = ax^2+b$ si $x\ge 0$ soit continue.}
    \bad{Toute fonction impaire de $\Rr$ dans $\Rr$ est continue en $0$.}
    \bad{La fonction $\frac{\sqrt{|x|}}{x}$ est prolongeable par continuité en $0$.}
\end{answers}
\end{question}


\begin{question}[Théorème des valeurs intermédiaires, fonctions bornées]
\begin{answers}
    \good{La méthode de dichotomie est basée sur le théorème des valeurs intermédiaires.}
    \bad{Tout polynôme de degré $\ge 3$ a au moins une racine réelle.}
    \bad{La fonction $f(x) = \frac{1}{x^3(x^2+1)}$ admet au moins une racine réelle dans $]-1,+1[$.}
    \good{Pour $f : \Rr^+ \longrightarrow \Rr$ continue admettant une limite finie en $+\infty$, $f$ est bornée.}
    \bad{Pour $f : \Rr^+ \longrightarrow \Rr$ continue admettant une limite finie qui vaut $f(0)$ en $+\infty$ alors $f$ est bornée et atteint ses bornes.}
\end{answers}
\end{question}

\begin{question}[Dérivation]
\begin{answers}
    \bad{La fonction $f(x) = 1/x$ est décroissante sur $\Rr^*$.}
    \good{La fonction $f(x) = x\sin\frac1x$ est continue et dérivable en $0$.}
    \good{La fonction définie par $x\mapsto 0$ si $x\in\Qq$ et $x\mapsto x^2$ si $x\notin\Qq$ est dérivable en $0$.}
    \good{Si $f(x) = P(x)e^x$ avec $P$ un polynôme alors pour tout $n\in \Nn$ il existe un polynôme $Q_n$ tel que $f^{(n)}(x) = Q_n(x)e^x$.}
    \bad{Si $f(x) = \sqrt x \ln x$ si $x\in \Rr^*$ et $f(0)=0$ alors $f$ est dérivable en $0$.}
\end{answers}
\end{question}


\begin{question}[Théorème de Rolle et des accroissements finis]
\begin{answers}
    \bad{Si $f$ est dérivable sur $[a,b]$ avec $f(a) = f(b)$ il existe un unique $c \in ]a,b[$ tel que $f'( c ) = 0$.}
    \good{Si $f$ est une fonction continue sur $[a,b]$ et dérivable sur $]a,b[$ et $f'(x)$ tend vers $\ell$ quand  $x$ tend vers $a$
alors $f$ est dérivable en $a$ et $f'(a) = \ell$.}
    \bad{Soit $f(x) = \ln x$ si $x>0$ et $f(0)= 0$. Pour $x>0$ il existe $c\in]0,x[$ tel que $\ln x = \frac xc$.}
    \good{Si $f$ est dérivable sur $\Rr$ et $\lim f(x) = +1$ quand $x\rightarrow +\infty$ et $\lim f(x) = +1$ quand $x\rightarrow -\infty$ alors il existe $c \in \Rr$ tel que $f'(c)=0$.}
    \good{$\forall x > 0\ e^x \le xe^x+1$.}
\end{answers}
\end{question}


\begin{question}[Fonctions usuelles]
\begin{answers}
    \good{$\forall n \in \Nn \ \lim_{x\rightarrow +\infty} \frac{e^x}{x^n} = +\infty$.}
    \good{$\forall x \in \Rr \  \operatorname{ch} x \ge \operatorname{sh} x$.}
    \good{$\frac{\operatorname{ch} x}{\operatorname{sh} x}$ tend vers $1$ quand $x$ tend vers $+\infty$.}
    \good{$\operatorname{ch} 2x = 1 + 2\operatorname{sh}^2 x$.}
    \bad{$\operatorname{th}(a+b) = \frac{\operatorname{th} a+ \operatorname{th} b}{1- \operatorname{th} a \operatorname{th} b}$.}
\end{answers}
\end{question}


\begin{question}[Fonctions réciproques]
\begin{answers}
    \bad{Un fonction continue $\Rr \longrightarrow \Rr$ strictement décroissante est bijective.}
    \good{Si $f$ est une fonction continue bijective croissante alors $f^{-1}$ est croissante.}
    \bad{Si $f$ est une fonction continue bijective ne s'annulant jamais alors $(\frac 1 f)^{-1} = f$.}
    \bad{$\operatorname{arcsin} (\sin x) = x$ pour tout $x \in [0, 2\pi[$.}
    \bad{Si $f(x) = \operatorname{arctan} (x^2)$ alors $f'(x) = \frac{1}{1+x^4}$.}
\end{answers}
\end{question}

