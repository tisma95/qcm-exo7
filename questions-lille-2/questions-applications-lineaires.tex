
\qcmtitle{Applications linéaires}
\qcmauthor{Abdellah Hanani, Mohamed Mzari}


%%%%%%%%%%%%%%%%%%%%%%%%%%%%%%%%%%%%%%%%%%%%%%%%%%%%%%%%%
\section{Applications linéaires }
\subsection{Applications linéaires | Niveau 1}

\begin{question}
\qtags{motcle=Application linéaire}
On considère les deux applications suivantes : 
$$\begin{array}{rccc}f:&\Rr&\to&\Rr\\
& x&\to & \sin x \end{array} \quad \mbox{et} \quad \begin{array}{rccc}g:&\Rr^2&\to&\Rr^2\\
& (x,y)&\to &(y,x). \end{array}$$ 
Quelles sont les assertions vraies ?
\begin{answers}  
\good{$f(0)=0$}.
\bad{$f$ est une application linéaire}.
\bad{$g(x,y) =  g(y,x)$, pour tout $(x,y) \in \Rr^2$}.
\good{$g$ est une application linéaire}.
\end{answers}
\begin{explanations} $f$ n'est pas linéaire car $f(\pi)=f(\frac{\pi}{2}+ \frac{\pi}{2})=0$ 
et $f(\frac{\pi}{2})+ f(\frac{\pi}{2}) =2$. 
On vérifie que $g$ est linéaire.  
\end{explanations}
\end{question}


\begin{question}
\qtags{motcle=Application linéaire}
On considère les deux applications suivantes : 
$$\begin{array}{rccc}f:&\Rr^2&\to&\Rr^2\\
& (x,y)&\to & (x,y^2) \end{array} \quad \mbox{et} \quad \begin{array}{rccc}g:&\Rr^2&\to&\Rr^2\\
& (x,y)&\to &(x,-x). \end{array}$$ 
Quelles sont les assertions vraies ?
\begin{answers}  
\good{$f(0,2)=(0,4)$}.
\bad{$f$ est une application linéaire}.
\good{$g(0,0)=(0,0)$}.
\good{$g$ est une application linéaire}.
\end{answers}
\begin{explanations} L'application $f$ n'est pas linéaire. Contre-exemple : $f(2(0,1))=f(0,2)=(0,4)$ et $2f(0,1)=(0,2)$. On vérifie que l'application $g$ est linéaire.
\end{explanations}
\end{question}


\begin{question}
\qtags{motcle=Application linéaire}
On considère les deux applications suivantes : 
$$\begin{array}{rccc}f:&\Rr^3&\to&\Rr^2\\
& (x,y,z)&\to &(x+y,x-z) \end{array} \quad \mbox{et} \quad \begin{array}{rccc}g:&\Rr^3&\to&\Rr^2\\
& (x,y,z)&\to &(xy,xz). \end{array}$$ 
Quelles sont les assertions vraies ?
\begin{answers}  
\good{$f(0,0,0)=(0,0)$}.
\good{$f$ est une application linéaire}.
\bad{$g(1,1,0)=g(1,0,0)+ g(0,1,0)$}.
\bad{$g$ est une application linéaire}.
\end{answers}
\begin{explanations} $f$ est linéaire. $g$ ne l'est pas,
puisque $g((1,0,0)+ (0,1,0)) = g(1,1,0) = (1,0)$ et $ g(1,0,0)+ g(0,1,0) = (0,0)$.
\end{explanations}
\end{question}

\subsection{Applications linéaires | Niveau 2}



\begin{question}
\qtags{motcle=Application linéaire/Polynômes}
On note $\Rr_n[X]$ l'espace des polynômes à coefficients réels de degré $\le n$,  $n\in \Nn$. On considère les deux applications suivantes :
$$\begin{array}{rccc}f:&\Rr_3[X]&\to&\Rr\\
& P&\to &P(0)+P'(0)\,  \end{array}   \quad \mbox{et} \quad \begin{array}{rccc}g:&\Rr_3[X]&\to&\Rr_2[X]\\
& P&\to &1+P'+XP'',\end{array}$$ 
où $P'$ (resp. $P''$) est la dérivée première (resp. seconde) de $P$. Quelles sont les assertions vraies ?
\begin{answers} 
\bad{$f(0)=1$}.
\good{$f$ est une application linéaire}.
\good{$g(0)=1$}.
\bad{$g$ est une application linéaire}.
\end{answers}
\begin{explanations} On vérifie que $f$ est linéaire. Par contre, $g$ ne l'est pas, puisque $g(0)=1\neq 0$.
\end{explanations}
\end{question}


\begin{question}
\qtags{motcle=Application linéaire/Nombres complexes}
On considère les applications suivantes : 
$$\begin{array}{rccc}f:&\Cc&\to&\Cc\\
& z&\to& \Re (z)\end{array}   \quad \mbox{et} \quad  \begin{array}{rccc}g:&\Cc&\to&\Cc\\
& z&\to& \Im (z), \end{array}$$
où $\Re (z)$ (resp. $\Im (z)$) est la partie réelle (resp. imaginaire) de $z$. Quelles sont les assertions vraies ?
\begin{answers}  
\bad{$f$ est $\Cc$-linéaire}.
\good{$f$ est $\Rr$-linéaire}.
\good{$g$ est  $\Rr$-linéaire}.
\bad{$g$ est $\Cc$-linéaire}.
\end{answers}
\begin{explanations} On vérifie que $f$ et $g$ sont $\Rr$-linéaires. Par contre, elles ne sont pas $\Cc$-linéaires.
\end{explanations}
\end{question}


\begin{question}
\qtags{motcle=Application linéaire/Nombres complexes}
On considère les applications suivantes : 
$$\begin{array}{rccc}f:&\Cc&\to&\Cc\\
& z&\to& |z|\end{array}  \quad \mbox{et} \quad \begin{array}{rccc}g:&\Cc&\to&\Cc\\
& z&\to& \overline{z},
\end{array} $$
où $|z|$ (resp. $\overline{z}$) est le module (resp. le conjugué) de $z$. Quelles sont les assertions vraies ?
\begin{answers}  
\bad{$f$ est $\Cc$-linéaire}.
\bad{$f$ est $\Rr$-linéaire}.
\good{$g$ est $\Rr$-linéaire}.
\bad{$g$ est $\Cc$-linéaire}.
\end{answers}
\begin{explanations} On vérifie que $f$ n'est pas $\Rr$-linéaire (donc n'est pas $\Cc$-linéaire) et que $g$ est $\Rr$-linéaire, mais non $\Cc$-linéaire.
\end{explanations}
\end{question}



\subsection{Applications linéaires | Niveau 3}

\begin{question}
\qtags{motcle=Application linéaire}
On considère les deux applications suivantes : 
$$\begin{array}{rccc}f:&\Rr^2&\to&\Rr\\
& (x,y)&\to & |x+y| \end{array} \quad \mbox{et} \quad \begin{array}{rccc}g:&\Rr^2&\to&\Rr^2\\
& (x,y)&\to &\big(\max (x,y)\, , \, \min (x,y)\, \big). \end{array}$$ 
Quelles sont les assertions vraies ?
\begin{answers}  
\good{$f(1,-1)=0$}.
\bad{$f$ est une application linéaire}.
\good{$g(0,0)=(0,0)$}.
\bad{$g$ est une application linéaire}.
\end{answers}
\begin{explanations} $f$ n'est pas linéaire. Contre-exemple : $f((1,0)+(-1,0))=f(0,0)=0$ et $f(1,0)+f(-1,0)=|1|+|-1|=2$.
\vskip0mm
$g$ n'est pas linéaire. Contre-exemple :  $g((1,0)+(-1,0))=g(0,0)=(0,0)$ 
et $g(1,0)+g(-1,0)=(1,0)+(0,-1)=(1,-1)$.
\end{explanations}
\end{question}

\begin{question}
\qtags{motcle=Application linéaire}
On considère les applications suivantes : 
$$\begin{array}{rccc}f:&\Rr^3&\to&\Rr^2\\
& (x,y,z)&\to &(x-y,y+2z+a) \,  \end{array}  \quad \mbox{et} \quad  \begin{array}{rccc}g:&\Rr^3&\to&\Rr\\
& (x,y,z)&\to &(ax+b)(x+y).\end{array} $$ 
où $a$ et $b$  sont des réels. Quelles sont les assertions vraies ?
\begin{answers}  
\bad{Pour tout $a\in \Rr$, $f$ est une application linéaire}.
\good{$f$ est une application linéaire si et seulement si $a=0$}.
\bad{$g$ est une application linéaire si et seulement si $a=b=0$}.
\good{$g$ est une application linéaire si et seulement si $a=0$}.
\end{answers}
\begin{explanations} Si $a\neq 0$, alors $f(0,0,0)=(0,a)\neq(0,0)$, donc $f$ n'est pas linéaire. On vérifie aussi que, si $a=0$, alors $f$ est linéaire.
\vskip0mm
Si $a\neq 0$, $g(2(1,0,0))=4a+2b$ et $2g(1,0,0)=2a+2b \neq 4a+2b$, donc $g$ n'est pas linéaire.
\vskip0mm
On vérifie que si $a=0$ et $b$ est quelconque, $g$ est linéaire.
\end{explanations}
\end{question}

\begin{question}
\qtags{motcle=Application linéaire}
On considère les applications suivantes : 
$$\begin{array}{rccc}f:&\Rr^3&\to&\Rr^2\\
& (x,y,z)&\to &(z,x+ax^2) \,  \end{array}  \quad \mbox{et} \quad  \begin{array}{rccc}g:&\Rr^3&\to&\Rr^3\\
& (x,y,z)&\to &(z+a\sin x, y+be^x, c|x|+1).\end{array} $$ 
où $a,b$ et $c$ sont des réels. Quelles sont les assertions vraies ?
\begin{answers}  
\bad{Pour tout $a\in \Rr$, $f$ est une application linéaire}.
\good{$f$ est une application linéaire si et seulement si $a=0$}.
\bad{$g$ est une application linéaire si et seulement si $a=b=c=0$}.
\good{Pour tous $a,b, c \in \Rr$, $g$ n'est pas une application linéaire}.
\end{answers}
\begin{explanations} On vérifie que $f$ est linéaire si et seulement si $a=0$ et que $g$ n'est pas linéaire pour 
tous réels $a,b$ et $c$.
\end{explanations}
\end{question}

\begin{question}
\qtags{motcle=Application linéaire/Polynôme}
On note $\Rr_n[X]$ l'espace des polynômes à coefficients réels de degré $\le n$,  $n\in \Nn$. On considère les deux applications suivantes : 
$$\begin{array}{rccc}f:&\Rr_3[X]&\to&\Rr_2[X]\\
& P&\to &R \end{array} \quad \mbox{et} \quad 
 \begin{array}{rccc}g:&\Rr_3[X]&\to& \Rr_2[X]\\
& P&\to &Q,\end{array}$$
où $R$ (resp. $Q$) est le reste (resp. le quotient) de la division euclidienne de $P$ par $X^3+1$. Quelles sont les assertions vraies ?
\begin{answers} 
\good{$f(0)=0$}.
\good{$f$ est une application linéaire}.
\good{$g(0)=0$}.
\bad{$g$ n'est pas une application linéaire}.
\end{answers}
\begin{explanations} On vérifie que $f$ et $g$ sont linéaires.
\end{explanations}
\end{question}

\subsection{Applications linéaires | Niveau 4}

\begin{question}
\qtags{motcle=Application linéaire}
Quelles sont les assertions vraies ?
\begin{answers}  
\good{Une application $f:\Rr \to \Rr$ est linéaire si et seulement s'il existe un réel $a$ tel que $f(x)=ax$, pour tout $x\in \Rr$}.
\bad{Une application $f:\Rr^2 \to \Rr^2$ est linéaire si et seulement s'il existe des réels $a$ et $b$ tels que $f(x,y)=(ax,by)$, pour tout $(x,y) \in \Rr^2$}.
\good{Une application $f:\Rr^2 \to \Rr^2$ est linéaire si et seulement s'il existe des réels $a,b,c$ et $d$ tels que $f(x,y)=(ax+by,cx+dy)$, pour tout $(x,y)\in \Rr^2$}.
\bad{Une application $f:\Rr^3 \to \Rr^3$ est linéaire si et seulement s'il existe des réels $a,b$ et $c$ tels que $f(x,y,z)=(ax,by,cz)$, pour tout $(x,y,z)\in \Rr^3$}.
\end{answers}
\begin{explanations} Une application $f:\Rr^n \to \Rr^m$ est linéaire si et seulement s'il existe des réels $a_{i,j}, 1\le i \le m, 1\le j \le n$,  tels que :\\ $f(x_1,x_2,\dots,x_n)=(a_{1,1}x_1+a_{1,2}x_2 + \dots + a_{1,n}x_n,\dots, a_{m,1}x_1+a_{m,2}x_2 + \dots + a_{m,n}x_n)$, pour tout $(x_1,x_2,\dots,x_n) \in \Rr^n$.
\end{explanations}
\end{question}

\subsection{Noyau et image  | Niveau 1}

\begin{question}
\qtags{motcle=Noyau/Image}
Soit $E$ et $F$ deux espaces vectoriels et $f:E\to F$ une application linéaire. Quelles sont les assertions vraies ?
\begin{answers}  
\bad{$\ker f$ peut-être vide}.
\good{$\ker f$ est un sous-espace vectoriel de $E$}.
\bad{$0_E \in \Im f$}.
\good{$\Im f$ est un sous-espace vectoriel de $F$}.
\end{answers}
\begin{explanations} $\ker f$ est un sous-espace vectoriel de $E$, il contient au moins $0_E$.
\vskip0mm
$\Im f $ est un sous-espace vectoriel de $F$, il contient au moins $0_F$, puisque $f(0_E)=0_F$.
\end{explanations}
\end{question}

\begin{question}
\qtags{motcle=Noyau/Image/Injection/Surjection/Bijection}
Soit $E$ et $F$ deux espaces vectoriels et $f:E\to F$ une application linéaire.
Quelles sont les assertions vraies ?
\begin{answers}  
\bad{$f$ est injective si et seulement si $\ker f$ est vide}.
\bad{$f$ est injective si et seulement si $\ker f$ est une droite vectorielle}.
\good{$f$ est surjective si et seulement si $\Im f=F$}.
\bad{$f$ est bijective si et seulement si $\Im f=F$}.
\end{answers}

\begin{explanations} $f$ est injective si et seulement si $\ker f=\{0_E\}$.
\vskip0mm
$f$ est surjective si et seulement si $\Im f=F$.
\vskip0mm
$f$ est bijective si et seulement si $\ker f=\{0_E\}$ et $\Im f=F$.
\end{explanations}
\end{question}


\subsection{Noyau et image  | Niveau 2}

\begin{question}
\qtags{motcle=Noyau/Image/Injection/Surjection/Bijection/Théorème du rang}
Soit $f$ une application linéaire de $\Rr^3$ dans $\Rr^5$. Quelles sont les assertions vraies ?
\begin{answers}  
\bad{Si $\ker f = \{(0,0,0)\}$, alors $f$ est surjective}.
\good{Si $\ker f $ est une droite vectorielle, alors $\Im f $ est un plan vectoriel}.
\good{$f$ est injective si seulement si $\dim \Im f =3$}.
\bad{$f$ est  bijective si et seulement si $\ker f=\{(0,0,0)\}$}.
\end{answers}
\begin{explanations} $f$ est injective si et seulement si $\ker f = \{(0,0,0)\}$. $f$ ne peut pas être surjective, puisque d'après le théorème du rang, la dimension de $\Im f $ est au plus $3$.
\end{explanations}
\end{question}


\begin{question}
\qtags{motcle=Noyau/Image/Injection/Surjection/Théorème du rang}
On considère l' application linéaire : 
$$\begin{array}{rccc}f:&\Rr^3&\to&\Rr^3\\
& (x,y,z)&\to &(x-z,y+z,x+y). \end{array}$$
Quelles sont les assertions vraies ?
\begin{answers}  
\good{$\{(1,-1,1)\}$ est une base de $\ker f$}.
\bad{$f$ est injective}.
\good{$\{(1,0,1), (0,1,1)\}$ est une base de $\Im f$}.
\bad{$f$ est surjective}.
\end{answers}
\begin{explanations} $\ker f = \{(x,y,z) \in \Rr^3 \; ; \; (x-z,y+z,x+y)=(0,0,0)\} = \{(x,-x,x)  \; ; \; x\in \Rr\}$. Donc $ \{(1,-1,1)\}$ est une base de $\ker f$. Comme $\ker f \neq \{(0,0,0)\}$, $f$ n'est pas injective.
\vskip0mm
D'après le théorème du rang, $\dim \Im f = 2 $ et comme 
$f(e_1)=(1,0,1)$ et $ f(e_2)=(0,1,1)$ ne sont pas colinéaires, ils forment une base de  $\Im f$. Comme $\dim \Im f = 2 $, $\Im f \neq \Rr^3 $, donc $f$ n'est pas surjective.
\end{explanations}
\end{question}

\begin{question}
\qtags{motcle=Noyau/Image/Injection/Surjection/Bijection/Théorème du rang}
On considère l' application linéaire : 
$$\begin{array}{rccc}f:&\Rr^3&\to&\Rr^3\\
& (x,y,z)&\to &(x-y,y-z,x+z). \,  \end{array}$$
Quelles sont les assertions vraies ?
\begin{answers}  
\bad{$\dim \ker f = 1$}.
\good{$f$ est injective}.
\good{$\dim \Im f = 3$}.
\bad{$f$ n'est pas bijective}.
\end{answers}
\begin{explanations} $\ker f = \{(x,y,z) \in \Rr^3 \; ; \; (x-y,y-z,x+z)=(0,0,0)\} = \{(0,0,0)\}$. Donc $\dim \ker f = 0$ et  $f$ est injective.
\vskip0mm
D'après le théorème du rang, $\dim \Im f = 3=\dim \Rr^3$ et comme $\Im f$ est un sous-espace de $\Rr^3$, $\Im f=\Rr^3$, donc $f$ est surjective. Par conséquent, $f$ est bijective.
\end{explanations}
\end{question}


\begin{question}
\qtags{motcle=Noyau/Image/Injection/Surjection/Bijection/Théorème du rang}
On considère l' application linéaire : 
$$\begin{array}{rccc}f:&\Rr^3&\to&\Rr^2\\
& (x,y,z)&\to &(x+y+z, x+y-z). \,  \end{array}$$
Quelles sont les assertions vraies ?
\begin{answers}  
\good{$\dim \ker f = 1$}.
\bad{$f$ est injective}.
\bad{$\mbox{rg} (f) =1$}.
\good{$f$ n'est pas bijective}.   
\end{answers}
\begin{explanations}  $\ker f = \{(x,y,z) \in \Rr^3 \; ; \; (x+y+z,x+y-z)=(0,0,0)\} =\{(x,-x,0)  \; ; \; x\in \Rr\}$. Donc $ \{(1,-1,0)\}$ est une base de $\ker f$. Comme $\ker f \neq \{(0,0,0)\}$, $f$ n'est pas injective, donc $f$ n'est pas bijective. Du théorème du rang, on déduit que $\mbox{rg} (f)= \dim \Im f = 2$.
\end{explanations}
\end{question}

\begin{question}
\qtags{motcle=Noyau/Image/Théorème du rang}
On considère $\Rr^3$ muni de la base canonique ${\cal{ B}}= \{e_1,e_2,e_3\}$ et $f$ l'endomorphisme de $\Rr^3$ défini par 
$f(e_1)= e_3,\, f(e_2)= e_1+e_2,\,  f(e_3)= e_1+e_2+e_3$. 
Quelles sont les assertions vraies ?
\begin{answers}  
\bad{$\{e_1+e_2-e_3\}$ est une base de $\Im f $}.
\good{$\dim \Im f =2$}.
\good{$\{e_1+e_2-e_3\}$ est une base de $\ker f$}.
\bad{$\dim \ker f=2 $}.
\end{answers}
\begin{explanations}  On remarque que $f(e_3)= f(e_1)+f(e_2) $ et que $f(e_1)$ et $f(e_2)$ ne sont pas colinéaires, donc $\{e_1+e_2,e_3\}$ est une base de $\Im f$ et donc $\dim \Im f =2$.
\vskip0mm
D'après le théorème du rang, $\dim \ker f = 1 $ et comme $f(e_1+e_2-e_3) = 0$ et  $e_1+e_2-e_3 \neq 0$, $\{e_1+e_2-e_3\}$ est une base de $\ker f$.
\end{explanations}
\end{question}

\subsection{Noyau et image  | Niveau 3}

\begin{question}
\qtags{motcle=Noyau/Image/Injection/Surjection/Bijection/Théorème du rang/Polynôme}
On considère l' application linéaire : 
$$\begin{array}{rccc}f:&\Rr_2[X]&\to&\Rr_2[X]\\
& P&\to &P',  \end{array}$$
où $\Rr_2[X]$ est l'ensemble des polynômes à coefficients réels de degré $\le 2$ et $P'$ est la dérivée de $P$. Quelles sont les assertions vraies ?
\begin{answers}  
\good{$\{1\}$ est une base de $\ker f $}.
\good{$\{1, X\}$ est une base de $\Im f $}.
\bad{$\{0, 1, X\}$ est une base de $\Im f $}.
\bad{$f$ est  surjective}.
\end{answers}
\begin{explanations} $\ker f = \{P \in \Rr_2[X] \; ; \; P'=0\} = \Rr$. Donc $ \{1\}$ est une base de $\ker f$. Comme $\ker f \neq \{0\}$, $f$ n'est pas injective.
\vskip0mm
D'après le théorème du rang, $\dim \Im f = 2 $ et comme $f(X)=1$ et $ f(X^2) = 2X$ ne sont pas colinéaires, $\{1,X\}$ est une base de  $\Im f$. Donc $f$ n'est pas surjective, puisque $\dim \Im f = 2 $ et 
$\dim \Rr_2[X] =3 $.
\end{explanations}
\end{question}

\begin{question}
\qtags{motcle=Noyau/Image/Théorème du rang/Polynôme}
On considère l'application linéaire : 
$$\begin{array}{rccc}f:&\Rr_2[X]&\to&\Rr_2[X]\\
& P&\to &XP'-X^2P'',  \end{array}$$
où $\Rr_2[X]$ est l'ensemble des polynômes à coefficients réels de degré $\le 2$ et $P'$ (resp. $P''$) est la dérivée première (resp. seconde) de $P$. Quelles sont les assertions vraies ?
\begin{answers}  
\bad{$\{1+X^2\}$ est une base de $\ker f $}.
\good{$\{1, X^2\}$ est une base de $\ker f $}.
\bad{$\{1+X\}$ est une base de $\Im f $}.
\good{ $\mbox{rg} (f) =1$}.
\end{answers}
\begin{explanations}  $\ker f=\{P \in \Rr_2[X] \; ; \; XP'-X^2P''=0\}=\{aX^2+b \, ; \; a,b\in \Rr\}$. Donc $\{1,X^2\}$ est une base de $\ker f$. Du théorème du rang, on déduit que $\mbox{rg}(f)=\dim \Im f=1 $ et comme $f(X)=X\neq 0$, $\{X\}$ est une base de $\Im f$. 
\end{explanations}
\end{question}

\begin{question}
\qtags{motcle=Noyau/Image/Théorème du rang/Polynôme}
On note $\Rr_n[X]$ l'espace des polynômes à coefficients réels de degré $\le n$, $n\in \Nn$. On considère l'application linéaire : 
$$\begin{array}{rccc}f:&\Rr_3[X]&\to&\Rr_2[X]\\
& P&\to & R,  \end{array}$$
où $R$ est le reste de la division euclidienne de $P$ par $(X+1)^3$. Quelles sont les assertions vraies ?
\begin{answers}  
\bad{$\{X^3\}$ est une base de $\ker f $}.
\good{$\dim \ker f =1$}.
\bad{$\{1+X+X^2\}$ est une base de $\Im f $}.
\bad{$\mbox{rg} (f) =3$}.
\end{answers}
\begin{explanations}  $\ker f=\{P \in \Rr_3[X] \; ; \; R=0\}=\{(X+1)^3Q \; ; \; Q\in\Rr\}$. Donc $\{(1+X)^3\}$ est une base de $\ker f$. D'après le théorème du rang, $\mbox{rg}(f)=\dim \Im f=3$ et l'on vérifie que $\{1,X,X^2\}$ est une base de $\Im f$. 
\end{explanations}
\end{question}

\begin{question}
\qtags{motcle=Noyau/Image/Injection/Surjection/Théorème du rang/Polynôôme}
On considère $\Rr_3[X]$, l'espace des polynômes à coefficients réels de degré $\le 3$, muni de sa base canonique ${\cal{ B}}= \{1,X,X^2,X^3\} $ et $f$ l'endomorphisme de $\Rr_3[X]$
défini par :
$$f(1)=X,\;  f(X)=1+X,\; f(X^2)= (X-1)^2,\; f(X^3)=(X-1)^3.$$
Quelles sont les assertions vraies ?
\begin{answers}  
\bad{$\dim \ker f =1$}.
\good{$f$ est injective}.
\bad{$f$ n'est pas injective}.
\good{$\mbox{rg} (f)=4$}.
\end{answers}
\begin{explanations}  Soit $P= aX^3+bX^2+cX+d,\, a,b,c,d \in \Rr $. On a :
$$P\in \ker f\Leftrightarrow a(X-1)^3+b(X-1)^2+(c+d)(X-1)+2c+d=0.$$ Comme $\{1,X-1,(X-1)^2,(X-1)^3\}$ est une famille libre, on déduit que $P=0$. Donc $\dim \ker f=0$ et $f$ est injective.
\vskip0mm
D'après le théorème du rang, $\mbox{rg}(f)=\dim \Im f=4$ et comme $\Im f$ est un sous-espace vectoriel de $\Rr_3[X]$, $\Im f=\Rr_3[X]$ et donc $f$ est surjective.
\end{explanations}
\end{question}

\begin{question}
\qtags{motcle=Noyau/Image/Injection/Surjection/Théorème du rang}
Soit $E$ et $F$ deux $\Rr$-espaces vectoriels de dimensions finies et $f$ une application linéaire de $E$ dans $F$. On pose $\dim E=n $ et $\dim F=m$. Quelles sont les assertions vraies ?
\begin{answers}  
\good{Si $f$ est injective, alors $n \le m$}.
\bad{Si $n \le m$, alors $f$ est injective}.
\good{Si $f$ est surjective, alors $n \ge m$}.
\bad{Si $n \ge m$, alors $f$ est surjective}. 
\end{answers}
\begin{explanations} On a : $\dim Im f \le \dim F=m$.
\vskip0mm
Du théorème du rang, on déduit que si $n>m$,  $\dim \ker f >0$ et donc $f$ n'est pas injective et que si $n<m$,  $\dim \Im f <m$ et donc $f$ n'est pas surjective.
\vskip0mm
Si $n\le m$, alors $f$ n'est pas nécessairement injective et si $n \ge m$, alors $f$ n'est pas nécessairement surjective. Exemple : L'application nulle de  $\Rr$ dans $\Rr$.
\end{explanations}
\end{question}

\begin{question}
\qtags{motcle=Noyau/Image/Injection/Surjection/Théorème du rang}
Soit $E$ et $F$ deux $\Rr$-espaces vectoriels de  dimensions finies tels que $\dim E= \dim F=n$ et $f$ une application linéaire de $E$ dans $F$. Quelles sont les assertions vraies ?
\begin{answers}  
\bad{Si $\dim \ker f =0$, alors $\dim \Im f <n$}.
\good{si $f$ est injective, alors $f$ est surjective}.
\good{Si $\dim \Im f <n$, alors $\dim \ker f >0$}.
\good{si $f$ est surjective, alors $f$ est injective}.  
\end{answers}
\begin{explanations} Du théorème du rang, on déduit que $f$ est bijective si, et seulement si, $f$ est injective ou surjective.
\end{explanations}
\end{question}

\begin{question}
\qtags{motcle=Noyau/Image/Injection/Surjection/Théorème du rang}
Soit $E$ et $F$ deux $\Rr$-espaces vectoriels de dimensions finies   
et $f$ une application linéaire de $E$ dans $F$. Quelles sont les assertions vraies ?
\begin{answers}  
\bad{Si $f$ est injective, alors $f$ est surjective}.
\bad{Si $f$ est surjective, alors $f$ est injective}.
\bad{Si $\dim E = \dim F$, alors $f$ est bijective}.
\good{Si $f$ est bijective, alors $\dim E = \dim F$}.
\end{answers}
\begin{explanations} Si $\dim E \neq \dim F$, alors $f$ peut-être injective (resp. surjective) sans qu'elle soit surjective (resp. injective).
\vskip0mm
Si $\dim E = \dim F$, $f$ n'est pas nécessairement bijective.
\vskip0mm
Par contre, si  $f$ est bijective, comme  $E$ et $F$ sont deux $\Rr$-espaces vectoriels de  dimensions finies, du théorème du rang, on déduit que $\dim E = \dim F$.
\end{explanations}
\end{question}


\begin{question}
\qtags{motcle=Famille libre/Génératrice/Noyau/Image/Injection/Surjection}
Soit $E$ et $F$ deux $\Rr$-espaces vectoriels et $f$ une application linéaire de $E$ dans $F$. Soit $k\in \Nn^*$, ${\cal F} = \{u_1,u_2, \dots , u_k\}$ une famille de vecteurs de $E$ et ${\cal F}' = \{f(u_1),f(u_2), \dots , f(u_k)\}$. Quelles sont les assertions vraies ?
\begin{answers}  
\bad{Si ${\cal F}$ est une famille libre, alors ${\cal F}'$ est une famille libre}.
\good{Si ${\cal F}$ est une famille libre  et $f$ est injective, alors ${\cal F}'$ est une famille libre}.
\bad{Si ${\cal F}$ est une famille génératrice de $E$, alors ${\cal F}'$ est  une famille génératrice de $F$}.
\good{Si ${\cal F}$ est une famille génératrice de $E$ et $f$ est surjective, alors ${\cal F}'$ est une famille génératrice de $F$}.
\end{answers}
\begin{explanations} Si ${\cal F}$ est libre, ${\cal F}'$ n'est pas nécessairement libre. Exemple : l'application nulle de $\Rr^2$ dans $\Rr^2$ et ${\cal F}$ la base canonique de $\Rr^2$.
\vskip0mm
Par contre, si de plus $f$ est injective, alors ${\cal F}'$ est injective. En effet, soit $\lambda_1, \dots ,\lambda_k $ des réels tels que $\lambda_1f(u_1)+ \dots \lambda_kf(u_k)=0 $, alors 
$\lambda_1u_1+ \dots \lambda_ku_k \in \ker f$ et comme $f$ est injective, $\lambda_1u_1+ \dots \lambda_ku_k=0$. Puisque ${\cal F}$ est libre, on déduit que  $\lambda_1= \dots= \lambda_k=0 $.
\vskip0mm
Si ${\cal F}$ est génératrice de $E$, ${\cal F}'$ n'est pas nécessairement génératrice de $F$. Exemple : l'application nulle de 
$\Rr^2$ dans $\Rr^2$ et ${\cal F}$ la base canonique de $\Rr^2$.
\vskip0mm
Par contre, si de plus $f$ est surjective, alors ${\cal F}'$ est génératrice de $F$. En effet, puisque $f$ est surjective, $F=f(E)$ et comme  $E=\mbox{Vect} \, \left({\cal F}\right)$, on déduit que $F=\mbox{Vect} \, \left({\cal F'}\right)$.
\end{explanations}
\end{question}

\subsection{Noyau et image  | Niveau 4}

\begin{question}
\qtags{motcle=Noyau/Image/injection/Surjection}
On considère $E$ un $\Rr$-espace vectoriel et $f$ un endomorphisme de $E$ tel que : $f^2+f+Id = 0$, où $Id$ est l'application identité de $E$. Quelles sont les assertions vraies ?
\begin{answers} 
\bad{$\dim \ker f = 1$}.
\good{$f$ est injective}.
\good{$f$ est bijective et $f^{-1}=f^2$}.
\good{$f$ est  bijective et $f^{-1}=-f-Id$}.
\end{answers}
\begin{explanations} Soit $x\in E$ tel que $f(x)=0$. De l'égalité  $f^2(x)+f(x)+x = 0$, on déduit que $x=0$, donc $\dim \ker f = 0$ et donc $f$ est injective.
\vskip0mm
De l'égalité  $f^2+f+Id = 0$, on déduit que $fo(-f-Id)=Id$, donc $f$ est bijective et $f^{-1}=-f-Id=f^2$.
\end{explanations}
\end{question}

\begin{question}
\qtags{motcle=Espaces supplémentaires/Noyau/Image}
Soit $E$ un espace vectoriel, $F$ et $G$ deux sous-espaces supplémentaires dans $E$ et $f$ l'application de $E$ dans $E$ définie par : $$\begin{array}{rccc}f:&E=F\oplus G&\to&E\\
& x=x_1+x_2, \; (x_1\in F, x_2 \in G)&\to &x_1.  \end{array}$$ 
$f$ est appelée la projection vectorielle de $E$ sur $F$ parallèlement à $G$. Quelles sont les assertions vraies ?
\begin{answers}  
\good{$f$ est un endomorphisme de $E$}.
\bad{$f^2=0$}.
\good{$f^2=f$}.
\good{$F=\Im f$ et $G=\ker f$}.  
\end{answers}
\begin{explanations} On vérifie que $f$ est un endomorphisme, $f^2=f$, $\ker f=G$ et $\Im f=F$.
\end{explanations}
\end{question}

\begin{question}
\qtags{motcle=Espaces supplémentaires/Symétrie vectorielle}
Soit $E$ un espace vectoriel, $F$ et $G$ deux sous-espaces supplémentaires dans $E$ et $f$ l'application de $E$ dans $E$ définie par : $$\begin{array}{rccc}f:&E=F\oplus G&\to&E\\
& x=x_1+x_2, \; (x_1\in F, x_2 \in G)&\to &x_1-x_2.  \end{array}$$ 
$f$ est appelée la symétrie vectorielle de $E$ par rapport à $F$ parallèlement à $G$. Quelles sont les assertions vraies ?
\begin{answers}  
\good{$f$ est un endomorphisme de $E$}.
\bad{$f^2=f$}.
\good{$f^2=Id$, où $Id$ est l'identité de $E$}.
\good{$F=\{x\in E\; ;\; f(x)=x\}$ et $G=\{x\in E\; ;\; f(x)=-x\}$}.
\end{answers}
\begin{explanations} On vérifie que $f$ est un endomorphisme, $f^2=Id$, $F=\{x\in E \; ; \; f(x)=x\}$ et $G=\{x\in E \; ; \; f(x)=-x\}$.
\end{explanations}
\end{question}

\begin{question}
\qtags{motcle=Noyau/Image/Espaces supplémentaires}
Soit $E$ un espace vectoriel et $f$ un projecteur de $E$, c.à.d. un endomorphisme de $E$ tel que $f^2=f$. On notera $Id$ l'identité de $E$. Quelles sont les assertions vraies ?
\begin{answers}  
\bad{$f$ est injective}.
\good{$Id-f$ est un projecteur de $E$}.
\good{$E=\ker f\oplus \Im f $}.
\good{$\Im f =\ker (Id- f) $}.
\end{answers}
\begin{explanations} $f$ n'est pas nécessairement injective. Contre exemple : l'application nulle de $\Rr $ dans $ \Rr$.
\vskip0mm
Comme $f^2=f$, $(Id - f)o(Id - f)=Id - 2f+f^2= Id - f$, donc $Id - f$ est un projecteur.
\vskip0mm
Soit $y\in  \Im f \cap \ker f$, alors il existe $x\in E$ tel que $y=f(x)$ et $f(y)=0$, donc $f^2(x)=0$. Or $f^2=f$, on déduit que $y=0$.
\vskip0mm
Soit $x\in E$, alors $x= f(x) + (x-f(x))$. Comme $f(x)\in\Im f$ et $x-f(x)\in\ker f$, on déduit que $E=\ker f+\Im f $. Par conséquent,   $E=\ker f \oplus \Im f$.
\vskip0mm
Soit $y\in\Im f,$  alors il existe $x\in E$ tel que $y=f(x)$, donc $(Id-f)(y)=f(x)-f^2(x)=0,$ puisque $f^2=f$. Réciproquement, si $(Id - f)(y)=0$, alors $y=f(y)$ et donc $y\in \Im f$. Par conséquent,  $ \Im f =\ker (Id - f) $.
\end{explanations}
\end{question}

\begin{question}
\qtags{motcle=Endomorphisme/Injectif/Surjectif/Bijectif}
Soit $E$ un espace vectoriel et $f$ un endomorphisme nilpotent de $E$, c.à.d. un endomorphisme non nul de $E$ tel qu'il existe un entier $n\ge 2$, vérifiant  $f^n=0$. On notera $Id$ l'identité de $E$. Quelles sont les assertions vraies ?
\begin{answers}  
\bad{$f$ est injective}.
\bad{$f$ est surjective}.
\good{$Id-f$ est injective}.
\good{$Id-f$ est bijective}.
\end{answers}
\begin{explanations} Soit $f$ l'endomorphisme de $\Rr^2$ défini par $f(x,y)=(x-y,x-y)$. Alors, $f^2=0$ et $f$ n'est ni injective, ni surjective.
\vskip0mm
D'une manière générale, si $f$ est nilpotent, $f$ n'est pas bijectif. En effet, supposons qu'il existe une application $g$ telle que $gof=Id$ et considérons un élément $x\in E$ tel que $f(x)\neq 0$ et $k$ le plus petit entier $\ge 2$ tel que $f^k(x)=0$. Alors, $g(f^k(x))=0=gof(f^{k-1}(x)) = f^{k-1}(x)$, ce qui est absurde.
\vskip0mm
Soit $x\in E$ tel que $(Id-f)(x)=0,$ alors $f(x)=x$ et, par récurrence, $f^n(x)=x$. Or $f^n=0$, donc $x=0$. On en déduit que  $Id-f$ est injective.
\vskip0mm
De l'égalité : $(Id-f)(Id+f+f^2+\dots + f^{n-1})=Id-f^n=Id$, on déduit que $Id-f$ est bijective et que $(Id-f)^{-1}=Id+f+f^2+\dots + f^{n-1}$.
\end{explanations}
\end{question}

\begin{question}
\qtags{motcle=Noyau/Image/Bijection/Espaces supplémentaires}
Soit $E$ un espace vectoriel et $f$ un endomorphisme involutif de $E$, c.à.d. un endomorphisme non nul de $E$ tel que $f^2=Id$, où $Id$ est l'identité de $E$. Quelles sont les assertions vraies ?
\begin{answers}  
\good{$f$ est bijective}.
\bad{$\Im (Id+f) \cap \Im (Id-f) = E$}.
\good{$E=\Im (Id+f) + \Im (Id-f)$}.
\bad{$\Im (Id+f)$ et $\Im (Id-f)$ ne sont pas supplémentaires dans $E$}.
\end{answers}
\begin{explanations} De l'égalité $f^2=Id$, on déduit que $f$ est bijective et $f^{-1}=f$.
\vskip0mm
Soit $y\in \Im (Id+f) \cap \Im (Id-f)$, alors il existe $x,x'\in E$ tels que $y=x+f(x)=x'-f(x')$. De l'égalité $f^2=Id$, on déduit que  $f(y)=f(x)+x=f(x')-x'=y=-y$, donc $y=0$.
\vskip0mm
Soit $x\in E$, alors $\displaystyle x=\frac{1}{2}(x+f(x))+\frac{1}{2}(x-f(x)) \in \Im (Id+f)+\Im (Id-f)$. On en déduit que $E= \Im (Id+f) + \Im (Id-f)$. Par conséquent, $\Im (Id+f)$ et $\Im (Id-f)$ sont supplémentaires dans $E$.
\end{explanations}
\end{question}

\begin{question}
\qtags{motcle=Noyau/Image/Bijection}
Soit $E$ un espace vectoriel  et $f$ un endomorphisme de $E$. Quelles sont les assertions vraies ?
\begin{answers} 
\bad{Si $f^2 =0$, alors $f=0$}.
\bad{Si $f^2 =0$, alors $f$ est bijective}.
\good{Si $f^2 =0$, alors $\Im f \subset \ker f$}.
\good{Si $\Im f \subset \ker f$, alors  $f^2 =0$}.
\end{answers}
\begin{explanations} Si $f^2 =0$, $f$ n'est pas forcément nulle. Exemple : $f:\Rr^2 \to \Rr^2$ définie par : $f(x,y)=(x-y,x-y)$.
\vskip0mm
Supposons que $f^2 =0$ et que $f\neq 0$. Alors, il existe $x_0\in E$ tel que $f(x_0) \neq 0$ et $f(x_0) \in \ker f$, puisque $f^2(x_0) =0$. Donc $f$ est non injective et donc non bijective.
\vskip0mm
Soit $y\in \Im f$, alors il existe $x\in E$ tel que $y=f(x)$. De l'égalité $f^2 =0$, on déduit que $f(y)=0$.
\vskip0mm
Soit $x\in E$, alors $f(x)\in \Im f$. Comme  $\Im f \subset \ker f$, $f^2(x)=0$.
\end{explanations}
\end{question}
  
\begin{question}
\qtags{motcle=Noyau/Image/Espaces supplémentaires}
Soit $E$ un $\Rr$-espace vectoriel de dimension finie et $f$ un endomorphisme de $E$. Quelles sont les assertions vraies ?
\begin{answers}  
\bad{$E= \ker f \oplus \Im f$}.
\good{Si $\ker f= \ker f^2$, alors $E= \ker f \oplus \Im f$}.
\good{Si $\Im f= \Im f^2$, alors $\ker f= \ker f^2$}.
\good{Si $E= \ker f \oplus \Im f $, alors $\ker f= \ker f^2$}.
\end{answers}
\begin{explanations} Le noyau et l'image d'un endomorphisme ne sont pas nécessairement supplémentaires.
\vskip0mm
Soit $y\in \ker f \cap \Im f$. Donc $f(y)=0$ et il existe $x\in E$ tel que $y=f(x)$, donc  $f^2(x)=0$. Or $\ker f^2 \subset ker f$, donc $f(x)=0$ et donc $y=0$. En utilisant le théorème du rang, on déduit que $E= \ker f \oplus \Im f $.
\vskip0mm
D'après le théorème du rang, $\dim \ker f+ \dim \Im f = \dim \ker f^2+ \dim \Im f^2=\dim E$. Si  $\Im f= \Im f^2$, 
on déduit que $\dim \ker f= \dim \ker f^2$. Or $\ker f$ est un sous-espace de $\ker f^2$, donc $\ker f= \ker f^2$.
\vskip0mm
On suppose que $E= \ker f \oplus \Im f $. Soit $x\in \ker f^2$, alors $f^2(x)=f(f(x))=0$. Donc $f(x) \in \Im f \cap \ker f$. Or $\Im f \cap \ker f = \{0\}$, donc $x\in \ker f$ et donc $\ker f^2 \subset \ker f$. On déduit que $\ker f^2 = \ker f$, puisque $\ker f \subset \ker f^2$ pour tout endomorphisme $f$.
\end{explanations}
\end{question}
