

\begin{multi}[multiple,feedback=
{On a : \(\displaystyle (\sqrt{u})'=\frac{u'}{2\sqrt{u}}\), \(\displaystyle (\arctan (u))'=\frac{u'}{1+u^2}\) et \(\left(\mathrm{e}^u\right)'=u'\mathrm{e}^u\). Donc \(\displaystyle 2\sqrt{u}\) est une primitive de \(\displaystyle \frac{u'}{\sqrt{u}}\), \(\displaystyle \arctan (u)\) est une primitive de \(\displaystyle \frac{u'}{1+u^2}\) et \(\mathrm{e}^u\) n'est pas une primitive de \(\displaystyle \mathrm{e}^u\). Enfin, \(\ln (u)\) peut ne pas être définie. Il suffit de prendre \(u=-1-x^2\), par exemple.
}]{Question}
Parmi les affirmations suivantes, cocher celles qui sont vraies :

    \item* Si \(u\) est une fonction dérivable, strictement positive, alors \(\sqrt{u}\) est une primitive de \(\displaystyle \frac{u'}{2\sqrt{u}}\).
    \item* Si \(u\) est une fonction dérivable, alors \(\arctan (u)\) est une primitive de \(\displaystyle \frac{u'}{1+u^2}\).
    \item Si \(u\) est une fonction dérivable, alors \(\mathrm{e}^u\) est une primitive de \(\displaystyle \mathrm{e}^u\).
    \item Si \(u\) est une fonction dérivable et ne s'annulant pas, alors \(\ln (u)\) est une primitive de \(\displaystyle \frac{u'}{u}\).
\end{multi}


\begin{multi}[multiple,feedback=
{On calcule \(F'(x)\). Il en découle que \(\displaystyle x^2+\frac{\mathrm{e}^{2x}}{2}\) est une primitive de \(\displaystyle 2x+\mathrm{e}^{2x}\) et que \(\displaystyle (2x-2)\mathrm{e}^{x}\) est une primitive de \(\displaystyle 2x\mathrm{e}^{x}\).
}]{Question}
Parmi les affirmations suivantes, cocher celles qui sont vraies :

    \item \(\displaystyle F(x)=x^2+\mathrm{e}^{x^2}\) est une primitive de \(\displaystyle f(x)=2x+\mathrm{e}^{2x}\) sur \(\Rr\).
    \item* \(\displaystyle F(x)=x^2+\frac{\mathrm{e}^{2x}}{2}\) est une primitive de \(\displaystyle f(x)=2x+\mathrm{e}^{2x}\) sur \(\Rr\).
    \item \(\displaystyle F(x)=x^2\mathrm{e}^{x}\) est une primitive de \(\displaystyle f(x)=2x\mathrm{e}^{x}\) sur \(\Rr\).
    \item* \(\displaystyle F(x)=(2x-2)\mathrm{e}^{x}\) est une primitive de \(\displaystyle f(x)=2x\mathrm{e}^{x}\) sur \(\Rr\).
\end{multi}


\begin{multi}[multiple,feedback=
{Le calcul de \(F'(x)\) montre que \(\displaystyle 2\sqrt{x+1}+\mathrm{e}^{x}\) est une primitive de \(\displaystyle \frac{1}{\sqrt{x+1}}+\mathrm{e}^{x}\) sur \(]-1,+\infty[\). Donc, toute primitive de \(\displaystyle f(x)\) sur \(]-1,+\infty[\) est de la forme : \(\displaystyle F(x)=2\sqrt{x+1}+\mathrm{e}^{x}+k\), \(k\in \Rr\), et celle qui vérifie \(F(0)=0\) est \(\displaystyle F(x)=2\sqrt{x+1}+\mathrm{e}^{x}-3\). 
}]{Question}
Parmi les affirmations suivantes, cocher celles qui sont vraies :

    \item \(\displaystyle F(x)=\sqrt{x+1}+\mathrm{e}^{x}\) est une primitive de \(\displaystyle f(x)=\frac{1}{\sqrt{x+1}}+\mathrm{e}^{x}\) sur \(]-1,+\infty[\).
    \item* \(\displaystyle F(x)=2\sqrt{x+1}+\mathrm{e}^{x}\) est une primitive de \(\displaystyle f(x)=\frac{1}{\sqrt{x+1}}+\mathrm{e}^{x}\) sur \(]-1,+\infty[\).
    \item La primitive de \(\displaystyle f(x)=\frac{1}{\sqrt{x+1}}+\mathrm{e}^{x}\) sur \(]-1,+\infty[\) qui s'annule en \(0\) est \(\displaystyle F(x)=\sqrt{x+1}+\mathrm{e}^{x}-2\).
    \item* La primitive de \(\displaystyle f(x)=\frac{1}{\sqrt{x+1}}+\mathrm{e}^{x}\) sur \(]-1,+\infty[\) qui s'annule en \(0\) est \(\displaystyle F(x)=2\sqrt{x+1}+\mathrm{e}^{x}-3\).
\end{multi}


\begin{multi}[multiple,feedback=
{On a : \(\displaystyle \Big[(x+1)^2+\cos (2x)\Big]'=2x+2-2\sin (2x)\). Donc \(\displaystyle (x+1)^2+\cos (2x)\) est une primitive de \(\displaystyle 2x+2-2\sin (2x)\) sur \(\Rr\). De même, \(\displaystyle -2x\cos (2x)+\sin (2x)\) est une primitive de \(\displaystyle 4x\sin (2x)\) sur \(\Rr\).
}]{Question}
Parmi les affirmations suivantes, cocher celles qui sont vraies :

    \item* \(\displaystyle F(x)=(x+1)^2+\cos (2x)\) est une primitive de \(\displaystyle f(x)=2x+2-2\sin (2x)\) sur \(\Rr\).
    \item \(\displaystyle F(x)=x^2+2x-\cos (2x)\) est une primitive de \(\displaystyle f(x)=2x+2-2\sin (2x)\) sur \(\Rr\).
    \item \(\displaystyle F(x)=2x^2\cos (x^2)\) est une primitive de \(\displaystyle f(x)=4x\sin (2x)\) sur \(\Rr\).
    \item* \(\displaystyle F(x)=-2x\cos (2x)+\sin (2x)\) est une primitive de \(\displaystyle f(x)=4x\sin (2x)\) sur \(\Rr\).
\end{multi}


\begin{multi}[multiple,feedback=
{\(\displaystyle \left[\frac{1}{2}\mathrm{e}^{-x}\left(\sin x-\cos x\right)\right]'=\mathrm{e}^{-x}\cos x\) et \(\displaystyle \left[\frac{1}{\sqrt{2}}\mathrm{e}^{-x}\sin \left(x-\frac{\pi}{4}\right)\right]'=\mathrm{e}^{-x}\cos x\). Donc \(\displaystyle \frac{1}{2}\mathrm{e}^{-x}\left(\sin x-\cos x\right)\) et \(\displaystyle \frac{1}{\sqrt{2}}\mathrm{e}^{-x}\sin \left(x-\frac{\pi}{4}\right)\) sont des primitives de \(\displaystyle \mathrm{e}^{-x}\cos x\) sur \(\Rr\).
}]{Question}
Parmi les affirmations suivantes, cocher celles qui sont vraies :

    \item Une primitive de \(\displaystyle f(x)=\mathrm{e}^{-x}\cos x\) sur \(\Rr\) est \(\displaystyle F(x)=\mathrm{e}^{-x}\cos x\).
    \item Une primitive de \(\displaystyle f(x)=\mathrm{e}^{-x}\cos x\) sur \(\Rr\) est \(\displaystyle F(x)=\frac{1}{2}\mathrm{e}^{-x}\left(\cos x+\sin x\right)\).
    \item* Une primitive de \(\displaystyle f(x)=\mathrm{e}^{-x}\cos x\) sur \(\Rr\) est \(\displaystyle F(x)=\frac{1}{2}\mathrm{e}^{-x}\left(\sin x-\cos x\right)\).
    \item* Une primitive de \(\displaystyle f(x)=\mathrm{e}^{-x}\cos x\) sur \(\Rr\) est \(\displaystyle F(x)=\frac{1}{\sqrt{2}}\mathrm{e}^{-x}\sin \left(x-\frac{\pi}{4}\right)\).
\end{multi}


\begin{multi}[multiple,feedback=
{On a : \(\displaystyle \int \frac{2\, \mathrm{d}x}{x}=\ln (x^2)+k\) et \(\displaystyle \int\mathrm{e}^{2x}\mathrm{d}x=\frac{1}{2}\mathrm{e}^{2x}+k\). D'où, par linéarité,
\[\int \left(\frac{2}{x}+\mathrm{e}^{2x}\right)\mathrm{d}x=\ln (x^2)+\frac{1}{2}\mathrm{e}^{2x}+k.\]
De même, on vérifie que : \(\displaystyle \int \left(\frac{1}{2\sqrt{x}}-\sin x\right)\mathrm{d}x=\sqrt{x}+\cos x+k\), où \(k\in \Rr\).
}]{Question}
Parmi les affirmations suivantes, cocher celles qui sont vraies. Sur \(]0,+\infty[\), on a :

    \item \(\displaystyle \int \left(\frac{2}{x}+\mathrm{e}^{2x}\right)\mathrm{d}x=\frac{-2}{x^2}+2\mathrm{e}^{2x}+k\), où \(k\in \Rr\).
    \item* \(\displaystyle \int \left(\frac{2}{x}+\mathrm{e}^{2x}\right)\mathrm{d}x=\ln (x^2)+\frac{1}{2}\mathrm{e}^{2x}+k\), où \(k\in \Rr\).
    \item* \(\displaystyle \int \left(\frac{1}{2\sqrt{x}}-\sin x\right)\mathrm{d}x=\sqrt{x}+\cos x+k\), où \(k\in \Rr\).
    \item \(\displaystyle \int \left(\frac{1}{2\sqrt{x}}-\sin x\right)\mathrm{d}x=\sqrt{x}-\cos x+k\), où \(k\in \Rr\).
\end{multi}


\begin{multi}[multiple,feedback=
{Avec \(u=1+x\), on a : \(\mathrm{d}u=\mathrm{d}x\),
\[\int \frac{\mathrm{d}x}{(x+1)^3}=\int u^{-3}\mathrm{d}u=\frac{-1}{2}u^{-2}+k=\frac{-1}{2(x+1)^2}+k,\; k\in \Rr,\]
et \(\displaystyle \int \frac{\mathrm{d}x}{x+1}=\int \frac{\mathrm{d}u}{u}=\ln |u|+k=\ln (x+1)+k\), \(k\in \Rr\).
}]{Question}
Parmi les affirmations suivantes, cocher celles qui sont vraies. Sur \(]-1,+\infty[\), on a :

    \item \(\displaystyle \int \frac{\mathrm{d}x}{(x+1)^3}=\frac{-3}{(x+1)^4}+k\), où \(k\in \Rr\).
    \item* \(\displaystyle \int \frac{\mathrm{d}x}{(x+1)^3}=\frac{-1}{2(x+1)^2}+k\), où \(k\in \Rr\).
    \item* \(\displaystyle \int \frac{\mathrm{d}x}{x+1}=\ln (x+1)+k\), où \(k\in \Rr\).
    \item \(\displaystyle \int \frac{\mathrm{d}x}{x+1}=\frac{-1}{(x+1)^2}+k\), où \(k\in \Rr\).
\end{multi}


\begin{multi}[multiple,feedback=
{Par linéarité, les primitives de \(\displaystyle \cos (\pi x)+\frac{1}{x}\) sur \(]0,+\infty[\) sont
\[F(x)=\int \left(\cos (\pi x)+\frac{1}{x}\right)\mathrm{d}x=\frac{\sin (\pi x)}{\pi}+\ln x+k,\; k\in \Rr.\]
Ensuite \(F(1)=0\Rightarrow k=0\). Donc la primitive de \(\displaystyle \cos (\pi x)+\frac{1}{x}\) sur \(]0,+\infty[\) qui s'annule en \(1\) est \(\displaystyle F(x)=\frac{\sin (\pi x)}{\pi}+\ln x\).
}]{Question}
Parmi les assertions suivantes, cocher celles qui sont vraies :

    \item Une primitive de \(\displaystyle \cos (\pi x)+\frac{1}{x}\) sur \(]0,+\infty[\) est \(\displaystyle \sin (\pi x)+\ln x\).
    \item* Une primitive de \(\displaystyle \cos (\pi x)+\frac{1}{x}\) sur \(]0,+\infty[\) est \(\displaystyle \frac{\sin (\pi x)}{\pi}+\ln (\pi x)\).
    \item La primitive de \(\displaystyle \cos (\pi x)+\frac{1}{x}\) sur \(]0,+\infty[\) qui s'annule en \(1\) est \(\displaystyle \sin (\pi x)+\ln x\).
    \item* La primitive de \(\displaystyle \cos (\pi x)+\frac{1}{x}\) sur \(]0,+\infty[\) qui s'annule en \(1\) est \(\displaystyle \frac{\sin (\pi x)}{\pi}+\ln x\).
\end{multi}


\begin{multi}[multiple,feedback=
{Une intégration par parties avec \(u=x\) et \(v=\mathrm{e}^x\) donne
\[F(x)=x\mathrm{e}^x-\int \mathrm{e}^x\mathrm{d}x=x\mathrm{e}^x-\mathrm{e}^x+k,\mbox{ où }k\in \Rr.\]
}]{Question}
On note par \(F\) une primitive de \(f(x)=x\mathrm{e}^x\) sur \(\Rr\). Parmi les affirmations suivantes, cocher celles qui sont vraies :

    \item \(\displaystyle F(x)=x\ln x+k\), où \(k\in \Rr\).
    \item \(\displaystyle F(x)=\frac{x^2}{2}\mathrm{e}^x+k\), où \(k\in \Rr\).
    \item* \(\displaystyle F(x)=x\mathrm{e}^x-\int \mathrm{e}^x\mathrm{d}x\).
    \item* \(\displaystyle F(x)=(x-1)\mathrm{e}^x+k\), où \(k\in \Rr\).
\end{multi}


\begin{multi}[multiple,feedback=
{Une intégration par parties avec \(u=\ln x\) et \(v=x\) donne
\[\int \ln x\mathrm{d}x=x\ln x-\int \mathrm{d}x=x\ln x-x+k,\; k\in \Rr.\]
}]{Question}
On note par \(F\) une primitive de \(f(x)=\ln x\) sur \(]0,+\infty[\). Parmi les affirmations suivantes, cocher celles qui sont vraies :

    \item \(\displaystyle F(x)=\mathrm{e}^{x}+k\), \(k\in \Rr\).
    \item* \(\displaystyle F(x)=x\ln x-\int \mathrm{d}x\).
    \item* \(\displaystyle F(x)=x\ln x-x+k\), \(k\in \Rr\).
    \item \(\displaystyle F(x)=\frac{1}{x}+k\), \(k\in \Rr\).
\end{multi}


\begin{multi}[multiple,feedback=
{Avec \(u=1+x^3\), on a : \(\mathrm{d}u=3x^2\mathrm{d}x\) et
\[\int \frac{x^2\mathrm{d}x}{1+x^3}=\frac{1}{3}\int \frac{\mathrm{d}u}{u}=\frac{1}{3}\ln |u|+k=\frac{1}{3}\ln (1+x^3)+k,\; k\in \Rr.\]
De m\^eme, avec \(u=x-1\), on a \(\mathrm{d}u=\mathrm{d}x\) et
\[\int \sqrt{x-1}\mathrm{d}x=\int \sqrt{u}\mathrm{d}u=\frac{2}{3}u^{3/2}+k=\frac{2}{3}(x-1)^{3/2}+k,\; k\in \Rr.\]
}]{Question}
Parmi les affirmations suivantes, cocher celles qui sont vraies :

    \item* Sur \(]-1,+\infty[\), on a : \(\displaystyle \int \frac{x^2\mathrm{d}x}{1+x^3}=\frac{1}{3}\ln (1+x^3)+k\), où \(k\in \Rr\).
    \item Sur \(]-1,+\infty[\), on a : \(\displaystyle \int \frac{x^2\mathrm{d}x}{1+x^3}=\frac{x^3}{3(1+x^3)}+k\), où \(k\in \Rr\).
    \item Sur \(]1,+\infty[\), on a : \(\displaystyle \int \sqrt{x-1}\mathrm{d}x=\frac{1}{\sqrt{x-1}}+k\), où \(k\in \Rr\).
    \item* Sur \(]1,+\infty[\), on a : \(\displaystyle \int \sqrt{x-1}\mathrm{d}x=\frac{2}{3}(x-1)^{3/2}+k\), où \(k\in \Rr\).
\end{multi}


\begin{multi}[multiple,feedback=
{Avec \(u=2+x^4\), on a : \(\mathrm{d}u=4x^3\mathrm{d}x\) et
\[\int 4x^3(2+x^4)^3\mathrm{d}x=\int u^3\mathrm{d}u=\frac{u^4}{4}+k=\frac{(2+x^4)^4}{4}+k,\; k\in \Rr.\]
De m\^eme, avec \(u=1+3x^2\), on a : \(\mathrm{d}u=6x\mathrm{d}x\) et
\[\int \frac{6x\mathrm{d}x}{1+3x^2}=\int \frac{\mathrm{d}u}{u}=\ln (u)+k=\ln(1+3x^2)+k,\; k\in \Rr.\]
}]{Question}
Parmi les égalités suivantes, cocher celles qui sont vraies :

    \item \(\displaystyle \int 4x^3(2+x^4)^3\mathrm{d}x=x^4(2+x^4)^3+k\), où \(k\in \Rr\).
    \item* \(\displaystyle \int 4x^3(2+x^4)^3\mathrm{d}x=\frac{(2+x^4)^4}{4}+k\), où \(k\in \Rr\).
    \item \(\displaystyle \int \frac{6x\mathrm{d}x}{1+3x^2}=6\int x\mathrm{d}x\times\int \frac{\mathrm{d}x}{1+3x^2}=3x^2\times\ln (1+3x^2)+k\), \(k\in \Rr\).
    \item* \(\displaystyle \int \frac{6x\mathrm{d}x}{1+3x^2}=\ln(1+3x^2)+k\), où \(k\in \Rr\).
\end{multi}


\begin{multi}[multiple,feedback=
{Avec \(u=2x^2+1\), on a : \(\mathrm{d}u=4x\mathrm{d}x\) et
\[\int 4x\sqrt{2x^2+1}\mathrm{d}x=\int \sqrt{u}\mathrm{d}u=\frac{2}{3}u^{3/2}+k=\frac{2}{3}(2x^2+1)^{3/2}+k,\; k\in \Rr.\]
De m\^eme, avec \(u=3x^2+2\), on a : \(\mathrm{d}u=6x\mathrm{d}x\) et
\[\int \frac{3x\mathrm{d}x}{\sqrt{3x^2+2}}=\int \frac{\mathrm{d}u}{2\sqrt{u}}=\sqrt{u}+k=\sqrt{3x^2+2}+k,\; k\in \Rr.\]
}]{Question}
Parmi les égalités suivantes, cocher celles qui sont vraies :

    \item* \(\displaystyle \int 4x\sqrt{2x^2+1}\mathrm{d}x=\frac{2}{3}(2x^2+1)^{3/2}+k\), où \(k\in \Rr\).
    \item \(\displaystyle \int \sqrt{2x^2+1}\mathrm{d}x=\frac{2}{3}(2x^2+1)^{3/2}+k\), où \(k\in \Rr\).
    \item* \(\displaystyle \int \frac{3x\mathrm{d}x}{\sqrt{3x^2+2}}=\sqrt{3x^2+2}+k\), où \(k\in \Rr\).
    \item \(\displaystyle \int \frac{\mathrm{d}x}{2\sqrt{3x^2+2}}=\sqrt{3x^2+2}+k\), où \(k\in \Rr\).
\end{multi}


\begin{multi}[multiple,feedback=
{Avec \(u=x^4+1\), on a : \(\mathrm{d}u=4x^3\, \mathrm{d}x\). D'où
\[\int 16x^3(x^4+1)^3\mathrm{d}x=\int 4u^3\, \mathrm{d}u=u^4+k=(x^4+1)^4+k,\; k\in \Rr,\]
et
\[\int \frac{4x^3\, \mathrm{d}x}{\sqrt{x^4+1}}=\int \frac{\mathrm{d}u}{\sqrt{u}}=2\sqrt{u}+k=2\sqrt{x^4+1}+k,\; k\in \Rr.\]
}]{Question}
Parmi les assertions suivantes, cocher celles qui sont vraies :

    \item Les primitives de \(16x^3(x^4+1)^3\) sur \(\Rr\) sont données par \(\displaystyle F(x)=x^4(x^4+1)^4+k\), \(k\in \Rr\).
    \item* Les primitives de \(16x^3(x^4+1)^3\) sur \(\Rr\) sont données par \(\displaystyle F(x)=(x^4+1)^4+k\), \(k\in \Rr\).
    \item* Les primitives de \(\displaystyle \frac{4x^3}{\sqrt{x^4+1}}\) sur \(\Rr\) sont données par \(\displaystyle F(x)=2\sqrt{x^4+1}+k\), \(k\in \Rr\).
    \item Les primitives de \(\displaystyle \frac{4x^3}{\sqrt{x^4+1}}\) sur \(\Rr\) sont données par \(\displaystyle F(x)=\frac{x^4}{\sqrt{x^4+1}}+k\), \(k\in \Rr\).
\end{multi}


\begin{multi}[multiple,feedback=
{Utiliser la linéarité en remarquant que : \(\displaystyle \int \mathrm{e}^{3x}\mathrm{d}x=\frac{\mathrm{e}^{3x}}{3}+k_1\),
\[\int \frac{\mathrm{d}x}{1+4x^2}=\frac{\arctan (2x)}{2}+k_2\quad \mbox{et}\quad \int \frac{4\, \mathrm{d}x}{1+x^2}=4\arctan x+k_3,\; k_i\in \Rr.\]
}]{Question}
Parmi les égalités suivantes, cocher celles qui sont vraies :

    \item \(\displaystyle \int \left(\mathrm{e}^{3x}+\frac{1}{1+4x^2}\right)\mathrm{d}x=\mathrm{e}^{3x}+\arctan (2x)+k,\; k\in \Rr\).
    \item* \(\displaystyle \int \left(\mathrm{e}^{3x}+\frac{1}{1+4x^2}\right)\mathrm{d}x=\frac{\mathrm{e}^{3x}}{3}+\frac{\arctan (2x)}{2}+k,\; k\in \Rr\).
    \item* \(\displaystyle \int \left(\mathrm{e}^{3x}+\frac{4}{1+x^2}\right)\mathrm{d}x=\frac{\mathrm{e}^{3x}}{3}+4\arctan x+k,\; k\in \Rr\).
    \item \(\displaystyle \int \left(\mathrm{e}^{3x}+\frac{4}{1+x^2}\right)\mathrm{d}x=\mathrm{e}^{3x}+4\arctan x+k,\; k\in \Rr\).
\end{multi}


\begin{multi}[multiple,feedback=
{Avec \(u=\ln x\), on aura : \(\displaystyle \mathrm{d}u=\frac{\mathrm{d}x}{x}\). D'où
\[\int \frac{\ln x}{x}\mathrm{d}x=\int u\,\mathrm{d}u=\frac{1}{2}u^2+k=\frac{1}{2}\left(\ln x\right)^2+k,\; k\in \Rr\]
et
\[\int \frac{\mathrm{d}x}{x\ln x}=\int \frac{\mathrm{d}u}{u}=\ln |u|+k=\ln (\ln x)+k,\; k\in \Rr.\]
}]{Question}
Parmi les égalités suivantes, cocher celles qui sont vraies :

    \item Sur \(]0,+\infty[\), on a : \(\displaystyle \int \frac{\ln x}{x}\mathrm{d}x=\int \frac{\mathrm{d}x}{x}\ln x=\ln x\times \ln x+k\), \(k\in \Rr\).
    \item* Sur \(]0,+\infty[\), on a : \(\displaystyle \int \frac{\ln x}{x}\mathrm{d}x=\frac{1}{2}\ln x\times \ln x+k\), \(k\in \Rr\).
    \item* Sur \(]1,+\infty[\), on a : \(\displaystyle \int \frac{\mathrm{d}x}{x\ln x}=\ln (\ln x)+k\), \(k\in \Rr\).
    \item Sur \(]1,+\infty[\), on a : \(\displaystyle \int \frac{\mathrm{d}x}{x\ln x}=\ln (x\ln x)+k\), \(k\in \Rr\).
\end{multi}


\begin{multi}[multiple,feedback=
{Avec \(u=\ln x\), on aura : \(\displaystyle \mathrm{d}u=\frac{\mathrm{d}x}{x}\). D'où
\[\int \frac{\mathrm{d}x}{x\ln ^2x}=\int \frac{\mathrm{d}u}{u^2}=\frac{-1}{u}+k=\frac{-1}{\ln x}+k,\; k\in \Rr\]
et
\[\int \frac{\mathrm{d}x}{2x\sqrt{\ln x}}=\int \frac{\mathrm{d}u}{2\sqrt{u}}=\sqrt{u}+k=\sqrt{\ln x}+k,\; k\in \Rr.\]
}]{Question}
Parmi les égalités suivantes, cocher celles qui sont vraies :

    \item* Sur \(]1,+\infty[\), on a : \(\displaystyle \int \frac{\mathrm{d}x}{x\ln ^2x}=\frac{-1}{\ln x}+k\), \(k\in \Rr\).
    \item Sur \(]1,+\infty[\), on a : \(\displaystyle \int \frac{\mathrm{d}x}{x\ln ^2x}=\ln (x\ln ^2x)+k\), \(k\in \Rr\).
    \item Sur \(]1,+\infty[\), on a : \(\displaystyle \int \frac{\mathrm{d}x}{2x\sqrt{\ln x}}=\int \frac{\mathrm{d}x}{x}\times \frac{1}{2\sqrt{\ln x}}=\frac{\sqrt{\ln x}}{2}+k\), \(k\in \Rr\).
    \item* Sur \(]1,+\infty[\), on a : \(\displaystyle \int \frac{\mathrm{d}x}{2x\sqrt{\ln x}}=\sqrt{\ln x}+k\), \(k\in \Rr\).
\end{multi}


\begin{multi}[multiple,feedback=
{Avec \(u=x^2+1\), on a \(\mathrm{d}u=2x\mathrm{d}x\). D'où
\[\int x\sqrt{x^2+1}\mathrm{d}x=\frac{1}{2}\int \sqrt{u}\mathrm{d}u=\frac{u^{3/2}}{3}+k=\frac{1}{3}(1+x^2)^{3/2}+k,\; k\in \Rr,\]
et
\[\int \frac{x\mathrm{d}x}{\sqrt{x^2+1}}=\int \frac{\mathrm{d}u}{2\sqrt{u}}=\sqrt{u}+k=\sqrt{x^2+1}+k,\; k\in \Rr.\]
}]{Question}
Parmi les affirmations suivantes, cocher celles qui sont vraies :

    \item \(\displaystyle \int x\sqrt{x^2+1}\, \mathrm{d}x=\frac{x^2}{2}(1+x^2)^{3/2}+k\), \(k\in \Rr\).
    \item* \(\displaystyle \int x\sqrt{x^2+1}\, \mathrm{d}x=\frac{1}{3}(1+x^2)^{3/2}+k\), \(k\in \Rr\).
    \item \(\displaystyle \int \frac{x\, \mathrm{d}x}{\sqrt{x^2+1}}=\frac{x^2}{2}\frac{1}{\sqrt{x^2+1}}+k\), \(k\in \Rr\).
    \item* \(\displaystyle \int \frac{x\, \mathrm{d}x}{\sqrt{x^2+1}}=\sqrt{x^2+1}+k\), \(k\in \Rr\).
\end{multi}


\begin{multi}[multiple,feedback=
{Avec \(u=1+x^2\), on a : \(\mathrm{d}u=2x\mathrm{d}x\). D'où
\[\int \frac{x\mathrm{d}x}{1+x^2}=\frac{1}{2}\int \frac{\mathrm{d}u}{u}=\frac{1}{2}\ln (u)+k=\ln\sqrt{1+x^2}+k,\; k\in \Rr,\]
et 
\[\int x\mathrm{e}^{(1+x^2)}\mathrm{d}x=\frac{1}{2}\int \mathrm{e}^u\mathrm{d}u=\frac{1}{2}\mathrm{e}^u+k=\frac{1}{2}\mathrm{e}^{(1+x^2)}+k,\; k\in \Rr.\]
}]{Question}
Parmi les affirmations suivantes, cocher celles qui sont vraies :

    \item \(\displaystyle \int \frac{x\mathrm{d}x}{1+x^2}=\arctan (x)+k\), où \(k\in \Rr\).
    \item \(\displaystyle \int x\mathrm{e}^{(1+x^2)}\mathrm{d}x=\frac{x^2}{2}\mathrm{e}^{(1+x^2)}+k\), où \(k\in \Rr\).
    \item* \(\displaystyle \int \frac{x\mathrm{d}x}{1+x^2}=\ln\sqrt{1+x^2}+k\), où \(k\in \Rr\).
    \item* \(\displaystyle \int x\mathrm{e}^{(1+x^2)}\mathrm{d}x=\frac{1}{2}\mathrm{e}^{(1+x^2)}+k\), où \(k\in \Rr\).
\end{multi}


\begin{multi}[multiple,feedback=
{Avec \(u=1+x^4\), on a : \(\mathrm{d}u=4x^3\mathrm{d}x\). D'où \(\displaystyle \int 4x^3(1+x^4)^3\mathrm{d}x=\int u^3\mathrm{d}u\). De même, avec \(u=2x\), on a : \(\mathrm{d}u=2\mathrm{d}x\). D'où \(\displaystyle \int \frac{\mathrm{d}x}{1+4x^2}=\frac{1}{2}\int \frac{\mathrm{d}u}{1+u^2}\).
}]{Question}
Parmi les affirmations suivantes, cocher celles qui sont vraies :

    \item Le changement de variable \(\displaystyle u=x(1+x^4)\) donne \(\displaystyle \int 4x^3(1+x^4)^3\mathrm{d}x=\int 4u^3\mathrm{d}u\).
    \item* Le changement de variable \(\displaystyle u=1+x^4\) donne \(\displaystyle \int 4x^3(1+x^4)^3\mathrm{d}x=\int u^3\mathrm{d}u\).
    \item Le changement de variable \(\displaystyle u=2x\) donne \(\displaystyle \int \frac{\mathrm{d}x}{1+4x^2}=\int \frac{\mathrm{d}u}{1+u^2}\).
    \item* Le changement de variable \(\displaystyle u=2x\) donne \(\displaystyle \int \frac{\mathrm{d}x}{1+4x^2}=\frac{1}{2}\int \frac{\mathrm{d}u}{1+u^2}\).
\end{multi}


\begin{multi}[multiple,feedback=
{Une intégration par parties, avec \(u=\ln x\) et \(v=x^2/2\), donne
\[F(x)=\frac{x^2}{2}\ln x-\frac{1}{2}\int x\mathrm{d}x=\frac{x^2}{2}\ln x-\frac{x^2}{4}+k,\; k\in \Rr.\]
}]{Question}
On note par \(F\) une primitive de \(f(x)=x\ln x\) sur \(]0,+\infty[\). Parmi les affirmations suivantes, cocher celles qui sont vraies :

    \item \(\displaystyle F(x)=\frac{x^2}{2}\ln x+k\), où \(k\in \Rr\).
    \item* \(\displaystyle F(x)=\frac{x^2}{2}\ln x-\frac{1}{2}\int x\mathrm{d}x\).
    \item \(\displaystyle F(x)=\frac{x^2}{2}\ln x-\frac{x}{2}+k\), où \(k\in \Rr\).
    \item* \(\displaystyle F(x)=\frac{x^2}{2}\ln x-\frac{x^2}{4}+k\), où \(k\in \Rr\).
\end{multi}


\begin{multi}[multiple,feedback=
{Avec \(u=1+x^2\), on a : \(\mathrm{d}u=2x\mathrm{d}x\) et 
\[\int x\cos(1+x^2)\mathrm{d}x=\frac{1}{2}\int \cos (u)\mathrm{d}u=\frac{1}{2}\times \sin (u)+k=\frac{1}{2}\sin(1+x^2)+k,\; k\in \Rr.\]
De même, avec \(u=2+\mathrm{e}^x\), on a : \(\mathrm{d}u=\mathrm{e}^x\mathrm{d}x\) et 
\[\int \mathrm{e}^x\sin\left(2+\mathrm{e}^x\right)\mathrm{d}x=\int \sin (u)\mathrm{d}u=-\cos (u)+k=-\cos\left(2+\mathrm{e}^x\right)+k,\; k\in \Rr.\]
}]{Question}
Parmi les égalités suivantes, cocher celles qui sont vraies :

    \item* \(\displaystyle \int x\cos(1+x^2)\mathrm{d}x=\frac{1}{2}\sin(1+x^2)+k\), où \(k\in \Rr\).
    \item \(\displaystyle \int \mathrm{e}^x\sin\left(2+\mathrm{e}^x\right)\mathrm{d}x=\int \mathrm{e}^x\mathrm{d}x\times \int\sin\left(2+\mathrm{e}^x\right)\mathrm{d}x\).
    \item \(\displaystyle \int x\cos(1+x^2)\mathrm{d}x=\frac{x^2}{2}\sin(1+x^2)+k\), où \(k\in \Rr\).
    \item* \(\displaystyle \int \mathrm{e}^x\sin\left(2+\mathrm{e}^x\right)\mathrm{d}x=-\cos\left(2+\mathrm{e}^x\right)+k\), où \(k\in \Rr\).
\end{multi}


\begin{multi}[multiple,feedback=
{Avec \(u=\sqrt{x}\), on a : \(\displaystyle \mathrm{d}u=\frac{\mathrm{d}x}{2\sqrt{x}}\). D'où
\[\int \frac{\sin (\sqrt{x})}{\sqrt{x}}\mathrm{d}x=2\int \sin (u)\mathrm{d}u=-2\cos (u)+k,\; k\in \Rr.\]
}]{Question}
Le changement de variable \(u=\sqrt{x}\) donne :

    \item \(\displaystyle \int \frac{\sin (\sqrt{x})}{\sqrt{x}}\mathrm{d}x=\int \frac{\sin (u)}{u}\mathrm{d}u\).
    \item \(\displaystyle \int \frac{\sin (\sqrt{x})}{\sqrt{x}}\mathrm{d}x=\cos (u)+k\), \(k\in \Rr\).
    \item* \(\displaystyle \int \frac{\sin (\sqrt{x})}{\sqrt{x}}\mathrm{d}x=2\int \sin (u)\mathrm{d}u\).
    \item* \(\displaystyle \int \frac{\sin (\sqrt{x})}{\sqrt{x}}\mathrm{d}x=-2\cos (u)+k\), \(k\in \Rr\).
\end{multi}


\begin{multi}[multiple,feedback=
{Avec \(\displaystyle u=\frac{1}{x}\), on aura : \(\displaystyle \mathrm{d}u=\frac{-\mathrm{d}x}{x^2}\). D'où
\[\int \frac{\sin (1/x)}{x^2}\mathrm{d}x=-\int \sin (u)\mathrm{d}u=\cos (u)+k=\cos\left(1/x\right)+k,\; k\in \Rr.\]
}]{Question}
Le changement de variable \(\displaystyle u=\frac{1}{x}\) donne :

    \item* \(\displaystyle \int \frac{\sin (1/x)}{x^2}\mathrm{d}x=-\int \sin (u)\mathrm{d}u\).
    \item* \(\displaystyle \int \frac{\sin (1/x)}{x^2}\mathrm{d}x=\cos\left(1/x\right)+k\), \(k\in \Rr\).
    \item \(\displaystyle \int \sin (1/x)\mathrm{d}x=\int \sin (u)\mathrm{d}u\).
    \item \(\displaystyle \int \sin (1/x)\mathrm{d}x=-\cos (u)+k\), \(k\in \Rr\).
\end{multi}


\begin{multi}[multiple,feedback=
{Avec \(u=\sin x\), on a : \(\mathrm{d}u=\cos x\mathrm{d}x\),
\[\int \cos x\sin x\mathrm{d}x=\int u\mathrm{d}u=\frac{1}{2}u^2+k=\frac{1}{2}\sin ^2x+k,\; k\in \Rr,\]
et
\[\int \cos x\sin ^2x\mathrm{d}x=\int u^2\mathrm{d}u=\frac{1}{3}u^3+k=\frac{1}{3}\sin ^3x+k,\; k\in \Rr.\]
}]{Question}
Parmi les affirmations suivantes, cocher celles qui sont vraies :

    \item* \(\displaystyle \int \cos x\sin x\mathrm{d}x=\frac{1}{2}\sin ^2x+k\), où \(k\in \Rr\).
    \item \(\displaystyle \int \cos x\sin x\mathrm{d}x=\frac{1}{2}\cos ^2x+k\), où \(k\in \Rr\).
    \item* \(\displaystyle \int \cos x\sin ^2x\mathrm{d}x=\frac{1}{3}\sin ^3x+k\), où \(k\in \Rr\).
    \item \(\displaystyle \int \cos ^2x\sin ^2x\mathrm{d}x=\frac{1}{4}\sin ^4x+k\), où \(k\in \Rr\).
\end{multi}


\begin{multi}[multiple,feedback=
{On a : \(\cos ^3x=(1-\sin ^2x)\cos x\). Donc, avec \(u=\sin x\), on a : \(\mathrm{d}u=\cos x\, \mathrm{d}x\) et
\[\int \cos ^3x\mathrm{d}x=\int (1-\sin ^2x)\cos x\mathrm{d}x=\int (1-u^2)\mathrm{d}u.\]
De m\^eme, on a : \(\sin ^3x=(1-\cos ^2x)\sin x\). Donc, avec \(u=\cos x\), on a : \(\mathrm{d}u=-\sin x\mathrm{d}x\) et
\[\int \cos ^3x\mathrm{d}x=\int (1-\cos ^2x)\sin x\mathrm{d}x=-\int (1-u^2)\mathrm{d}u.\]
}]{Question}
Parmi les assertions suivantes, cocher celles qui sont vraies :

    \item Le changement de variable \(u=\cos x\) donne \(\displaystyle \int \cos ^3x\mathrm{d}x=\int u^3\mathrm{d}u\).
    \item* Le changement de variable \(u=\sin x\) donne \(\displaystyle \int \cos ^3x\mathrm{d}x=\int (1-u^2)\mathrm{d}u\).
    \item Le changement de variable \(u=\sin x\) donne \(\displaystyle \int \sin ^3x\mathrm{d}x=\int u^3\mathrm{d}u\).
    \item* Le changement de variable \(u=\cos x\) donne \(\displaystyle \int \sin ^3x\mathrm{d}x=\int (u^2-1)\mathrm{d}u\).
\end{multi}


\begin{multi}[multiple,feedback=
{Avec \(u=2+\cos x\), on a : \(\mathrm{d}u=-\sin x\, \mathrm{d}x\). D'où
\[\int \sin x(2+\cos x)^5\mathrm{d}x=-\int u^5\mathrm{d}u=-\frac{u^6}{6}+k,\; k\in \Rr,\]
et
\[\int \sin x(2+\cos x)^{-3}\mathrm{d}x=-\int u^{-3}\mathrm{d}u=\frac{u^{-2}}{2}+k,\; k\in \Rr.\]
}]{Question}
Le changement de variable \(u=2+\cos x\) donne :

    \item* \(\displaystyle \int \sin x(2+\cos x)^5\mathrm{d}x=-\frac{(2+\cos x)^6}{6}+k\), où \(k\in \Rr\).
    \item \(\displaystyle \int (2+\cos x)^5\mathrm{d}x=\frac{((2+\cos x)^5)^6}{6}+k\), où \(k\in \Rr\).
    \item* \(\displaystyle \int \sin x\left(2+\cos x\right)^{-3}\mathrm{d}x=\frac{\left(2+\cos x\right)^{-2}}{2}+k\), où \(k\in \Rr\).
    \item \(\displaystyle \int \left(2+\cos x\right)^{-3}\mathrm{d}x=\frac{\left(2+\cos x\right)^{-2}}{2}+k\), où \(k\in \Rr\).
\end{multi}


\begin{multi}[multiple,feedback=
{Avec \(u=2+\sin x\), on a : \(\mathrm{d}u=\cos x\mathrm{d}x\) et puis
\[\int \cos x\mathrm{e}^{2+\sin x}\mathrm{d}x=\int \mathrm{e}^{u}\mathrm{d}u=\mathrm{e}^{u}+k=\mathrm{e}^{2+\sin x}+k,\; k\in \Rr.\]
Ensuite, \(\displaystyle \int \frac{\cos x\mathrm{d}x}{2+\sin x}=\int \frac{\mathrm{d}u}{u}=\ln |u|+k=\ln (2+\sin x)+k\), \(k\in \Rr\).
}]{Question}
Le changement de variable \(u=2+\sin x\) donne :

    \item* \(\displaystyle \int \cos x\mathrm{e}^{2+\sin x}\mathrm{d}x=\mathrm{e}^{2+\sin x}+k\), où \(k\in \Rr\).
    \item \(\displaystyle \int \mathrm{e}^{2+\sin x}\mathrm{d}x=\int \mathrm{e}^u\mathrm{d}u\).
    \item \(\displaystyle \int \frac{\mathrm{d}x}{2+\sin x}=\int \frac{\mathrm{d}u}u\).
    \item* \(\displaystyle \int \frac{\cos x\mathrm{d}x}{2+\sin x}=\ln(2+\sin x)+k\), où \(k\in \Rr\).
\end{multi}


\begin{multi}[multiple,feedback=
{Avec \(u=\sin x\), on aura \(\displaystyle \mathrm{d}u=\cos x\mathrm{d}x\). D'où
\[\int \frac{\cos x}{1+\sin ^2x}\mathrm{d}x=\int \frac{\mathrm{d}u}{1+u^2}=\arctan (u)+k,\; k\in \Rr.\]
}]{Question}
Le changement de variable \(u=\sin x\) donne :

    \item* \(\displaystyle \int \frac{\cos x}{1+\sin ^2x}\mathrm{d}x=\int \frac{\mathrm{d}u}{1+u^2}\).
    \item* \(\displaystyle \int \frac{\cos x}{1+\sin ^2x}\mathrm{d}x=\arctan (u)+k\), \(k\in \Rr\).
    \item \(\displaystyle \int \frac{\mathrm{d}x}{1+\sin ^2x}=\int \frac{\mathrm{d}u}{1+u^2}\).
    \item \(\displaystyle \int \frac{\mathrm{d}x}{1+\sin ^2x}=\arctan (u)+k\), \(k\in \Rr\).
\end{multi}


\begin{multi}[multiple,feedback=
{Avec \(u=1+\sin (2x)\), on aura \(\displaystyle \mathrm{d}u=2\cos (2x)\mathrm{d}x\). D'où
\[\int \cos (2x)\sqrt{1+\sin (2x)}\mathrm{d}x=2\int \sqrt{u}\mathrm{d}u=\frac{1}{3}u^{3/2}+k,\; k\in \Rr.\]
De même, avec \(u=2-\cos (3x)\), on aura \(\displaystyle \mathrm{d}u=3\sin(3x)\mathrm{d}x\). D'où
\[\int \frac{\sin (3x)\mathrm{d}x}{2-\cos (3x)}=\frac{1}{3}\int \frac{\mathrm{d}u}{u}=\frac{1}{3}\ln |u|+k,\; k\in \Rr.\]
}]{Question}
Parmi les égalités suivantes, cocher celles qui sont vraies :

    \item \(\displaystyle \int \cos (2x)\sqrt{1+\sin (2x)}\mathrm{d}x=\frac{\cos (2x)}{\sqrt{1+\sin (2x)}}+k\), où \(k\in \Rr\).
    \item* \(\displaystyle \int \cos (2x)\sqrt{1+\sin (2x)}\mathrm{d}x=\frac{1}{3}\left[1+\sin (2x)\right]^{3/2}+k\), où \(k\in \Rr\).
    \item* \(\displaystyle \int \frac{\sin (3x)\mathrm{d}x}{2-\cos (3x)}=\frac{1}{3}\ln \left[2-\cos (3x)\right]+k\) où \(k\in \Rr\).
    \item \(\displaystyle \int \frac{\mathrm{d}x}{2-\cos (3x)}=\ln \left[2-\cos (3x)\right]+k\), \(k\in \Rr\).
\end{multi}


\begin{multi}[multiple,feedback=
{Avec \(u=\mathrm{e}^x\), on aura : \(\mathrm{d}u=\mathrm{e}^x\mathrm{d}x\). D'où
\[\displaystyle \int \frac{\mathrm{e}^x\mathrm{d}x}{1+\mathrm{e}^{2x}}=\int \frac{\mathrm{d}u}{1+u^2}=\arctan(u)+k,\; k\in \Rr.\]
}]{Question}
Le changement de variable \(u=\mathrm{e}^x\) donne :

    \item* \(\displaystyle \int \frac{\mathrm{e}^x\mathrm{d}x}{1+\mathrm{e}^{2x}}=\int \frac{\mathrm{d}u}{1+u^2}\).
    \item* \(\displaystyle \int \frac{\mathrm{e}^x\mathrm{d}x}{1+\mathrm{e}^{2x}}=\arctan(u)+k\), \(k\in \Rr\).
    \item \(\displaystyle \int \frac{\mathrm{d}x}{1+\mathrm{e}^{2x}}=\int \frac{\mathrm{d}u}{1+u^2}\).
    \item \(\displaystyle \int \frac{\mathrm{d}x}{1+\mathrm{e}^{2x}}=\arctan(u)+k\), \(k\in \Rr\).
\end{multi}


\begin{multi}[multiple,feedback=
{Avec \(u=\mathrm{e}^{-x}\), on aura : \(\mathrm{d}u=-\mathrm{e}^{-x}\mathrm{d}x\). D'où
\[\displaystyle \int \frac{\mathrm{d}x}{\sqrt{\mathrm{e}^{2x}-1}}=\int \frac{\mathrm{e}^{-x}\mathrm{d}x}{\sqrt{1-\mathrm{e}^{-2x}}}=\int \frac{-\mathrm{d}u}{\sqrt{1-u^2}}=-\arcsin(u)+k,\; k\in \Rr.\]
Et
\[\displaystyle \int \frac{\mathrm{e}^{-x}\mathrm{d}x}{\sqrt{\mathrm{e}^{2x}-1}}=\int \frac{\mathrm{e}^{-2x}\mathrm{d}x}{\sqrt{1-\mathrm{e}^{-2x}}}=\int \frac{-u\mathrm{d}u}{\sqrt{1-u^2}}=\sqrt{(1-u^2)}+k,\; k\in \Rr.\]
}]{Question}
On se place sur \(]0,+\infty[\). Le changement de variable \(u=\mathrm{e}^{-x}\) donne

    \item \(\displaystyle \int \frac{\mathrm{d}x}{\sqrt{\mathrm{e}^{2x}-1}}=\int \frac{\mathrm{d}u}{\sqrt{1-u^2}}\).
    \item* \(\displaystyle \int \frac{\mathrm{d}x}{\sqrt{\mathrm{e}^{2x}-1}}=-\arcsin (u)+k\), \(k\in \Rr\).
    \item \(\displaystyle \int \frac{\mathrm{e}^{-x}\mathrm{d}x}{\sqrt{\mathrm{e}^{2x}-1}}=\int \frac{u\mathrm{d}u}{\sqrt{1-u^2}}\).
    \item* \(\displaystyle \int \frac{\mathrm{e}^{-x}\mathrm{d}x}{\sqrt{\mathrm{e}^{2x}-1}}=\sqrt{(1-u^2)}+k\), \(k\in \Rr\).
\end{multi}


\begin{multi}[multiple,feedback=
{Une intégration par parties avec \(u=\arcsin (x)\) et \(v=x\) donne
\[F(x)=x\arcsin (x)-\int \frac{x\mathrm{d}x}{\sqrt{1-x^2}}=x\arcsin (x)+\sqrt{1-x^2}+k,\; k\in \Rr.\]
}]{Question}
On note par \(F\) une primitive de \(f(x)=\arcsin (x)\) sur \(]-1,1[\). Parmi les propositions suivantes, cocher celles qui sont vraies :

    \item \(\displaystyle F(x)=\frac{1}{\sqrt{1-x^2}}+k\), où \(k\in \Rr\).
    \item* \(\displaystyle F(x)=x\arcsin (x)-\int \frac{x\mathrm{d}x}{\sqrt{1-x^2}}\).
    \item \(\displaystyle F(x)=\arccos (x)+k\), où \(k\in \Rr\).
    \item* \(\displaystyle F(x)=x\arcsin (x)+\sqrt{1-x^2}+k\), où \(k\in \Rr\).
\end{multi}


\begin{multi}[multiple,feedback=
{Une intégration par parties avec \(u=\arctan (x)\) et \(v=x\) donne
\[\int \arctan (x)\mathrm{d}x=x\arctan (x)-\int \frac{x\mathrm{d}x}{1+x^2}=x\arctan (x)-\frac{1}{2}\ln (1+x^2)+k,\; k\in \Rr.\]
}]{Question}
On note par \(F\) une primitive de \(f(x)=\arctan (x)\) sur \(\Rr\). Parmi les propositions suivantes, cocher celles qui sont vraies :

    \item \(\displaystyle F(x)=\frac{1}{1+x^2}+k\), où \(k\in \Rr\).
    \item \(\displaystyle F(x)=x\arctan (x)+k\), où \(k\in \Rr\).
    \item* \(\displaystyle F(x)=x\arctan (x)-\int \frac{x\mathrm{d}x}{1+x^2}\).
    \item* \(\displaystyle F(x)=x\arctan (x)-\frac{1}{2}\ln (1+x^2)+k\), où \(k\in \Rr\).
\end{multi}


\begin{multi}[multiple,feedback=
{Une intégration par parties avec \(u=\ln (x+1)\) et \(v=x+1\) donne
\[\int \ln (x+1)\, \mathrm{d}x=(x+1)\ln (x+1)-\int \mathrm{d}x=(x+1)\ln (x+1)-x+k,\; k\in \Rr.\]
Une intégration par parties avec \(u=\ln (x+1)\) et \(v=x^2/2\) donne
\[\int x\ln (1+x)\mathrm{d}x=\frac{x^2}{2}\ln (1+x)-\int \frac{x^2\mathrm{d}x}{2(1+x)}.\]
}]{Question}
On se place sur \(]-1,+\infty[\). Parmi les affirmations suivantes, cocher celles qui sont vraies :

    \item \(\displaystyle \int \ln (x+1)\, \mathrm{d}x=\mathrm{e}^{x+1}+k\), où \(k\in \Rr\).
    \item* \(\displaystyle \int \ln (x+1)\, \mathrm{d}x=(x+1)\ln (x+1)-\int \mathrm{d}x\).
    \item \(\displaystyle \int x\ln (1+x)\mathrm{d}x=\int x\mathrm{d}x\times\ln (1+x)=\frac{x^2}{2}\ln (1+x)+k\), où \(k\in \Rr\).
    \item* \(\displaystyle \int x\ln (1+x)\mathrm{d}x=\frac{x^2}{2}\ln (1+x)-\int \frac{x^2\mathrm{d}x}{2(1+x)}\).
\end{multi}


\begin{multi}[multiple,feedback=
{La fonction \(f\) est continue sur \(\Rr\), donc elle y admet des primitives. Avec \(t=\mathrm{e}^{-x}\), on a : \(\mathrm{d}t=-\mathrm{e}^{-x}\, \mathrm{d}x\) et
\[\int \frac{\mathrm{d}x}{\mathrm{e}^x+\mathrm{e}^{-x}}=\int \frac{\mathrm{e}^{-x}\, \mathrm{d}x}{1+\mathrm{e}^{-2x}}=-\int \frac{\mathrm{d}t}{1+t^2}=-\arctan (t)+k=-\arctan \left(\mathrm{e}^{-x}\right)+k,\; k\in \Rr.\]
La condition \(F(0)=0\) implique que \(\displaystyle k=\frac{\pi}{4}\).
}]{Question}
Soit \(\displaystyle f(x)=\frac{1}{\mathrm{e}^x+\mathrm{e}^{-x}}\). Parmi les affirmations suivantes, cocher celles qui sont vraies :

    \item* \(f\) admet une primitive sur \(\Rr\).
    \item La fonction \(F\) telle que \(\displaystyle F(x)=\ln \left(\mathrm{e}^x+\mathrm{e}^{-x}\right)\) est une primitive de \(f\) sur \(\Rr\).
    \item* La fonction \(F\) telle que \(\displaystyle F(x)=\frac{\pi}{4}-\arctan \left(\mathrm{e}^{-x}\right)\) est une primitive de \(f\) sur \(\Rr\).
    \item La primitive de \(f\) sur \(\Rr\) qui s'annule en \(0\) est d\'efinie par \(\displaystyle F(x)=\frac{\pi}{4}-\arctan \left(\mathrm{e}^{x}\right)\).
\end{multi}


\begin{multi}[multiple,feedback=
{D'abord \(\displaystyle \int \frac{\mathrm{d}x}{\sqrt{1-x^2}}=\arcsin (x)+k\), \(k\in \Rr\). Une intégration par parties avec \(u=\sqrt{1-x^2}\) et \(v=x\) donne
\[\int \sqrt{1-x^2}\mathrm{d}x=x\sqrt{1-x^2}-\int \frac{-x^2\mathrm{d}x}{\sqrt{1-x^2}}.\]
Ensuite, en écrivant \(\displaystyle -x^2=(1-x^2)-1\), on obtient : 
\[\int \sqrt{1-x^2}\mathrm{d}x=x\sqrt{1-x^2}-\int \sqrt{1-x^2}\mathrm{d}x+\int \frac{\mathrm{d}x}{\sqrt{1-x^2}}.\]
D'où : \(\displaystyle 2\int \sqrt{1-x^2}\mathrm{d}x=x\sqrt{1-x^2}+\arcsin x+k\).
}]{Question}
Parmi les affirmations suivantes, cocher celles qui sont vraies :

    \item \(\displaystyle \int \frac{\mathrm{d}x}{\sqrt{1-x^2}}=-\arcsin (x)+k\), \(k\in \Rr\).
    \item* \(\displaystyle \int \sqrt{1-x^2}\mathrm{d}x=x\sqrt{1-x^2}+\int \frac{x^2\mathrm{d}x}{\sqrt{1-x^2}}\).
    \item L'égalité \(x^2=(x^2-1)+1\) donne : \(\displaystyle \int \frac{x^2\mathrm{d}x}{\sqrt{1-x^2}}=-\arcsin x-\int \sqrt{1-x^2}\mathrm{d}
x\).
    \item* Une primitive de \(\displaystyle \sqrt{1-x^2}\) sur \(]-1,1[\) est \(\displaystyle \frac{1}{2}x\sqrt{1-x^2}+\frac{1}{2}\arcsin (x)\).
\end{multi}


\begin{multi}[multiple,feedback=
{Avec \(t=\cos x\), on a : \(\mathrm{d}t=-\sin x\mathrm{d}x\) et 
\[\int \frac{\mathrm{d}x}{\sin x}=\int \frac{\sin x\mathrm{d}x}{\sin ^2x}=\int \frac{-\mathrm{d}t}{1-t^2}=\int \frac{\mathrm{d}t}{t^2-1}.\]
Or, \(\displaystyle \frac{2}{t^2-1}=\frac{1}{t-1}-\frac{1}{t+1}\). Donc
\(\displaystyle \int \frac{\mathrm{d}x}{\sin x}=\frac{1}{2}\ln \left(\frac{1-t}{1+t}\right)+k\), \(k\in \Rr\).
}]{Question}
On pose \(t=\cos x\) pour \(x\in ]0,\pi[\). Parmi les assertions suivantes, cocher celles qui sont vraies :

    \item \(\sin x=\sqrt{1-\cos ^2x}\) et \(\displaystyle \int \frac{\mathrm{d}x}{\sin x}=\int \frac{\mathrm{d}t}{\sqrt{1-t^2}}=\arcsin t+k\), \(k\in \Rr\).
    \item* \(\displaystyle \int \frac{\mathrm{d}x}{\sin x}=\int \frac{\mathrm{d}t}{t^2-1}\).
    \item* \(\forall t\in \Rr\setminus \{-1,1\}\), \(\displaystyle \frac{2}{t^2-1}=\frac{1}{t-1}-\frac{1}{t+1}\).
    \item \(\displaystyle \int \frac{\mathrm{d}x}{\sin x}=\ln \left(\frac{1-\cos x}{1+\cos x}\right)+k\), \(k\in \Rr\).
\end{multi}


\begin{multi}[multiple,feedback=
{On a : \(2+4x^2-4x=1+(2x-1)^2\) et avec \(u=2x-1\), on a : \(\mathrm{d}u=2\mathrm{d}x\) et 
\[F(x)=\frac{1}{2}\int \frac{\mathrm{d}u}{1+u^2}=\frac{1}{2}\arctan u+k=\frac{1}{2}\arctan (2x-1)+k,\; k\in \Rr.\]
}]{Question}
On note par \(F\) une primitive sur \(\Rr\) de \(\displaystyle f(x)=\frac{1}{2+4x^2-4x}\). Parmi les assertions suivantes, cocher celles qui sont vraies :

    \item* \(\displaystyle f(x)=\frac{1}{1+(2x-1)^2}\).
    \item \(\displaystyle F(x)=\int \frac{\mathrm{d}u}{1+u^2}\) avec \(u=2x-1\).
    \item \(\displaystyle F(x)=\arctan (2x-1)+k\), \(k\in \Rr\).
    \item* \(\displaystyle F(x)=\frac{1}{2}\arctan (2x-1)+k\),\; \(k\in \Rr\).
\end{multi}


\begin{multi}[multiple,feedback=
{L'égalité \(2x+3=(2x+2)+1\) donne : \(\displaystyle f(x)=\frac{2x+2}{x^2+2x+2}+\frac{1}{1+(x+1)^2}\). Donc, par linéarité : \(F(x)=\ln (x^2+2x+2)+\arctan (x+1)+k\) pour un \(k\in \Rr\).
}]{Question}
On note par \(F\) une primitive sur \(\Rr\) de \(\displaystyle f(x)=\frac{2x+3}{x^2+2x+2}\). Parmi les assertions suivantes, cocher celles qui sont vraies :

    \item* \(\forall x\in \Rr\), \(\displaystyle f(x)=\frac{2x+2}{x^2+2x+2}+\frac{1}{x^2+2x+2}\).
    \item \(\displaystyle F(x)=\ln (x^2+2x+2)+\frac{1}{x^2+2x+2}+k\), \(k\in \Rr\).
    \item \(\displaystyle F(x)=\ln (x^2+2x+2)+\arctan(x^2+2x+2)+k\), \(k\in \Rr\).
    \item* \(\displaystyle F(x)=\ln (x^2+2x+2)+\arctan (x+1)+k\), \(k\in \Rr\).
\end{multi}


\begin{multi}[multiple,feedback=
{Avec \(x=\sin t\), on a : \(\mathrm{d}x=\cos t\, \mathrm{d}t\) et \(\sqrt{1-x^2}=\cos t\) car \(t\in [-\pi/2,\pi/2]\). D'où
\[\int \sqrt{1-x^2}\, \mathrm{d}x=\int \cos ^2t\, \mathrm{d}t=\int \left(\frac{1}{2}+\frac{\cos (2t)}{2}\right)\mathrm{d}t=\frac{t}{2}+\frac{\sin (2t)}{4}+k,\; k\in \Rr.\]
Ainsi
\[\int \sqrt{1-x^2}\, \mathrm{d}x=\frac{t}{2}+\frac{\sin t\cos t}{2}+k=\frac{\arcsin x}{2}+\frac{x\sqrt{1-x^2}}{2}+k,\; k\in \Rr.\]
}]{Question}
Parmi les assertions suivantes, cocher celles qui sont vraies :

    \item Le changement de variable \(x=\sin t\) pour \(t\in [-\pi/2,\pi/2]\) donne :
\[\int \sqrt{1-x^2}\mathrm{d}x=\int \sqrt{1-\sin ^2t}\, \mathrm{d}t=\int \cos t\, \mathrm{d}t.\]
    \item* Le changement de variable \(x=\sin t\) pour \(t\in [-\pi/2,\pi/2]\) donne :
\[\int \sqrt{1-x^2}\, \mathrm{d}x=\int \cos ^2t\, \mathrm{d}t.\]
    \item \(\displaystyle \int \cos ^2t\, \mathrm{d}t=\frac{t}{2}-\frac{\sin (2t)}{4}+k=\frac{t}{2}-\frac{\sin t\cos t}{2}+k\), \(k\in\Rr\).
    \item* Une primitive de \(\displaystyle \sqrt{1-x^2}\) sur \([-1,1]\) est \(\displaystyle \frac{\arcsin x}{2}+\frac{x\sqrt{1-x^2}}{2}\).
\end{multi}


\begin{multi}[multiple,feedback=
{Avec \(t= x^2\), on a : \(\mathrm{d}t=2x\mathrm{d}x\) et 
\(\displaystyle \int \frac{\mathrm{d}x}{x^4-x^2-2}=\frac{1}{2}\int \frac{\mathrm{d}t}{(t-2)(t+1)}\). Or, \(\forall t\in \Rr\setminus \{-1,2\}\), \(\displaystyle \frac{3}{t^2-t-2}=\frac{1}{t-2}-\frac{1}{t+1}\). Donc \(\displaystyle \int \frac{3\, \mathrm{d}t}{t^2-t-2}=\ln\left|\frac{t-2}{t+1}\right|+k\), \(k\in \Rr\), et ensuite : 
\(\displaystyle \int \frac{x\mathrm{d}x}{x^4-x^2-2}=\frac{1}{6}\ln\left|\frac{x^2-2}{x^2+1}\right|+k\), \(k\in \Rr\).
}]{Question}
Pour \(x\in [0,+\infty[\), on pose \(t=x^2\). Parmi les assertions suivantes, cocher celles qui sont vraies :

    \item \(\displaystyle \int \frac{x\mathrm{d}x}{x^4-x^2-2}=\int \frac{\sqrt{t}\mathrm{d}t}{t^2-t-2}\).
    \item* \(\displaystyle \int \frac{x\mathrm{d}x}{x^4-x^2-2}=\frac{1}{2}\int \frac{\mathrm{d}t}{t^2-t-2}\).
    \item* \(\forall t\in \Rr\setminus \{-1,2\}\), \(\displaystyle \frac{3}{t^2-t-2}=\frac{1}{t-2}-\frac{1}{t+1}\) et \(\displaystyle \int \frac{3\, \mathrm{d}t}{t^2-t-2}=\ln\left|\frac{t-2}{t+1}\right|+k\), \(k\in \Rr\).
    \item \(\displaystyle \int \frac{x\mathrm{d}x}{x^4-x^2-2}=\frac{1}{2}\ln\left|\frac{x^2-2}{x^2+1}\right|+k\), \(k\in \Rr\).
\end{multi}


\begin{multi}[multiple,feedback=
{Avec \(t= x^2\), on a : \(\mathrm{d}t=2x\mathrm{d}x\) et 
\[\int \frac{2\mathrm{d}x}{x(x^2+1)^2}=\int \frac{2x\mathrm{d}x}{x^2(x^2+1)^2}=\int \frac{\mathrm{d}t}{t(t+1)^2}.\]
Ensuite, on décompose en éléments simples donne :
\[\frac{1}{t(t+1)^2}=\frac{1}{t}-\frac{1}{t+1}-\frac{1}{(t+1)^2}.\]
}]{Question}
Le changement de variable \(t=x^2\), pour \(x\in ]0,+\infty[\), donne :

    \item \(\displaystyle \int \frac{2\mathrm{d}x}{x(x^2+1)^2}=\int \frac{\mathrm{d}t}{(t+1)^2}\).
    \item* \(\displaystyle \int \frac{2\mathrm{d}x}{x(x^2+1)^2}=\int \frac{\mathrm{d}t}{t(t+1)^2}\).
    \item \(\displaystyle \int \frac{2\mathrm{d}x}{x(x^2+1)^2}=\int \frac{\mathrm{d}t}{\sqrt{t}(t+1)^2}\).
    \item* \(\displaystyle \int \frac{2\mathrm{d}x}{x(x^2+1)^2}=\int \frac{\mathrm{d}t}{t}-\int \frac{\mathrm{d}t}{t+1}-\int \frac{\mathrm{d}t}{(t+1)^2}\).
\end{multi}


\begin{multi}[multiple,feedback=
{Avec \(t=\tan x\), on aura \(\displaystyle \frac{1}{1+\sin ^2x}=\frac{1+t^2}{1+2t^2}\) et \(\displaystyle \mathrm{d}x=\frac{\mathrm{d}t}{1+t^2}\). D'où
\[\int \frac{\mathrm{d}x}{1+\sin ^2x}=\int \frac{\mathrm{d}t}{1+2t^2}=\frac{1}{\sqrt{2}}\arctan(\sqrt{2}t)+k,\; k\in \Rr.\]
}]{Question}
Pour \(x\in ]-\pi/2,\pi/2[\), on pose \(t=\tan (x)\) et on rappelle que \(\displaystyle \sin ^2x=\frac{\tan ^2x}{1+\tan ^2x}\). Parmi les affirmations suivantes, cocher celles qui sont vraies :

    \item \(\displaystyle \frac{1}{1+\sin ^2x}=\frac{1+t^2}{1+2t^2}\) et \(\displaystyle \int \frac{\mathrm{d}x}{1+\sin ^2x}=\int \frac{1+t^2}{1+2t^2}\mathrm{d}t\).
    \item \(\displaystyle \int \frac{\mathrm{d}x}{1+\sin ^2x}=\frac{t}{2}-\frac{1}{2\sqrt{2}}\arctan(\sqrt{2}t)+k\), \(k\in \Rr\).
    \item* \(\displaystyle \int \frac{\mathrm{d}x}{1+\sin ^2x}=\int \frac{1}{1+2t^2}\mathrm{d}t\).
    \item* Sur \(]-\pi/2,\pi/2[\), on a : \(\displaystyle \int \frac{\mathrm{d}x}{1+\sin ^2x}=\frac{1}{\sqrt{2}}\arctan\left(\sqrt{2}\tan x\right)+k\), \(k\in \Rr\).
\end{multi}


\begin{multi}[multiple,feedback=
{Avec \(t=\cos x\), on aura \(\displaystyle \frac{\tan x}{1+\cos ^2x}=\frac{\sin x}{\cos x(1+\cos ^2x)}\) et \(\displaystyle \mathrm{d}t=-\sin x\mathrm{d}x\). D'où
\[\int \frac{\tan x\mathrm{d}x}{1+\cos ^2x}=-\int \frac{\mathrm{d}t}{t(1+t^2)}=\frac{1}{2}\ln (1+t^2)-\ln |t|+k,\; k\in \Rr.\]
}]{Question}
Pour \(x\in ]-\pi/2,\pi/2[\), on pose \(t=\cos x\). Parmi les assertions suivantes, cocher celles qui sont vraies :

    \item \(\displaystyle \int \frac{\tan x\mathrm{d}x}{1+\cos ^2x}=\tan x\int \frac{\mathrm{d}t}{1+t^2}=\tan x.\arctan (\cos x)+k\), \(k\in \Rr\).
    \item* \(\displaystyle \int \frac{\tan x\mathrm{d}x}{1+\cos ^2x}=-\int \frac{\mathrm{d}t}{t(1+t^2)}\).
    \item \(\forall t\in \Rr^*\), \(\displaystyle \frac{1}{t(1+t^2)}=\frac{t}{1+t^2}-\frac{1}{t}\) et donc \(\displaystyle \int \frac{\mathrm{d}t}{t(1+t^2)}=\ln \left(\frac{\sqrt{1+t^2}}{t}\right)+k\), \(k\in \Rr\).
    \item* Sur \(]-\pi/2,\pi/2[\), on a : \(\displaystyle \int \frac{\tan x\, \mathrm{d}x}{1+\cos ^2x}=\ln \left(\frac{\sqrt{1+\cos ^2x}}{\cos x}\right)+k\), \(k\in \Rr\).
\end{multi}


\begin{multi}[multiple,feedback=
{D'abord, \(\displaystyle \frac{\cos ^3x}{1+\sin ^2x}=\frac{\cos x(1-\sin ^2x)}{1+\sin ^2x}\). Ensuite \(t=\sin x\Rightarrow\mathrm{d}t=\cos x\mathrm{d}x\) et
\[\int \frac{\cos ^3x\mathrm{d}x}{1+\sin ^2x}=\int \frac{1-t^2}{1+t^2}\mathrm{d}t=\int \left(-1+\frac{2}{1+t^2}\right)\mathrm{d}t=-t+2\arctan t+k,\; k\in \Rr.\]
}]{Question}
Le changement de variable \(t=\sin x\) donne :

    \item \(\displaystyle \int \frac{\cos ^3x\mathrm{d}x}{1+\sin ^2x}=\cos ^3 x\int \frac{\mathrm{d}t}{1+t^2}\).
    \item \(\displaystyle \int \frac{\cos ^3x\mathrm{d}x}{1+\sin ^2x}=\cos ^3 x.\arctan (\sin x)+k\), \(k\in \Rr\).
    \item* \(\displaystyle \int \frac{\cos ^3x\mathrm{d}x}{1+\sin ^2x}=\int \frac{1-t^2}{1+t^2}\mathrm{d}t\).
    \item* \(\displaystyle \int \frac{\cos ^3x\mathrm{d}x}{1+\sin ^2x}=-t+2\arctan t+k\), \(k\in \Rr\).
\end{multi}


\begin{multi}[multiple,feedback=
{Avec \(t=\tan (x/2)\), on a : \(\displaystyle \mathrm{d}x=\frac{2\mathrm{d}t}{1+t^2}\),
\[\frac{1}{2+\cos x}=\frac{1+t^2}{3+t^2}\quad \mbox{et}\quad \frac{1}{2+\sin x}=\frac{1+t^2}{2(t^2+t+1)}.\]
D'où
\[\int \frac{\mathrm{d}x}{2+\cos x}=\int \frac{2\mathrm{d}t}{3+t^2}=\frac{2}{\sqrt{3}}\arctan\left(\frac{t}{\sqrt{3}}\right)+k,\; k\in \Rr.\]
et
\[\int \frac{\mathrm{d}x}{2+\sin x}=\int \frac{\mathrm{d}t}{t^2+t+1}=\frac{2}{\sqrt{3}}\arctan\left(\frac{2t+1}{\sqrt{3}}\right)+k,\; k\in \Rr.\]
}]{Question}
On se place dans l'intervalle \(]-\pi,\pi[\) et on rappelle que \(\displaystyle \cos x=\frac{1-\tan ^2(x/2)}{1+\tan ^2(x/2)}\) et que \(\displaystyle \sin x=\frac{2\tan (x/2)}{1+\tan ^2(x/2)}\). Le changement de variable \(t=\tan (x/2)\) donne :

    \item* \(\displaystyle \int \frac{\mathrm{d}x}{2+\cos x}=\int \frac{2\mathrm{d}t}{3+t^2}\).
    \item* \(\displaystyle \int \frac{\mathrm{d}x}{2+\sin x}=\int \frac{\mathrm{d}t}{t^2+t+1}\).
    \item \(\displaystyle \int \frac{\mathrm{d}x}{2+\cos x}=\frac{1}{\sqrt{3}}\arctan\left(\frac{t}{\sqrt{3}}\right)+k\), \(k\in \Rr\).
    \item \(\displaystyle \int \frac{\mathrm{d}x}{2+\sin x}=\frac{1}{\sqrt{2t+1}}\ln (t^2+t+1)+k\), \(k\in \Rr\).
\end{multi}


\begin{multi}[multiple,feedback=
{Avec \(\displaystyle t=\sqrt{x}\), on a : \(\displaystyle x=t^2\Rightarrow \mathrm{d}x=2t\, \mathrm{d}t\) et \(\displaystyle \int \frac{\mathrm{d}x}{\sqrt{x}(x+1)^2}=\int \frac{2\, \mathrm{d}t}{(t^2+1)^2}\).
Une intégration par parties avec \(\displaystyle u=\frac{1}{t^2+1}\) et \(v=t\) donne
\[\int \frac{\mathrm{d}t}{t^2+1}=\frac{t}{t^2+1}+\int \frac{2t^2\, \mathrm{d}t}{(t^2+1)^2}=\frac{t}{t^2+1}+\int \frac{2\, \mathrm{d}t}{t^2+1}-\int \frac{2\, \mathrm{d}t}{(t^2+1)^2}.\]
D'où \(\displaystyle \int \frac{2\, \mathrm{d}t}{(t^2+1)^2}=\frac{t}{t^2+1}+\arctan t+k\Rightarrow \int \frac{\mathrm{d}x}{\sqrt{x}(x+1)^2}=\frac{\sqrt{x}}{x+1}+\arctan \sqrt{x}+k\), \(k\in \Rr\).
}]{Question}
Parmi les affirmations suivantes, cocher celles qui sont vraies :

    \item Le changement de variable \(\displaystyle t=\sqrt{x}\) donne \(\displaystyle \int \frac{\mathrm{d}x}{\sqrt{x}(x+1)^2}=\int \frac{\mathrm{d}t}{(t^2+1)^2}\).
    \item* \(\displaystyle \int \frac{\mathrm{d}t}{t^2+1}=\frac{t}{t^2+1}+\int \frac{2t^2\, \mathrm{d}t}{(t^2+1)^2}\).
    \item* \(\displaystyle \int \frac{2\, \mathrm{d}t}{(t^2+1)^2}=\frac{t}{t^2+1}+\arctan t+k\), \(k\in \Rr\).
    \item Une primitive de \(\displaystyle \frac{1}{\sqrt{x}(x+1)^2}\) sur \(]0,+\infty[\) est \(\displaystyle \frac{1}{2}\left(\frac{\sqrt{x}}{x+1}+\arctan \sqrt{x}\right)\).
\end{multi}


\begin{multi}[multiple,feedback=
{Avec \(\displaystyle t=\sqrt{x}\), on a : \(\displaystyle x=t^2\Rightarrow \mathrm{d}x=2t\, \mathrm{d}t\) et \(\displaystyle \int \frac{\mathrm{d}x}{\sqrt{x}(x-1)^2}=\int \frac{2\, \mathrm{d}t}{(t^2-1)^2}\). La décomposition en éléments simples donne : \(\displaystyle \frac{4}{(t^2-1)^2}=\frac{1}{t+1}-\frac{1}{t-1}+\frac{1}{(t+1)^2}+\frac{1}{(t-1)^2}\). Donc
\[\int \frac{4\mathrm{d}t}{(t^2-1)^2}=\ln \left|\frac{t+1}{t-1}\right|-\frac{2t}{t^2-1}+k,\; k\in \Rr.\]
D'où
\[\int \frac{\mathrm{d}x}{\sqrt{x}(x-1)^2}=\frac{1}{2}\left(\ln \left|\frac{t+1}{t-1}\right|-\frac{2t}{t^2-1}\right)+k=\frac{1}{2}\left(\ln \left|\frac{\sqrt{x}+1}{\sqrt{x}-1}\right|-\frac{2\sqrt{x}}{x-1}\right)+k,\; k\in \Rr.\]
}]{Question}
Parmi les affirmations suivantes, cocher celles qui sont vraies :

    \item Le changement de variable \(\displaystyle t=\sqrt{x}\) donne \(\displaystyle \int \frac{\mathrm{d}x}{\sqrt{x}(x-1)^2}=\int \frac{\mathrm{d}t}{(t^2-1)^2}\).
    \item* \(\forall t\in \Rr\setminus \{-1,1\}\), \(\displaystyle \frac{4}{(t^2-1)^2}=\frac{1}{t+1}-\frac{1}{t-1}+\frac{1}{(t+1)^2}+\frac{1}{(t-1)^2}\).
    \item* \(\displaystyle \int \frac{4\mathrm{d}t}{(t^2-1)^2}=\ln \left|\frac{t+1}{t-1}\right|-\frac{2t}{t^2-1}+k\), \(k\in \Rr\).
    \item Une primitive de \(\displaystyle \frac{1}{\sqrt{x}(x-1)^2}\) sur \(]1,+\infty[\) est \(\displaystyle \ln \left|\frac{\sqrt{x}+1}{\sqrt{x}-1}\right|-\frac{2\sqrt{x}}{x-1}\).
\end{multi}


\begin{multi}[multiple,feedback=
{Il suffit de dériver : \(\displaystyle \left[C\arcsin \left(x\sqrt{a/b}\right)\right]'=\frac{C\sqrt{a}}{\sqrt{b-ax^2}}\neq f(x)\) pour tout \(C\in \Rr\). Par contre,
\[\left(C\sqrt{ax^2+b}\right)'=\frac{Cax}{\sqrt{ax^2+b}}=xf(x)\quad \mbox{si}\quad C=a^{-1}.\]
Si \(a\) est strictement négatif et \(b\) est strictement positif,
\[\left[C\arcsin \left(x\sqrt{-a/b}\right)\right]'=\frac{C\sqrt{-a}}{\sqrt{b+ax^2}}=f(x)\quad \mbox{si}\quad C\sqrt{-a}=1.\]
Par contre, si \(a\) est strictement positif et \(b\) est strictement négatif alors, pour tout \(C\in \Rr\), \(\left[C\arcsin \left(x\sqrt{a/(-b)}\right)\right]'\neq f(x)\).
}]{Question}
Soient \(a\) et \(b\) deux réels et \(f\) la fonction telle que \(\displaystyle f(x)=\frac{1}{\sqrt{ax^2+b}}\). Parmi les affirmations suivantes, cocher celles qui sont vraies :

    \item Si \(a\) et \(b\) sont strictement positifs, il existe une constante \(C\) telle que \(F(x)=C\arcsin \left(x\sqrt{a/b}\right)\) soit une primitive de \(f(x)\).
    \item* Si \(a\) et \(b\) sont strictement positifs, il existe une constante \(C\) telle que \(F(x)=C\sqrt{ax^2+b}\) soit une primitive de \(xf(x)\).
    \item* Si \(a\) est strictement négatif et \(b\) est strictement positif, il existe une constante \(C\) telle que \(F(x)=C\arcsin \left(x\sqrt{-a/b}\right)\) soit une primitive de \(f(x)\).
    \item Si \(a\) est strictement positif et \(b\) est strictement négatif, il existe une constante \(C\) telle que \(F(x)=C\arcsin \left[x\sqrt{a/(-b)}\right]\) soit une primitive de \(f(x)\).
\end{multi}
