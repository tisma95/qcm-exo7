
\qcmtitle{Courbes paramétrées}
\qcmauthor{Abdellah Hanani, Mohamed Mzari}

%%%%%%%%%%%%%%%%%%%%%%%%%%%%%%%%%%%%%%%%%%%%%
\section{Courbes paramétrées}
\subsection{Courbes paramétrées | Niveau 1}

\begin{question}
\qtags{motcle=tangentes, points doubles}
La trajectoire $\Gamma$ d'une particule en mouvement est donnée par les équations
$$x=t-1\quad \mbox{et} \quad y=t+1.$$
\begin{answers}  
\good{$\Gamma$ est la droite d'équation $y=2+x$.}
\bad{La tangente à $\Gamma$ au point $(x(0),y(0))$ est la droite d'équation $y=x$.}
\good{La tangente à $\Gamma$ au point $(x(1),y(1))$ est la droite d'équation $y=2+x$.}
\bad{$\Gamma$ possède un point double.}
\end{answers}
\begin{explanations}
En éliminant le paramètre $t$, on obtient $y=2+x$. $\Gamma$ est la droite d'équation $y=2+x$ et elle est confondue avec sa tangente en n'importe quel point.
\end{explanations}
\end{question}

\begin{question}
\qtags{motcle=tangentes, points doubles}
La trajectoire $\Gamma$ d'une particule en mouvement est donnée par les équations
$$x=1+\cos (2t)\quad \mbox{et} \quad y=1+\sin (2t)\qquad 0\leq t\leq \pi.$$
\begin{answers}  
\bad{$\Gamma$ est le cercle de centre $(1,1)$ et de rayon $2$.}
\good{$\Gamma$ possède un point double.}
\good{La tangente à $\Gamma$ au point $(x(0),y(0))$ est la droite d'équation $x=2$.}
\bad{La tangente à $\Gamma$ au point $(x(\pi),y(\pi))$ est la droite d'équation $y=x$.}
\end{answers}
\begin{explanations}
En éliminant le paramètre $t$, on obtient : $(x-1)^2+(y-1)^2=1$. Donc $\Gamma$ est le cercle de centre $(1,1)$ et de rayon $1$. Les points de paramètres $0$ et $\pi$ sont confondus. Enfin, $x'(0)=0$ et $y'(0)\neq 0$, donc la tangente au point $(x(0),y(0))$ est la droite d'équation $x=x(0)$.
\end{explanations}
\end{question}

\subsection{Courbes paramétrées | Niveau 2}

\begin{question}
\qtags{motcle=tangentes}
Un avion en papier a effectué un vol suivant la trajectoire $\Gamma$ donnée par
$$x=t-2\sin t\quad \mbox{et} \quad y=4-3\cos t.$$
\begin{answers}  
\bad{A l'instant $t=\pi$, l'avion volait en position verticale.}
\good{A l'instant $t=\pi$, l'avion volait en position horizontale.}
\bad{A l'instant $t=\pi/2$, l'avion volait suivant la droite d'équation $y=3x+10$.}
\good{A l'instant $t=\pi/3$, l'avion volait en position verticale.}
\end{answers}
\begin{explanations}
On a : $y'(\pi)=0$ et $x'(\pi)\neq 0$. Donc la tangente au point de paramètre $\pi$ est horizontale. A l'instant $t=\pi/2$, l'avion volait suivant la tangente \`a $\Gamma$ à cet instant. C'est-à-dire suivant la droite d'équation $y=3x+10-3\pi /2$. Enfin, $x'(\pi/3)=0$ et $y'(\pi/3)\neq 0$. Donc la tangente au point de paramètre $\pi/3$ est verticale.
\end{explanations}
\end{question}

\begin{question}
\qtags{motcle=tangentes, points stationnaires}
La trajectoire $\Gamma$ d'une particule en mouvement est donnée par les équations
$$x=t^2+t^3\quad \mbox{et} \quad  y=2t^2-t^3.$$
\begin{answers}  
\bad{Le point de paramètre $0$ est un point de rebroussement de seconde espèce.}
\good{La tangente à $\Gamma$ au point $(x(0),y(0))$ est la droite d'équation $y=2x$.}
\good{La tangente à $\Gamma$ au point $(x(1),y(1))$ est dirigée par le vecteur $(5,1)$.}
\bad{La tangente à $\Gamma$ au point $(x(1),y(1))$ est la droite d'équation $5x-y+3=0$.}
\end{answers}
\begin{explanations}
Le $DL_3(0)$ de $f(t)=(x(t),y(t))$ est 
$$f(t)=t^2(1,2)+t^3(1,-1)+t^3\left(\varepsilon_1(t),\varepsilon_2(t)\right)\mbox{ avec }\lim _{t\to 0}\varepsilon_i(t)=0.$$
Donc le point de paramètre $0$ est un point de rebroussement de première espèce et la tangente en ce point est la droite d'équation $y=2x$. La tangente à $\Gamma$ au point de paramètre $1$ est dirigée par $f'(1)=(5,1)$.
\end{explanations}
\end{question}

\begin{question}
\qtags{motcle=points stationnaires, branches infinies}
La trajectoire $\Gamma$ d'une particule en mouvement est donnée par les équations
$$x=2t^2-t^4\quad \mbox{et} \quad  y=t^2+t^4.$$
\begin{answers}  
\good{Le point de paramètre $0$ est un point de rebroussement de seconde espèce.}
\bad{La tangente à $\Gamma$ au point $(x(0),y(0))$ est la droite d'équation $y=2x$.}
\bad{$\Gamma$ admet la droite d'équation $y=-x$ comme asymptote quand $t$ tend vers l'infini.}
\good{$\Gamma$ admet une branche parabolique de direction asymptotique la droite d'équation $y=-x$.}
\end{answers}
\begin{explanations}
Le $DL_4(0)$ de $f(t)=(x(t),y(t))$ est 
$$f(t)=t^2(2,1)+t^4(-1,1)+t^4\left(\varepsilon_1(t),\varepsilon_2(t)\right)\mbox{ avec }\lim _{t\to 0}\varepsilon_i(t)=0.$$
Donc le point de paramètre $0$ est un point de rebroussement de seconde espèce et la tangente en ce point est la droite d'équation $2y-x=0$. Enfin $\displaystyle \lim _{t\to \pm \infty}[y(t)+x(t)]=+\infty$, donc $\Gamma$ admet une branche parabolique de direction asymptotique la droite d'équation $y=-x$.
\end{explanations}
\end{question}

\subsection{Courbes paramétrées | Niveau 3}

\begin{question}
\qtags{motcle=points stationnaires, branches infinies}
La trajectoire $\Gamma$ d'une particule en mouvement est donnée par les équations
$$x=\frac{1}{2}(t^2-2t)\quad \mbox{et} \quad  y=\frac{1}{3}t^{3}-\frac{1}{2}t^2.$$
\begin{answers}  
\bad{Le point de paramètre $1$ est un point de rebroussement de seconde espèce.}
\good{La tangente à $\Gamma$ au point $(x(1),y(1))$ est la droite d'équation $y=x$.}
\bad{Le point de paramètre $0$ est un point stationnaire.}
\good{$\Gamma$ admet une branche parabolique de direction asymptotique l'axe des $y$.}
\end{answers}
\begin{explanations}
Avec $f(t)=(x(t),(y(t))$. On a : $f'(1)=(0,0)$, $f''(1)=(1,1)$ et $f^{(3)}(1)=(0,2)$. Le point de paramètre $1$ est un point de rebroussement de première espèce et la tangente en ce point est la droite d'équation $x-y=0$. Enfin 
$$\displaystyle \lim _{t\to \pm \infty}x(t)
=+\infty,\qquad \lim _{t\to \pm \infty}y(t)=\pm\infty\quad\mbox{et}\quad \lim _{t\to \pm \infty}\frac{y(t)}{x(t)}=\pm\infty .$$
Donc $\Gamma$ admet une branche parabolique de direction asymptotique l'axe des ordonnées.
\end{explanations}
\end{question}

\begin{question}
\qtags{motcle=points stationnaires, tangentes}
La trajectoire $\Gamma$ d'une particule en mouvement est donnée par les équations
$$x=2+\cos t\quad \mbox{et} \quad y=\frac{t^2}{2}+\sin t.$$
\begin{answers}  
\good{$\Gamma$ n'admet pas de point stationnaire.}
\good{Le point de paramètre $t=\pi/2$ est un point d'inflexion.}
\bad{La tangente à $\Gamma$ au point de paramètre $t=\pi/2$ est verticale.}
\bad{$\Gamma$ est symétrique par rapport à l'axe des $x$.}
\end{answers}
\begin{explanations}
Notons $f(t)=(x(t),y(t))$. Le système $f'(t)=(0,0)$ n'admet pas de solution, donc $\Gamma$ n'admet pas de point stationnaire. $\displaystyle \lim _{t\to \pm\infty}x(t)$ n'existe pas. Un DL de f en $\pi/2$ montre que le point de paramètre $t=\pi/2$ est un point d'inflexion et la tangente en ce point est dirigée par le vecteur $f'(\pi/2)=(-1,\pi/2)$. Enfin $y$ n'est ni paire ni impaire, donc $\Gamma$ n'est pas symétrique par rapport à l'axe des $x$.
\end{explanations}
\end{question}

\begin{question}
\qtags{motcle=points stationnaires, tangentes}
La trajectoire $\Gamma$ d'une particule en mouvement est donnée par les équations
$$x=1+\cos t\quad \mbox{et} \quad y=t-\sin t.$$
\begin{answers}  
\bad{$\Gamma$ est symétrique par rapport à l'axe des $y$.}
\bad{$\Gamma$ admet un point double.}
\good{Le point de paramètre $t=0$ est un point de rebroussement de première espèce.}
\good{La tangente à $\Gamma$ au point de paramètre $t=0$ est horizontale.}
\end{answers}
\begin{explanations}
D'abord, $x$ est paire et $y$ est impaire, donc $\Gamma$ est symétrique par rapport à l'axe des $x$. Ensuite, en notant $f(t)=(x(t),y(t))$, la résolution du système $f(t_1)=f(t_2)$ donne $t_1=t_2$, donc $\Gamma$ ne possède pas de point double. Enfin, on a :
$$f(t)=(2,0)+t^2(-1/2,0)+t^3(0,1/6)+t^3\left(\varepsilon_1(t),\varepsilon_2(t)\right)\mbox{ avec }\lim _{t\to 0}\varepsilon_i(t)=0.$$
Donc le point de paramètre $t=0$ est un point de rebroussement de première espèce et la tangente en ce point est horizontale.
\end{explanations}
\end{question}

\begin{question}
\qtags{motcle=tangentes, asymptotes}
La trajectoire $\Gamma$ d'une particule en mouvement est donnée par les équations
$$x(t)=\frac{1}{1-t^2}\quad \mbox{et}\quad y(t)=\frac{1}{1+t^4}.$$
\begin{answers}  
\bad{$\Gamma$ est symétrique par rapport à l'origine du repère.}
\good{$\Gamma$ admet la droite d'équation $y=1/2$ comme asymptote quand $t$ tend vers $1$.}
\good{Le point de paramètre $t=0$ est un point de rebroussement de seconde espèce.}
\bad{La tangente à $\Gamma$ au point de paramètre $t=0$ est verticale.}
\end{answers}
\begin{explanations}
$y$ étant strictement positive, $\Gamma$ ne peut être symétrique par rapport à l'origine du repère. Ensuite, $\displaystyle \lim_{t\to 1^{\pm}}x(t)=\pm \infty$ et $\displaystyle \lim_{t\to 1}y(t)=1/2$, donc la droite d'équation $y=1/2$ est une asymptote. Le $DL_4(0)$ de $f(t)=(x(t),y(t))$ est 
$$f(t)=(1,1)+t^2(1,0)+t^4(1,-1)+t^4\left(\varepsilon_1(t),\varepsilon_2(t)\right)\mbox{ avec }\lim _{t\to 0}\varepsilon_i(t)=0.$$
On en déduit que le point de paramètre $t=0$ est un point de rebroussement de seconde espèce et que la tangente en ce point est horizontale.
\end{explanations}
\end{question}

\begin{question}
\qtags{motcle=points stationnaires, asymptotes}
La trajectoire $\Gamma$ d'une particule en mouvement est donnée par les équations
$$x=\frac{t^2}{1-t^2}\quad \mbox{et} \quad  y=\frac{t^2}{1+t}.$$
\begin{answers}  
\bad{Le point de paramètre $0$ est un point de rebroussement de seconde espèce.}
\good{La tangente à $\Gamma$ au point de paramètre $1$ est la droite d'équation $y=x$.}
\bad{$\Gamma$ admet la droite d'équation $y=2$ comme asymptote quand $t$ tend vers $1$.}
\good{$\Gamma$ admet la droite d'équation $y=2x-1/2$ comme asymptote quand $t$ tend vers $-1$.}
\end{answers}
\begin{explanations}
On a : $f(t)=t^2(1,1)+t^3(0,-1)+t^3\left(\varepsilon_1(t),\varepsilon_2(t)\right)$ avec $\displaystyle \lim _{t\to 0}\varepsilon_i(t)=0$. Donc le point de paramètre $0$ est un point de rebroussement de première espèce et la tangente en ce point est la droite d'équation $y=x$. Enfin, l'étude des branches infinies, quand $t$ tend vers $1$ et quand $t$ tend vers $-1$, montre que les droites d'équation $y=1/2$ et d'équation $y=2x-1/2$ sont des asymptotes.
\end{explanations}
\end{question}

\subsection{Courbes paramétrées | Niveau 4}

\begin{question}
\qtags{motcle=points stationnaires, asymptotes, points doubles}
La trajectoire $\Gamma$ d'une particule en mouvement est donnée par les équations
$$x=\frac{t}{4-t^2}\quad \mbox{et} \quad  y=\frac{t^2}{2-t}.$$
\begin{answers}  
\bad{$\Gamma$ admet un seul point stationnaire.}
\bad{La droite d'équation $x=1$ est une asymptote quand $t$ tend vers $-2$.}
\good{La droite d'équation $y=8x-3$ est une asymptote quand $t$ tend vers $2$.}
\good{Le point de coordonnées $(1/2,2)$ est un point double.}
\end{answers}
\begin{explanations}
D'abord, $\Gamma$ n'a pas de point stationnaire car $x'(t)\neq 0$. Puis, $\displaystyle \lim_{t\to -2^{\pm}}x(t)=\pm \infty$ et $\displaystyle \lim_{t\to -2}y(t)=1$, donc la droite d'équation $y=1$ est une asymptote quand $t$ tend vers $-2$. De même, $\displaystyle \lim_{t\to 2^{\pm}}x(t)=\lim_{t\to 2^{\pm}}y(t)=\pm \infty$, $\displaystyle \lim_{t\to 2^{\pm}}\frac{y(t)}{x(t)}=8$ et $\displaystyle \lim_{t\to 2^{\pm}}[y(t)-x(t)]=-3$. Donc la droite d'équation $y=8x-3$ est une asymptote quand $t$ tend vers $2$. Enfin, la résolution de $(x(t_1),y(t_1))=(x(t_2),y(t_2))$ montre que $(1/2,2)$ est un point double obtenu avec les paramètres $-1+\sqrt{5}$ et $-1-\sqrt{5}$.
\end{explanations}
\end{question}

\begin{question}
\qtags{motcle=points stationnaires, asymptotes}
La trajectoire $\Gamma$ d'une particule en mouvement est donnée par les équations
$$x=\frac{t^2}{1-t}\quad \mbox{et} \quad y=\frac{3t-1}{1-t^2}.$$
\begin{answers}  
\bad{$\Gamma$ admet un seul point stationnaire.}    
\bad{La droite d'équation $\displaystyle y=1/2$ est une asymptote quand $t$ tend vers $-1$.}
\good{La droite d'équation $\displaystyle y=x+1$ est une asymptote quand $t$ tend vers $1$.}
\good{Le point de paramètre $0$ est un méplat.}
\end{answers}
\begin{explanations}
D'abord, $\Gamma$ n'a pas de point stationnaire car $y'(t)\neq 0$. Puis, $\displaystyle \lim_{t\to -1}x(t)=1/2$ et $\displaystyle \lim_{t\to -1^{\pm}}y(t)=\pm\infty$, donc la droite d'équation $x=1/2$ est une asymptote quand $t$ tend vers $-1$. De même, $\displaystyle \lim_{t\to 1^{\pm}}x(t)=\lim_{t\to 1^{\pm}}y(t)=\pm \infty$, $\displaystyle \lim_{t\to 1^{\pm}}\frac{y(t)}{x(t)}=1$ et $\displaystyle \lim_{t\to 2^{\pm}}[y(t)-x(t)]=1$. Donc la droite d'équation $\displaystyle y=x+1$ est une asymptote quand $t$ tend vers $1$. Enfin, le $DL_2(0)$ de $f(t)=(x(t),y(t))$ est 
$$f(t)=(0,-1)+t(0,3)+t^2(1,-1)+t^2\left(\varepsilon_1(t),\varepsilon_2(t)\right)\mbox{ avec }\lim _{t\to 0}\varepsilon_i(t)=0.$$
Donc le point de paramètre $0$ est un point ordinaire.
\end{explanations}
\end{question}

