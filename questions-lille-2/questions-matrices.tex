
\qcmtitle{Calcul matriciel}
\qcmauthor{Abdellah Hanani, Mohamed Mzari}

%%%%%%%%%%%%%%%%%%%%%%%%%%%%%%%%%%%%%%%%%%%%%
\section{Calcul matriciel}
\subsection{Calcul matriciel | Niveau 1}

\begin{question}
\qtags{motcle=Matrice/Somme/Produit}
Soit $A$ et $B$ deux matrices. Quelles sont les assertions vraies ?
\begin{answers}  
\good{Si la matrice $A+B$ est définie, alors $B+A$ est définie}.
\bad{Si la matrice $A+B$ est définie, alors $AB$ est définie}. 
\bad{Si la matrice $AB$ est définie, alors $BA$ est définie}.
\good{Si la matrice $A+B$ est définie, alors $A^tB$ est définie, où $^tB$ est la transposée de la matrice $B$}.
\end{answers}
\begin{explanations} La somme de deux matrices est définie si les deux matrices admettent la même taille. Dans ce cas, on a $A+B=B+A$. Le produit $AB$ de deux matrices est défini si le nombre de colonnes de $A$ est égal au nombre de lignes de $B$. Le produit n'est pas commutatif.
\end{explanations}
\end{question}


\begin{question}
\qtags{motcle=Matrice/Somme/Produit}
On considère les matrices : 
$$A=  
\left(\begin{array}{rc}
1&2\\
3&4\end{array}\right),\; B=  
\left(\begin{array}{rc}
1&1\\
1&-1\end{array}\right),\; C=  
\left(\begin{array}{rc}
1&3\\
5&9\end{array}\right),\; D=  
\left(\begin{array}{rc}
3&-1\\
7&-1\end{array}\right)\mbox{ et }E=  
\left(\begin{array}{rc}
4&6\\
-2&2\end{array}\right).$$ 
Quelles sont les assertions vraies ?
\begin{answers}  
\good{$2A-B=C$}.
\good{$AB=D$}.
\bad{$BA=E$}.
\bad{$AB=BA$}.
\end{answers}
\begin{explanations}  On a : $2A-B=C$, $AB=D$ et $BA =  \left(\begin{array}{rc}4&6\\-2&-2 \end{array}\right)$.
\end{explanations}
\end{question}

\begin{question}
\qtags{motcle=Matrice/Somme/Produit}
On considère les matrices : 
$$A=\left(\begin{array}{rcc}
1&1&2\end{array}\right),\; B=  
\left(\begin{array}{rc}1\\-1\\ 1
\end{array}\right),\; C=  
\left(\begin{array}{rcc}
1&3&1\\ 1&1&1
\end{array}\right),\; D= \left(\begin{array}{rc}
0&1\\ 1&-1\\ 2&1 \end{array}\right)\mbox{ et }E=\left(\begin{array}{rc}5&-1\\3&1\end{array}\right).$$
Quelles sont les assertions vraies ?
\begin{answers}  
\bad{$A+B=B$}.
\good{$AB=\left(\begin{array}{rc}2\\ \end{array}\right)$}.
\bad{$CA=\left(\begin{array}{rc}6\\ 2\\\end{array}\right)$}.
\good{$CD=E$}.
\end{answers}
\begin{explanations} Les opérations $A+B$ et $CA$ ne sont pas définies.
\end{explanations}
\end{question}
    
\begin{question}
\qtags{motcle=Matrice/Base/Dimension}
On considère $M_{n,m} (\Rr)$ l'ensemble des matrices à $n$ lignes et $m$ colonnes, à  coefficients dans $\Rr$, muni de l'addition usuelle et la multiplication par un scalaire. Quelles sont les assertions vraies ?
\begin{answers}  
\good{$M_{n,m} (\Rr)$ est un espace vectoriel}.
\good{$\dim M_{n,m} (\Rr) = mn $}.
\bad{$\dim M_{n,m} (\Rr) = m+n $}.
\bad{$M_{n,m} (\Rr)$ est un espace vectoriel de dimension infinie}.
\end{answers}
\begin{explanations} On vérifie que $M_{n,m} (\Rr)$, muni des opérations usuelles est un $\Rr$- espace vectoriel.
\vskip0mm
Pour $1\le i\le n$ et $1\le j\le m$, on note $D_{i,j}$ la matrice dont le coefficient située à la ième ligne et jième colonne est $1$ et les autres coefficients sont nuls. Alors, $\{D_{i,j}\; ; \; 1\le i\le n, 1\le j\le m\}$ est une base de $M_{n,m} (\Rr)$. Par conséquent, $\dim M_{n,m} (\Rr) = mn $.
\end{explanations}
\end{question}
 
\subsection{Calcul matriciel | Niveau 2}

\begin{question}
\qtags{motcle=Matrice/Somme/Produit}
On considère les matrices : 
$$A=  
\left(\begin{array}{rcc}
1&1&2\\
-1&0&2\\
1&-1&1\\
\end{array}\right) \quad \mbox{et} \quad 
B=    
\left(\begin{array}{rcc}
1&1&-1\\
1&1&3\\
-1&1&0\end{array}\right). $$
Quelles sont les assertions vraies ?
\begin{answers}  
\good{$2A+3B=  
\left(\begin{array}{rcc}
5&5&1\\
1&3&13\\
-1&1&2\end{array}\right).$}
\bad{$A-B= 
\left(\begin{array}{rcc}
0&0&3\\
-2&-1&1\\
2&-2&1\end{array}\right).$}
\bad{$AB=  
\left(\begin{array}{rcc}
0&4&2\\
-3&1&1\\-1&1&4\end{array}\right).$}
\good{$BA=  
\left(\begin{array}{rcc}
-1&2&3\\
3&-2&7\\-2&-1&0\end{array}\right).$}
\end{answers}
\begin{explanations} On a : 
$A-B= \left(\begin{array}{rcc}
0&0&3\\
-2&-1&-1\\
2&-2&1\end{array}\right) \quad $ et 
$ \quad AB=  \left(\begin{array}{rcc}
0&4&2\\ -3&1&1\\
-1&1&-4\end{array}\right).$
\end{explanations}
\end{question}
  
\begin{question}
\qtags{motcle=Matrice/Transposée/Somme/Produit}
On considère les matrices : 
$$A=  \left(\begin{array}{rcc}1&2&4\\\end{array}\right) \; , \; 
B=    \left(\begin{array}{r}0\\1\\-1\\
\end{array}\right) \; , \;  C=\left(\begin{array}{rcc}
1&-1&1\\0&0&1\\
\end{array}\right) \quad  \mbox{et} \quad  D= \left(\begin{array}{rcc}1&-1&0\\2&1&1\\0&2&1\\
\end{array}\right).$$
On notera $^tM$ la transposée d'une matrice $M$. Quelles sont les assertions vraies ?
\begin{answers}  
\good{$A+\, ^tB = \left(\begin{array}{rcc}
1&3&3\end{array}\right).$}
\good{$B\, ^tB=\left(\begin{array}{rccc}
0&0&0\\
0&1&-1\\
0&-1&1\end{array}\right).$}
\bad{$A\, ^tC= \left(\begin{array}{rcc}
3&4&0\end{array}\right).$}
\good{$C\, ^tD=  
\left(\begin{array}{rcc}
2&2&-1\\
0&1&1\end{array}\right).$}
\end{answers}
\begin{explanations} On a :  $A\,^tC=  
\left(\begin{array}{rc}3&4\end{array}\right).$
\end{explanations}
\end{question}

\begin{question}
\qtags{motcle=Matrice/Somme/Produit}
Soit $A,B$ et $C$ des matrices d'ordre $n\ge 1$. Quelles sont les assertions vraies ?
\begin{answers}  
\bad{$AB=0 \Rightarrow A=0 \, \mbox{ou} \,  B=0$}.
\bad{$A(BC) =(AC)B$}.
\good{$A(B+C)=AC+AB$}.
\bad{$(A+B)^2=A^2+2AB+B^2$}.
\end{answers}
\begin{explanations} Le produit de deux matrices peut \^etre nul sans que l'une des deux matrices soit nul. Contre-exemple : avec $A=  
\left(\begin{array}{rc}0&0\\
1&0\end{array}\right)$ on a : $A^2=0$. Le produit de deux matrices est associatif et distributif par rapport à l'addition. Comme le produit n'est pas commutatif, en général, $(A+B)^2=A^2+AB+BA+B^2 \neq A^2+2AB+B^2$.
\end{explanations}
\end{question}
 
\begin{question}
\qtags{motcle=Matrice/Somme/Produit/Puissance}
Soit $A=\left(\begin{array}{rc}
1&1\\1&1\end{array}\right) $ et $I=  \left(\begin{array}{rc}
1&0\\0&1\end{array}\right) $, la matrice identité. 
Quelles sont les assertions vraies ?
\begin{answers}  
\good{$A^2=2A$}.
\bad{$A^n=2^nA$, pour tout entier $n \ge 1$}.
\good{$(A-I)^{2n}= I$, pour tout entier $n \ge 1$}.
\bad{$(A-I)^{2n+1}= A+I$, pour tout entier $n \ge 1$}.
\end{answers}
\begin{explanations} On vérifie que pour tout entier $n\ge 1$,
$A^n=2^{n-1}A\, , \, (A-I)^{2n}= I$ et $(A-I)^{2n+1}= A-I$.
\end{explanations}
\end{question}
       
\begin{question}
\qtags{motcle=Matrice/Rang}
On considère les matrices : 
$$A=  \left(\begin{array}{rcc}
1&0&1\\\end{array}\right),\quad B=  
\left(\begin{array}{rcc}2&-4\\
1&-2\\ 0&0\\
\end{array}\right),\quad C=  
\left(\begin{array}{rcc}1&0&0\\1&1&1\\ 0&1&2\\ 
\end{array}\right)\quad \mbox{et}\quad D=  
\left(\begin{array}{rcc}1&0&1\\1&1&0\\ 0&1&-1\\ 
\end{array}\right). $$
Quelles sont les assertions vraies ?
\begin{answers}  
\bad{Le rang de $A$ est $3$}.
\good{Le rang de $B$ est $1$}.
\good{Le rang de $C$ est $3$}.
\bad{Le rang de $D$ est $3$}.
\end{answers}
\begin{explanations} Le rang d'une matrice est le nombre maximum de vecteurs colonnes ou lignes qui sont linéairement indépendants. Le rang de $A$ est $1$, le rang de $B$ est $1$, le rang de $C$ est $3$ et le rang de $D$ est $2$.
\end{explanations}
\end{question} 
 
\begin{question}
\qtags{motcle=Matrice/Base/Dimension}
Soit $E= \Big\{M=\left(\begin{array}{rc}
a&b\\0&a\\ \end{array}\right) \mid a,b \in \Rr \Big\}$. 
Quelles sont les assertions vraies ?
\begin{answers}  
\bad{$E$ n'est pas un espace vectoriel}.
\bad{$E$ est un esapce vectoriel de dimension $1$}.
\bad{$E$ est un esapce vectoriel de dimension $ 4$}.
\good{$E$ est un esapce vectoriel de dimension $ 2$}.
\end{answers}
\begin{explanations} On vérifie que $E$ est un espace vectoriel et que $\left\{\left(\begin{array}{rc}
1&0\\0&1\\ \end{array}\right), \; \left(\begin{array}{rc}
0&1\\0&0\\ \end{array}\right)\right \}$ est une base de $E$. Donc $\dim E = 2$.
\end{explanations}
\end{question}


\begin{question}
\qtags{motcle=Matrice/Base/Dimension}
Soit $E=\Big\{M =\left(\begin{array}{rc}a-b&a-c\\b-c&b-a\end{array}\right)\mid a,b,c\in \Rr\Big\}$. Quelles sont les assertions vraies ?
\begin{answers}  
\bad{$E$ n'est pas un espace vectoriel}.
\bad{$E$ est un espace vectoriel de dimension $3$}.
\good{$E$ est un espace vectoriel de dimension $2$}.
\bad{$E$ est un espace vectoriel de dimension $4$}.
\end{answers}
\begin{explanations} On vérifie que $E$ est un espace vectoriel, que
$$E=\Big\{M =\left(\begin{array}{rc}
\alpha&\beta\\ \beta - \alpha&-\alpha\\ 
\end{array}\right)\mid \alpha, \beta \in \Rr \Big\}$$ 
et que $\left \{\left(\begin{array}{rc} 1&0\\
-1&-1\\ \end{array}\right), \;  \left(\begin{array}{rc}
0&1\\ 1&0\\ \end{array}\right) \right \}$ est  une base de $E$. Donc $\dim E = 2$.
\end{explanations}
\end{question}



\begin{question}
\qtags{motcle=Matrice/Application linéaire/Noyau/Image/Dimension}
Soit $ M_2(\Rr)$ l'ensemble des matrices carrées d'ordre $2$ à coefficients réels  et 
$f$ l'application  définie par :
$$\begin{array}{rccc}f:&M_2(\Rr)&\to& M_2(\Rr)\\
& M = \left(\begin{array}{rc}a&b\\ c&d\\ 
\end{array}\right) &\to &^tM = \left(\begin{array}{rc}
a&c\\b&d\\ \end{array}\right),  \end{array}$$
où $^tM$ est la transposée de $M$. Quelles sont les assertions vraies ?
\begin{answers}  
\good{$f$ est une application linéaire}.
\bad{$\dim \ker f = 1$}.
\good{$\dim \ker f = 0$}.
\bad{$\dim \Im f = 3$}.
\end{answers}
\begin{explanations} On vérifie que $f$ est une application linéaire, $\ker f =\{0\}$ et $\Im f = M_2(\Rr)$. Donc $\dim \ker f = 0$ et $ \dim \Im f = 4$.
\end{explanations}
\end{question}

\subsection{Calcul matriciel | Niveau 3}

\begin{question}
\qtags{motcle=Matrice/Rang}
Soit $A$ une matrice de rang $r$. Quelles sont les assertions vraies ?
\begin{answers}  
\good{$A$ admet $r$ vecteurs colonnes linéairement indépendants}.
\good{$A$ admet $r$ vecteurs lignes linéairement indépendants}.
\bad{Toute famille contenant $r$ vecteurs colonnes de $A$ est libre}.
\bad{Toute famille contenant $r$ vecteurs lignes de $A$ est libre}.
\end{answers}
\begin{explanations} Le rang d'une matrice est le nombre maximum de vecteurs colonnes ou  lignes qui sont linéairement indépendants.
\end{explanations}
\end{question}

\begin{question}
\qtags{motcle=Matrice/Addition/Multiplication}
Soit $E=\Big\{M = \left(\begin{array}{rc}a&b\\
0&a\end{array}\right)\mid a,b\in\Rr \Big\}$. Quelles sont les assertions vraies ?
\begin{answers}  
\good{$E$ est stable par addition}.
\good{$E$ est stable par multiplication de matrices}.
\bad{la multiplication de matrices de $E$ n'est pas commutative}.
\good{Soit $M \in \Rr_2(\Rr)$. Si $MM'=M'M, \; \forall M' \in E$, alors $M\in E$}.
\end{answers}
\begin{explanations} On vérifie que $E$ est stable par addition et par multiplication de matrices et que la multiplication de matrices de $E$ est commutative.
\vskip0mm
Soit $M \in \Rr_2(\Rr)$. On vérifie que si $MM'=M'M,$ pour toute matrice $M'$ de $ E$, alors $M\in E$.
\end{explanations}
\end{question}

\begin{question}
\qtags{motcle=Matrice/Application linéaire/Noyau/Image/Dimension}
Soit $ M_2(\Rr)$ l'ensemble des matrices carrées d'ordre $2$ à coefficients réels et $f$ l'application  définie par :
$$\begin{array}{rccc}f:&M_2(\Rr)&\to& \Rr\\
& M = \left(\begin{array}{rc} a&b\\
c&d\\  \end{array}\right) &\to &\mbox{tr}(M) = a+d, \end{array}$$
le réel $\mbox{tr}(M)$ est appelée la trace de $M$. Quelles sont les assertions vraies ?
\begin{answers}  
\good{$f$ est une application linéaire}.
\good{$\dim \ker f = 3$}.
\bad{$\dim \Im f = 2$}.
\good{$\Im f = \Rr$}.    
\end{answers}
\begin{explanations} On vérifie que $f$ est une application linéaire et que
$$\ker f= \left\{\left(\begin{array}{rc}
a&b\\c&-a\\ \end{array}\right)\; ;\; a,b,c \in \Rr \right\}.$$
Donc la famille $\left\{\left(\begin{array}{rc}
1&0\\ 0&-1\\ \end{array}\right), \;  \left(\begin{array}{rc}
0&1\\ 0&0\\ \end{array}\right), \; \left(\begin{array}{rc}
0&0\\ 1&0\\ \end{array}\right) \right\}$ est une base de $\ker f$ et $\dim \ker f=3$. Or, d'après le théorème du rang, $\dim \Im f=1=\dim \Rr$ et comme $\Im f$ est un sous-espace vectoriel de $\Rr$, donc $\Im f=\Rr$.
\end{explanations}
\end{question}

\begin{question}
\qtags{motcle=Matrice/Application linéaire/Noyau/Image/Dimension}
Soit $ M_2(\Rr)$ l'ensemble des matrices carrées d'ordre $2$ à coefficients réels et 
$f$ l'application  définie par :
$$\begin{array}{rccc}f:&M_2(\Rr)&\to& M_2(\Rr)\\
& M = \left(\begin{array}{rc}
a&b\\c&d\\ \end{array}\right) &\to &M- \, ^t M = 
\left(\begin{array}{rc}0&b-c\\c-b&0\\ 
\end{array}\right),  \end{array}$$
$^tM$ est la transposée de $M$. Quelles sont les assertions vraies ?
\begin{answers}  
\good{$f$ est une application linéaire}.
\good{$\dim \ker f = 3$}.
\bad{$\dim \Im f = 2$}.
\bad{$\dim \Im f = 3$}.
\end{answers}
\begin{explanations} On vérifie que $f$ est une application linéaire, $\ker f=\left\{ \left(\begin{array}{rc}
a&b\\b&d\\ \end{array}\right) \; ; \; a,b,d \in \Rr \right \}$ et  
$\Im f = \left\{ \left(\begin{array}{rc}0&\alpha\\-\alpha&0\\ 
\end{array}\right) \; ; \; \alpha \in \Rr  \right \}$. On  déduit que  $\dim \ker f=3$ et $\dim \Im f = 1$.
\end{explanations}
\end{question}

\subsection{Calcul matriciel | Niveau 4}

\begin{question}
\qtags{motcle=Matrice/Puissance/Formule du binôme}
Soit $a,b\in \Rr$, $A=\left(\begin{array}{rcc}a&1&b\\0&a&2\\ 0&0&a\\ 
\end{array}\right)$ et  $N= A-aI$, où $I= \left(\begin{array}{rcc}
1&0&0\\0&1&0\\ 0&0&1\\ \end{array}\right)$. Quelles sont les assertions vraies ?
\begin{answers}  
\good{$N^k = 0$, pour tout entier $k\ge 3$}.
\bad{On ne peut pas appliquer la formule du binôme pour le calcul de $A^n$}.
\good{Pour tout entier $n \ge 2,$ $\displaystyle A^n =a^nI+na^{n-1}N+\frac{n(n-1)}{2}a^{n-2}N^2 $}.
\good{Pour tout entier $n \ge 2,$ $A^n=\left(\begin{array}{rcc}
a^n&na^{n-1}&na^{n-1}b+n(n-1)a^{n-2}\\0&a^n&2na^{n-1}\\ 0&0&a^n\\ \end{array}\right)$}.
\end{answers}
\vskip2mm
\begin{explanations} On a : $N=\left(\begin{array}{rcc}
0&1&b\\0&0&2\\ 0&0&0\\ \end{array}\right)$,  $N^2=\left(\begin{array}{rcc}0&0&2\\0&0&0\\ 0&0&0\\ 
\end{array}\right)$ et $N^k=0, $ pour tout $k\ge 3$.
\vskip0mm
On a : $A= N+aI$. Comme le produit des matrices $N$ et $aI$ est commutatif, on peut appliquer la formule du binôme pour le calcul des puissances de $A$.
\end{explanations}
\end{question}

\begin{question}
\qtags{motcle=Matrice/Puissance/Suites}
Soit $A= \left(\begin{array}{rcc}1&2&3\\0&1&2\\ 0&0&1\\ 
\end{array}\right)$ et  $N= A-I$, où $I= \left(\begin{array}{rcc}
1&0&0\\0&1&0\\ 0&0&1\\ \end{array}\right).$\\
On considère $3$ suites récurrentes  $(u_n)_{n\ge 0}$, $(v_n)_{n\ge 0}$ et $(w_n)_{n\ge 0}$ définies par $u_0,v_0,w_0$ des réels donnés et pour $n\ge 1$ :
$$(\mathtt{S})\left\{\begin{array}{rcc}
u_n&=&u_{n-1}+2v_{n-1}+3w_{n-1}\\
v_n&=&v_{n-1}+2w_{n-1}\\ w_n&=&w_{n-1}. \\
\end{array}\right.$$
Quelles sont les assertions vraies ?
\begin{answers}  
\bad{$N^k = 0$, pour tout entier $k\ge 2$}.
\good{Pour tout entier $n \ge 2,$ $A^n =I+nN+\frac{n(n-1)}{2}N^2  $}.
\bad{Pour tout entier $n \ge 0,$ 
$$(\mathtt{S}) \left\{\begin{array}{rcc}
u_n&=&u_0+2nv_0+3nw_0\\v_n&=&v_0+2nw_0\\ w_n&=&w_0. \\
\end{array}\right.$$}
\good{Pour tout entier $n \ge 0,$ 
$$(\mathtt{S})  \left\{\begin{array}{rcc}
u_n&=&u_0+2nv_0+n(2n+1)w_0\\v_n&=&v_0+2nw_0\\ w_n&=&w_0. \\
\end{array}\right.$$}
\end{answers}
\begin{explanations} On a : $N=\left(\begin{array}{rcc}
0&2&3\\0&0&2\\ 0&0&0\\ 
\end{array}\right)$, $N^2=\left(\begin{array}{rcc}
0&0&4\\0&0&0\\ 
0&0&0\\ \end{array}\right)$ et $N^k=0, $ pour tout $k\ge 3$.\\
Comme $A= N+I$ et le produit des matrices $N$ et $I$ est commutatif, on peut appliquer la formule du binôme pour le calcul des 
puissances de $A$.
\vskip0mm
En calculant $A^n$ et  utilisant l'égalité : $\left(\begin{array}{r}
u_n\\v_n\\ w_n\\ 
\end{array}\right) = A^n \left(\begin{array}{r}
u_0\\v_0\\ w_0\\ 
\end{array}\right)$, on déduit que : $$\left\{\begin{array}{rcc}
u_n&=&u_0+2nv_0+n(2n+1)w_0\\
v_n&=&v_0+2nw_0\\ 
w_n&=&w_0. \\
\end{array}\right.$$
\end{explanations}
\end{question}

\begin{question}
\qtags{motcle=Matrice/Transposée/Base/Dimension/Espaces supplémentaires}
On note $ M_2(\Rr)$ l'espace des matrices carrées d'ordre $2$ à coefficients réels. Soit 
$$E= \{M \in M_2(\Rr)\mid ^tM = M\} \quad \mbox{et}\quad F= \{M \in M_2(\Rr) \; ; \;  ^tM = -M\},$$
où $^tM$  désigne la transposée de $M$. Quelles sont les assertions vraies ?
\begin{answers}  
\good{$E$ est un espace vectoriel de dimension $3$}.
\bad{$E$ est un espace vectoriel de dimension $2$}.
\good{$F$ est un espace vectoriel de dimension $1$}.
\good{$E$ et $F$ sont supplémentaires dans $ M_2(\Rr)$}.
\end{answers}
\begin{explanations} Soit $M=\left(\begin{array}{rc}
a&b\\ c&d\\ \end{array}\right)$ telle que $^tM = M$, alors $b=c$. Donc la famille
$$\left\{\left(\begin{array}{rc}
1&0\\0&0\\ 
\end{array}\right),\left(\begin{array}{rc}
0&1\\1&0\\ 
\end{array}\right),\left(\begin{array}{rc}
0&0\\0&1\\ \end{array}\right)\right\}$$
forme une base de $E$. D'où $\dim E = 3$.
\vskip0mm
Soit $M=\left(\begin{array}{rc}a&b\\c&d\end{array}\right)$ telle que $ ^tM = -M$, alors $a=d=0$ et $c=-b$. Donc $\left\{\left(\begin{array}{rc}
0&1\\-1&0\\ \end{array}\right)\right\}$ est une base de $F$ et donc $\dim F=1$.
\vskip0mm
On vérifie que $E\cap F=\{0_E\}$ et en utilisant le théorème de la dimension d'une somme, on déduit que $E$ et $F$ sont supplémentaires dans $ M_2(\Rr)$.
\end{explanations}
\end{question}


\begin{question}
\qtags{motcle=Matrice/Famille libre/Génératrice/Base/Dimension}
Dans $M_2(\Rr)$ l'espace vectoriel des matrices carrées d'ordre $2$ à coefficients réels, on considère la famille ${\cal {B'}}= \{ B_1,B_2,B_3,B_4\}$, où 
$$ B_1 = \left(\begin{array}{rc}1&1\\
0&0\\ \end{array}\right) \; , \; B_2 = \left(\begin{array}{rc}
0&1\\0&1\\ 
\end{array}\right) \; , \; B_3 = \left(\begin{array}{rc}
0&0\\1&1\\ 
\end{array}\right) \; ,\; B_4 = \left(\begin{array}{rc}
1&0\\1&0\\ 
\end{array}\right).$$
Quelles sont les assertions vraies ?
\begin{answers}  
\bad{${\cal {B'}}$ est une famille libre de $M_2(\Rr)$}.
\bad{${\cal {B'}}$ est une base de $M_2(\Rr)$}.
\bad{$\mbox{Vect} {\cal {B'}}=M_2(\Rr)$}.
\good{$\dim \mbox{Vect} {\cal {B'}}=3$}.
\end{answers}
\begin{explanations} On vérifie que $B_1+B_3=B_2+B_4$ et que $\{B_1,B_2,B_3\}$ est une famille libre. Donc 
$\dim \mbox{Vect} {\cal {B'}}=3$.
\end{explanations}
\end{question}

\begin{question}
\qtags{motcle=Matrice/Application linéaire/Noyau/Image/injection/Surjection}
On considère $M_2(\Rr)$ l'ensemble des matrices carrées d'ordre $2$ à coefficients réels,  \\
$ A= \left(\begin{array}{rcc}0&1\\1&0 \end{array}\right)$ et 
$f$ l' application linéaire définie par : 
$$\begin{array}{rccc}f:&M_2(\Rr)&\to&M_2(\Rr)\\
& M&\to &AM-MA.  \end{array}$$
Quelles sont les assertions vraies ?
\begin{answers}  
\good{$\dim \ker f =2$}.
\bad{$f$ est injective}.
\good{$\mbox{rg} (f) =2$}.
\bad{$f$ est surjective}.
\end{answers}
\begin{explanations} On vérifie que pour $M=\left(\begin{array}{rc}
a&b\\c&d\\ \end{array}\right) $, $f(M)=\left(\begin{array}{rc}
c-b&d-a\\a-d&b-c\\ \end{array}\right)$. Par conséquent,
$\ker f = \left\{ \left(\begin{array}{rc}
a&b\\b&a\\ \end{array}\right) \; ; \; a,b \in \Rr \right \}$ et 
$\Im f = \left\{ \left(\begin{array}{rc}
\alpha&\beta\\-\beta&-\alpha\\ 
\end{array}\right) \; ; \; \alpha, \beta \in \Rr  \right \}$. On  déduit que $\dim \ker f = 2$,  $\mbox{rg} (f)=\dim \Im f = 2$ et que $f$ n'est ni injective, ni surjective.
\end{explanations}
\end{question}

\subsection{Inverse d'une matrice | Niveau 1}

\begin{question}
\qtags{motcle=Matrice/Inversibilité}
Soit $A$ une matrice carrée d'ordre $n$  à coefficients réels et $I$ la matrice identité. Quelles sont les assertions vraies ?
\begin{answers}  
\good{$A$ est inversible si et seulement s'il existe une matrice $B$ telle que $AB=I$}.
\good{$A$ est inversible si et seulement s'il existe une matrice $B$ telle que $BA=I$}.
\bad{$A$ est inversible si et seulement si les coefficients de $A$ sont inversibles pour la multiplication dans $\Rr$}.
\good{$A$ est inversible si et seulement si pour toute matrice $Y$ à une colonne et $n$ lignes, il existe une matrice $X$ à une colonne et $n$ lignes telle que $AX=Y$}.
\end{answers}
\begin{explanations} Les propositions suivantes sont équivalentes :
\begin{enumerate}
\item[(i)] $A$ est inversible.
\item[(ii)] Il existe une matrice $B$ telle que $AB=BA=I$.
\item[(iii)] Il existe une matrice $B$ telle que $AB=I$.
\item[(iv)] Il existe une matrice $B$ telle que $BA=I$.
\item[(v)] Pour toute matrice $Y$ à une colonne et $n$ lignes, il existe une matrice $X$ à une colonne et $n$ lignes telle que $AX=Y.$
\end{enumerate}
\end{explanations}
\end{question}

\subsection{Inverse d'une matrice | Niveau 2}

\begin{question}
\qtags{motcle=Matrice/Inversibilité}
On considère les matrices 
$$A = \left(\begin{array}{rc}
1&2\\3&5\\ \end{array}\right),\quad B = \left(\begin{array}{rcc}
1&1&1\\2&0&1\\ 1&1&-1\\ \end{array}\right),\quad
C = \frac{1}{4}\left(\begin{array}{rcc}
-1&2&1\\3&-2&1\\ 2&0&2\\ \end{array}\right).$$
Quelles sont les assertions vraies ?
\begin{answers}  
\good{$A$ est inversible}.
\good{$B$ est inversible}.
\bad{$B$ est inversible et $B^{-1} = C$}.
\bad{$C$ est inversible}.   
\end{answers}
\begin{explanations} On vérifie que $A$ et $B$ sont inversibles,  que $\displaystyle B^{-1} = \frac{1}{4}\left(\begin{array}{rcc}-1&2&1\\ 3&-2&1\\ 2&0&-2\\ \end{array}\right) \neq C$ et que $C$ n'est pas inversible.
\end{explanations}
\end{question}

\begin{question}
\qtags{motcle=Matrice/Inversibilité}
On considère les matrices :
$$A=\left(\begin{array}{r} 5 \end{array}\right),\quad B = 
\left(\begin{array}{rc}1&-2\\ 2&-4 \end{array}\right),\quad C = 
\left(\begin{array}{rcc} 1&1&1\\ 1&0&-1\\  1&1&0
\end{array}\right),\quad D =\left(\begin{array}{rcc} -1&1&-2\\ 1&1&0\\ 2&-1&3 \end{array}\right).$$
Quelles sont les assertions vraies ?
\begin{answers}  
\good{$A$ est inversible}.
\bad{$B$ est inversible}.
\good{$C$ est inversible}.
\bad{$D$ est inversible}.
\end{answers}
\begin{explanations} $A$ est inversible et $A^{-1}=\left(\begin{array}{r}\displaystyle \frac{1}{5}\end{array}\right)$. $B$ n'est pas inversible, puisque les deux vecteurs colonnes sont proportionnels. $C$ est inversible et 
$$C^{-1} = \left(\begin{array}{rcc} 1&1&-1\\-1&-1&2\\ 1&0&-1
\end{array}\right).$$
$D$ n'est pas inversible, puisque les trois vecteurs colonnes sont linéairement dépendants.
\end{explanations}
\end{question}

\begin{question}
\qtags{motcle=Matrice/Transposée/Inversibilité}
Soit $ A$ une matrice inversible. On notera  $^tA$ la transposée de $A$. Quelles sont les assertions vraies ?
\begin{answers}  
\good{$3A$ est inversible}.
\good{$^tA$ est inversible}.
\good{$A^tA$ est inversible}.
\bad{$A+^tA$ est inversible}.
\end{answers}
\begin{explanations} Comme $A$ est inversible, il existe une matrice $B$ telle que $AB= BA=I$, où $I$ est la matrice identité. On en déduit : $3A$ est inversible et son inverse est $\displaystyle \frac{1}{3}B$.
\vskip0mm
$^tA$ est inversible et son inverse est $ ^tB$. En effet, $I=\, ^t (AB)=\, ^tB\, ^tA$.
\vskip0mm
$A^tA$ est inversible et son inverse est $^tBB$. En effet,  $(^tBB)A^tA=\, ^tB(BA)^tA= \, ^tB^tA=\, ^t (AB)=I$.
\vskip0mm
$A+^tA$ n'est pas nécessairement inversible. Contre exemple : 
$A=\left(\begin{array}{rc} 1&1\\ -1&0 \end{array}\right).$
\end{explanations}
\end{question}

\begin{question}
\qtags{motcle=Matrice/Inversibilité/Rang}
Soit $ M_n(\Rr)$ l'ensemble des matrices carrées d'ordre $n$ à coefficients réels et $I$ la matrice identité. Soit $A \in M_n(\Rr)$ telle qu'il existe un entier $m\ge 1$ vérifiant $A^m = I$. Quelles sont les assertions vraies ?
\begin{answers}  
\good{$A$ est inversible et $A^{-1} = A^{m-1}$}.
\good{Le rang de $A$ est $n$}.
\bad{$A$ n'est pas inversible}.
\good{Si $m=2$, $A$ est inversible et $A^{-1} = A$}.
\end{answers}
\begin{explanations} On a : $A \times A^{m-1} = I$, donc $A$ est inversible et $A^{-1} = A^{m-1}$. Comme $A$ est inversible, le rang de $A$ est $n$. Si $m=2$, alors $A^{-1}=A$.
\end{explanations}
\end{question}

\subsection{Inverse d'une matrice | Niveau 3}

\begin{question}
\qtags{motcle=Matrice/Inversibilité}
On considère la matrice : $A = \left(\begin{array}{rcc}
1&-1&-2\\0&1&1\\ -1&1&2\\ \end{array}\right)$. Quelles sont les assertions vraies ?
\begin{answers}  
\bad{$A$ est inversible}.
\bad{$A^2$ est inversible}.
\bad{$A^3+A^2$ est inversible}.
\good{$A+\,^tA$ est inversible, où $^tA$ est la transposée de $A$}.
\end{answers}
\begin{explanations} $A$ n'est pas inversible, puisque les trois vesteurs colonnes sont linéairement dépendants.
\vskip0mm
Si $A^2$ est inversible, alors il existe une matrice  $B$ telle que $A^2B=I$, donc $A(AB)=I$ et donc $A$ est inversible,
ce qui est absurde.
\vskip0mm
Si $A^3+A^2$ est inversible, alors il existe une matrice $B$ telle que $(A^3+A^2)B=I$. On en déduit que $A[(A^2+A)B]=I$ et donc $A$ est inversible, ce qui est absurde.
\vskip0mm
$A+\, ^tA$ est inversible, puisque les vecteurs colonnes de cette matrice sont linéairement indépendants.
\end{explanations}
\end{question}

\begin{question}
\qtags{motcle=Matrice/Inversibilité}
Soit $ A=(a_{i,j})$ une matrice carrée. On rappelle les définitions suivantes :
\begin{enumerate}
\item[.] $A$ est dite diagonale si tous les coefficients $a_{i,j}$, avec $i\neq j$ sont nuls.
\item[.] $A$ est dite symétrique si pour tous $i,j$, $a_{i,j}=a_{j,i}$.
\item[.] $A$ est dite triangulaire inférieurement (resp. supérieurement) si pour tous $i<j$, $a_{i,j}=0$ (resp. pour tous $i>j$, $a_{i,j}=0$).
\end{enumerate}
Quelles sont les assertions vraies ?
\begin{answers}  
\bad{Si $A$ est diagonale, $A$ est inversible si et seulement s'il existe un coefficient $a_{i,i}$ non nul}.
\good{Si $A$ est diagonale, $A$ est inversible si et seulement si tous les coefficients $a_{i,i}$ sont non nuls}.
\good{$A$ est symétrique si $\, ^tA=A$, où $^tA$ est la transposée de $A$}.
\bad{Si $A$ est triangulaire inférieurement, $A$ est inversible}.
\end{answers}
\begin{explanations} $A$ est inversible si et seulement si les vecteurs colonnes sont linéairement indépendants. On en déduit que 
si $A$ est diagonale, $A$ est inversible si et seulement si tous les coefficients $a_{i,i}$ sont non nuls et que si $A$ est triangulaire (inférieurement ou supérieurement), $A$ est inversible si et seulement si tous les coefficients $a_{i,i}$ sont non nuls. Par définition, $A$ est symétrique si $\, ^tA=A$.
\end{explanations}
\end{question}

\begin{question}
\qtags{motcle=Matrice/Transposée/Inversibilité/Rang}
Soit $A$ et $B$ deux matrices carrées d'ordre $n\ge 1$. On notera $^tA$ la transposée de $A$ et $\mbox{rg}\, (A)$ le rang de $A$. 
Quelles sont les assertions vraies ?
\begin{answers}  
\good{$\mbox{rg}(A)=\mbox{rg}( \, ^tA)$}.
\good{Si $A$ est inversible, $\mbox{rg}(A)=\mbox{rg}(A^{-1})$}.
\bad{$\mbox{rg}(A+B)=\max \big(\mbox{rg}(A), \mbox{rg}(B)\big)$}.
\bad{$\mbox{rg}(AB)= \mbox{rg}(BA)$}.   
\end{answers}
\begin{explanations} Le rang d'une matrice est le nombre maximum de vecteurs colonnes ou lignes qui sont linéairement indépendants. On en déduit que $\mbox{rg}(A)=\mbox{rg}(\, ^tA)$ et que, si $A$ est inversible, $\mbox{rg}(A)=\mbox{rg}(A^{-1})=n$.
\vskip0mm
En général, $\mbox{rg}(A+B)\neq \max \big(\mbox{rg}(A), \mbox{rg}(B)\big)$ et $\mbox{rg}(AB)\neq \mbox{rg}(BA)$. Contre-exemple : avec 
$$A = \left(\begin{array}{rc}
1&-1\\-1&1\\ \end{array}\right),\quad B = \left(\begin{array}{rc}
-1&1\\ 1&-1\\ \end{array}\right),\quad C = 
\left(\begin{array}{rc} 1&2\\1&2\\ \end{array}\right),$$
on vérifie que :  $A+B =0$, $AC =0$ et $CA=
\left(\begin{array}{rc}-1&1\\
-1&1\\ \end{array}\right)$. Donc $\mbox{rg}(A)=\mbox{rg}(B)=1$ mais $\mbox{rg}(A+B)=0$ et $\mbox{rg}(AC)= 0$ est différent de $\mbox{rg}(CA)=1$.
\end{explanations}
\end{question}

\begin{question}
\qtags{motcle=Matrice/Inversibilité}
On considère $ M_n(\Rr)$ l'ensemble des matrices carrées d'ordre $n$  à coefficients réels et $A$ et $B$ deux matrices non nulles telles que $AB=0$. Quelles sont les assertions vraies ?
\begin{answers}  
\bad{$A=0$ ou $B=0$}.
\bad{$A$ est inversible}.
\bad{$B$ est inversible}.
\good{$A$ n'est pas inversible}.
\end{answers}
\begin{explanations} Si $A$ est inversible, alors il existe une matrice $C$ telle que $CA=I$, où $I$ est la matrice identité. Donc  $(CA)B =B$. Or $(CA)B=C(AB)$ et $AB=0$, donc $B=0$, ce qui est absurde.
\end{explanations}
\end{question}

\begin{question}
\qtags{motcle=Matrice/Inversibilité}
On considère $ M_n(\Rr)$ l'ensemble des matrices carrées d'ordre $n$ à coefficients réels et $A$, $B$ et $C$ trois matrices non nulles deux à deux distinctes telles que $AB=AC$. Quelles sont les assertions vraies ?
\begin{answers}  
\bad{$B=C$}.
\bad{$A=0$}.
\good{$A$ n'est pas inversible}.
\bad{Le rang de $A$ est $n$}.
\end{answers}
\begin{explanations} On a : $A(B-C)=0$, $A \neq 0$ et $B-C\neq 0$ (le produit de deux matrices peut être nul sans que les 
deux matrices soient nulles).
\vskip0mm
Si $A$ est inversible, alors il existe une matrice $D$ telle que $DA=I$, où $I$ est la matrice identité. Donc  $(DA)(B-C) =B-C$. Or $(DA)(B-C)=D(A(B-C))$ et $A(B-C)=0$, donc $B-C=0$, ce qui est absurde. On déduit que $A$ n'est pas inversible et donc le rang de $A$ est $<n$.
\end{explanations}
\end{question}

\subsection{Inverse d'une matrice | Niveau 4}

\begin{question}
\qtags{motcle=Matrice/Puissance/Inversibilité/Rang}
On considère la matrice $A = \left(\begin{array}{rc}
\cos x&-\sin x\\\sin x&\cos x\\ \end{array}\right) \, , \; x \in \Rr$. Quelles sont les assertions vraies ?
\begin{answers}  
\bad{Le rang de $A$ est $1$}.
\good{$A$ est inversible et $A^{-1} = \left(\begin{array}{rc}
\cos x&\sin x\\-\sin x&\cos x\\ \end{array}\right)$, $x\in \Rr$}.
\good{Pour tout $n \in \Nn$, $(A+A^{-1})^n = (2^n\cos^n x) I$, où $I$ est la matrice identité}.
\good{Pour tout $n \in \Zz, $ $A^n =  \left(\begin{array}{rc}
\cos (nx)&-\sin (nx)\\\sin (nx)&\cos (nx)\\ \end{array}\right)$}.
\end{answers}
\begin{explanations} On vérifie que $A$ est inversible et que $A^{-1} = \left(\begin{array}{rc}\cos x&\sin x\\-\sin x&\cos x\\ 
\end{array}\right)$. le rang de $A$ est donc $2$.
\vskip0mm
De l'égalité $A+A^{-1}=(2\cos x) I$, on déduit que $(A+A^{-1})^n = (2^n\cos^n x) I$, pour tout entier $n$.
\vskip0mm
Par récurrence sur $n\in \Nn$, on démontre que 
$$A^n =  \left(\begin{array}{rc}\cos (nx)&-\sin (nx)\\
\sin (nx)&\cos (nx) \end{array}\right)\quad \mbox{et}\quad (A^{-1})^n =  \left(\begin{array}{rc}\cos (nx)&\sin (nx)\\ -\sin (nx)&\cos (nx)\end{array}\right).$$
On déduit que, pour tout $n \in \Zz$, $A^n=\left(\begin{array}{rc}\cos (nx)&-\sin (nx)\\\sin (nx)&\cos (nx)\end{array}\right)$.
\end{explanations}
\end{question}

\begin{question}
\qtags{motcle=Matrice/Inversibilité/Rang}
Soit $ M_n(\Rr)$ l'ensemble des matrices carrées d'ordre $n$  à coefficients réels  et $I$ la matrice identité.
Soit $A \in M_n(\Rr)$ telle qu'il existe un entier  $m \ge 1$ vérifiant :  $A^m+A^{m-1}+ \dots + A + I = 0$. Quelles sont les assertions vraies ?
\begin{answers}  
\good{$A$ est inversible et $A^{-1} = A^m$}.
\good{$A$ est inversible et $A^{-1} = -(A^{m-1}+ \dots + A+I)$}.
\good{Le rang de $A$ est $n$}.
\bad{$A$ n'est pas inversible}.
\end{answers}
\begin{explanations} De l'égalité : $A \times  (A^{m-1}+A^{m-2}+ \dots + I)= -I$, on déduit que  $A$ est inversible et que $A^{-1}=-(A^{m-1}+ \dots + A+I)=A^m$. Puisque $A$ est inversible, le rang de $A$ est $n$.
\end{explanations}
\end{question}

\begin{question}
\qtags{motcle=Matrice nilpotente/Inversibilité}
Soit $A$ une matrice nilpotente, c.à.d il existe un entier $n\ge 1$ tel que $A^n=0$. On notera $I$ la matrice identité. Quelles sont les assertions vraies ?
\begin{answers}  
\bad{$A$ est inversible}.
\bad{$A$ est inversible et $A^{-1} = A^{n-1}$}.
\good{Il existe $a\in \Rr$, tel que $A-aI$ n'est pas inversible}.
\good{Pour tout $a\in \Rr^*$, $A-aI$ est inversible}.   
\end{answers}
\begin{explanations} On suppose que $A$ est inversible, alors $A$ est non nul et il existe une matrice $C$ telle que $AC=I$. 
Soit $m$ le plus petit entier $\ge 1$ tel que $A^m=0$. Alors, $0=A^mC=A^{m-1}(AC)=A^{m-1}$, ce qui est absurde.
Par conséquent, $A$ n'est pas inversible.\\
Comme $A$ n'est pas inversible, pour $a=0$, $A-aI$ n'est pas inversible.\\
Soit $a\in \Rr^*$, de l'égalité : $(A-aI)(a^{n-1}I+a^{n-2}A+\dots + aA^{n-2}+A^{n-1})=A^n-a^nI=-a^nI$, on déduit que 
$A-aId$ est inversible et que 
$\displaystyle (A-aId)^{-1}=-\frac{1}{a^n}\left(a^{n-1}I+a^{n-2}A+\dots + aA^{n-2}+A^{n-1}\right)$.
\end{explanations}
\end{question}
