 %%%%%%%%%%%%%%%%%% PREAMBULE %%%%%%%%%%%%%%%%%%

\documentclass[12pt,a4paper]{article}

\usepackage[francais]{exo7qcm}
%\usepackage[francais,nosolutions]{exo7qcm}

\begin{document}

 
 
%%%%%%%%%%%%%%%%%% ENTETE %%%%%%%%%%%%%%%%%%

\LogoExoSept{2}

%\kern-2em
\hfill\textsf{Ann\'ee 2017}

\vspace*{0.5ex}
\hrule\vspace*{1.5ex} 
\hfil\textsf{\textbf{\Large QCM de mathématiques}}
\vspace*{1ex} \hrule 
\vspace*{5ex} 


\section{Pas de feedback}


\begin{question}

Combien font $2^{10}$ ? 

\begin{answers}  
    \bad{1000}
    \good {1024}
    \bad{2048}
    \bad{$\int 10^{10} \%$}
\end{answers}
\end{question}



\section{Avec feedback}

\begin{question}

Voici une question avec un feedback pour chaque réponse.

Qu'est-ce qu'un kilo-octet ?


\begin{answers}
	\bad{Un octet qui a pris trop de poids. 
	\feedback{Vous avez le sens de l'humour !}
	}
	
    \good{$1000$ octets.    
    \feedback{Oui ! Dans le système international le préfixe "kilo"     
    signifie $1000$.}   
    }

    \bad{$1024$ octets.\feedback{Non ! $1024$ octets s'appelle un "kibioctet", abrégé en kio. Blabla $2^{10} = 1024$}.}
\end{answers}

\begin{explanations}
Et voici l'explication globale...
\end{explanations}

\end{question}


%\section{Test export yaml, puis export tex}

%\input{export-feedback.tex}

\end{document}
